\documentclass{lmcs} %%% last changed 2014-08-20

%% optional lists of keywords
\keywords{optional comma separated list of keywords}

%% read in additional TeX-packages or personal macros here:
%% e.g. \usepackage{tikz}
\usepackage{hyperref}
\usepackage{wrapfig}
 \usepackage{mathtools}
\usepackage{sml-highlight}
\usepackage{listings}
\usepackage{xcolor}
\usepackage{epstopdf}
\usepackage{inputenc}
\usepackage{comment}
\usepackage{url}
\usepackage{graphicx} 
\graphicspath{{Figures/}}
\usepackage{proof}
\usepackage{todonotes}
\usepackage{amssymb}
\usepackage{tikz}
\usepackage{amsthm}
\usetikzlibrary{automata, positioning, arrows}
\tikzset{every picture/.style={line width=0.5pt}} %set default line width to 0.75pt        

%%% Random Macros

\newcommand{\sem}[1]{\{\![#1]\!\}}
\newcommand{\eval}[2]{#1\!\!\downarrow_{#2}}

\newcommand{\mypar}[1]{\noindent{\bf #1}}

\newcommand{\state}{\mathsf{state}}

\newcommand{\proj}{\mathsf{proj}}
\newcommand{\projE}[2]{#1\downarrow_#2}

\newcommand{\marco}[1]{\textcolor{cyan}{{\footnotesize Marco: #1}}}


% Syntax
\newcommand{\interactBase}[2]{\role{#1}\rightarrow \{\role{#2}\}}
\newcommand{\interact}[2]{\interactBase{#1}{#2}\,\Sigma_{j\in J}\lambda_j: u_j.\ C_j}
\newcommand{\doubleand}{\text{\&\&}}

\newcommand{\command}[4] {\commandBase {#1} {#2} \Sigma_{i\in I}#3_i: #4_i}
\newcommand{\commandBase}[3] {[#1]\ #2\ \rightarrow #3}
\newcommand{\ifTE}[4]{\textsf{if } #1@\role {#2} \textsf{ then } {#3} \textsf{ else } {#4}}


\newcommand{\role}[1]{\mathsf{#1}}
\newcommand{\Var}{\mathsf{Var}}
\newcommand{\Val}{\mathsf{Val}}


\newcommand{\CEnd}{\mathbf{0}}
\newcommand{\pp}{|\!|}
\newcommand{\ppp}[1]{|[#1]|}
\newcommand{\cmd}[6]{[#1] #2 \rightarrow \{#3: #4 = #5\}_{#6}}



  
% Semantics
\newcommand{\red}[1]{\ \longrightarrow_{#1}\ } 
\newcommand{\prismred}[1]{\ \stackrel{#1}{\rightsquigarrow}\ } 
\newcommand{\lab}[3]{\ [#1][#2:\;#3]\ } 

\definecolor{pblue}{rgb}{0.13,0.13,1}
\definecolor{pgreen}{rgb}{0,0.5,0}
\definecolor{pred}{rgb}{0.9,0,0}
\definecolor{pgrey}{rgb}{0.46,0.45,0.48}
\definecolor{keywordColor}{rgb}{0.50,0.00,0.33}
\definecolor{stringColor}{rgb}{0.16,0.00,1.00}
\definecolor{lineNumberColor}{rgb}{0.47,0.47,0.47}

\lstdefinelanguage{Eclipse}{
language=Java,
tabsize=3,
showspaces=false,
captionpos=b,
breaklines=true,
extendedchars=true,
numbers=none,
postbreak=\raisebox{0ex}[0ex][0ex]{\ensuremath{\color{red}\hookrightarrow\space}},
  basicstyle=\ttfamily\tiny,
  emphstyle=\bfseries,
  keywordstyle=\color{keywordColor}\bfseries,
  commentstyle=\markupComments,
  stringstyle=\color{stringColor},
  numberstyle=\color{lineNumberColor}\tiny,
  morecomment=[s][\markupJavadocs]{/**}{*/}, % For Javadoc comments
  showstringspaces=false,
  morekeywords={and},
  numbers=left,
  mathescape=true
%  ,frame=lines%shadowbox%trBL
}

\definecolor{carminepink}{rgb}{0.92, 0.3, 0.26}
\definecolor{jesuscolour}{rgb}{0.3, 0.92, 0.26}
\newcommand{\AV}[1]{\todo[inline, color=carminepink]{\textbf{AV}:~#1}}
\newcommand{\MC}[1]{\todo[inline, color=jesuscolour]{\textbf{MC}:~#1}}

%%% Local Variables: 
%%% mode: latex
%%% TeX-master: "main"
%%% End:

%% define non-standard environments BEYOND the ones already supplied
%% here, for example
\theoremstyle{plain}\newtheorem{satz}[thm]{Satz} %\crefname{satz}{Satz}{S\"atze}
%% Do NOT replace the proclamation environments lready provided by
%% your own.

\newtheorem{lemma}{Lemma}
\newtheorem{example}{Example}
\newtheorem{definition}{Definition}
\newtheorem{theorem}{Theorem}
\newtheorem{remark}{Remark}
\newtheorem{assumption}{Assumption}

\def\eg{{\em e.g.}}
\def\cf{{\em cf.}}

%% due to the dependence on amsart.cls, \begin{document} has to occur
%% BEFORE the title and author information:

\begin{document}

% If the title is longer than 55 characters, then specify a shorter running title as the optional argument to \title. The running title should be roughyl at most 55 characters:
\title[A Probabilistic Choreography Language for PRISM]
{A Probabilistic Choreography Language for PRISM}
% \titlecomment{{\lsuper*}OPTIONAL comment concerning the title, \eg, if
%   a variant or an extended abstract of the paper has appeared
%   elsewhere.}  \thanks{thanks, optional.} %optional

% affiliations are numbered automatically with a, b, c (see below)
% use the optional argument to indicate the affiliation(s) of each author
% omit the argument if there is only one author, or only one affiliation
\author[M.~Carbone]{Marco Carbone\lmcsorcid{0000-0001-9479-2632}}[a]
\author[A.~Veschetti]{Adele Veschetti\lmcsorcid{0000-0002-0403-1889}}[b]

% affiliation 1 (automatically numbered a)
\address{Computer Science Department, IT University of Copenhagen, Rued Langgaards Vej 7, 2300 Copenhagen S, Denmark}	%optional
% write emails for all authors having that affiliation
\email{carbonem@itu.dk, maca@itu.dk}  %optional

% affiliation 2 (automatically numbered b)
\address{Department of Computer Science, TU Darmstadt, Hochschulstraße 10, 64289 Darmstadt, Germany} %optional
\email{adele.veschetti@tu-darmstadt.de}  %optional

%% etc.

%% required for running head on odd and even pages, use suitable
%% abbreviations in case of long titles and many authors:

%%%%%%%%%%%%%%%%%%%%%%%%%%%%%%%%%%%%%%%%%%%%%%%%%%%%%%%%%%%%%%%%%%%%%%%%%%%

%% the abstract has to PRECEDE the command \maketitle:
%% be sure not to issue the \maketitle command twice!

\begin{abstract}
  \noindent We present a choreographic framework for modelling and
  analysing concurrent probabilistic systems based on the PRISM
  model-checker. This is achieved through the development of a
  choreography language, which is a specification language that allows
  to describe the desired interactions within a concurrent system from
  a global viewpoint. Using choreographies gives a clear and complete
  view of system interactions, making it easier to understand the
  process flow and identify potential errors, which helps ensure
  correct execution and improves system reliability. We equip our
  language with a probabilistic semantics and then define a formal
  encoding into the PRISM language and discuss its
  correctness. Properties of programs written in our choreographic
  language can be model-checked by the PRISM model-checker via their
  translation into the PRISM language.  Finally, we implement a
  compiler for our language and demonstrate its practical
  applicability via examples drawn from the use cases featured in the
  PRISM website.
\end{abstract}

\maketitle

\section{Introduction}\label{sec:intro}
\input introduction

\section{A Motivating Example}\label{sec:example}
\input example

\section{Choreography Language}\label{sec:chor}
\input chor

\section{The PRISM Language}\label{sec:prism}
\input prism

\section{Projection}\label{sec:proj}
\input proj

\section{Implementation}\label{sec:impl}
\input implementation
\section{Benchmarking}\label{sec:bench}
\input tests

\section{Related Work and Discussion}
\input discussion

\section*{Acknowledgment}
  \noindent The authors wish to acknowledge fruitful discussions with
  A and B.

%\bibliographystyle{alphaurl}
\bibliographystyle{plain}
\bibliography{biblio}

\end{document}
