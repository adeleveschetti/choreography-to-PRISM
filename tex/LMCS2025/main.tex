\documentclass{lmcs} %%% last changed 2014-08-20

%% optional lists of keywords
\keywords{optional comma separated list of keywords}

%% read in additional TeX-packages or personal macros here:
%% e.g. \usepackage{tikz}
\usepackage{hyperref}
%%\input{myMacros.tex}
%% define non-standard environments BEYOND the ones already supplied
%% here, for example
\theoremstyle{plain}\newtheorem{satz}[thm]{Satz} %\crefname{satz}{Satz}{S\"atze}
%% Do NOT replace the proclamation environments lready provided by
%% your own.

\def\eg{{\em e.g.}}
\def\cf{{\em cf.}}

%% due to the dependence on amsart.cls, \begin{document} has to occur
%% BEFORE the title and author information:

\begin{document}

% If the title is longer than 55 characters, then specify a shorter running title as the optional argument to \title. The running title should be roughyl at most 55 characters:
\title[Instructions]{A Probabilistic Choreography Language for PRISM}
% \titlecomment{{\lsuper*}OPTIONAL comment concerning the title, \eg, if
%   a variant or an extended abstract of the paper has appeared
%   elsewhere.}  \thanks{thanks, optional.} %optional

% affiliations are numbered automatically with a, b, c (see below)
% use the optional argument to indicate the affiliation(s) of each author
% omit the argument if there is only one author, or only one affiliation
\author[M.~Carbone]{Marco Carbone\lmcsorcid{0000-0001-9479-2632}}[a]
\author[A.~Veschetti]{Adele Veschetti\lmcsorcid{0000-0002-0403-1889}}[b]

% affiliation 1 (automatically numbered a)
\address{University 1, address1}	%optional
% write emails for all authors having that affiliation
\email{name1@email1, name2@email1, name3@email1}  %optional

% affiliation 2 (automatically numbered b)
\address{University 2, address2}	%optional
\email{name2@email2}  %optional

%% etc.

%% required for running head on odd and even pages, use suitable
%% abbreviations in case of long titles and many authors:

%%%%%%%%%%%%%%%%%%%%%%%%%%%%%%%%%%%%%%%%%%%%%%%%%%%%%%%%%%%%%%%%%%%%%%%%%%%

%% the abstract has to PRECEDE the command \maketitle:
%% be sure not to issue the \maketitle command twice!

\begin{abstract}
  \noindent We present a choreographic framework for modelling and
  analysing concurrent probabilistic systems based on the PRISM
  model-checker. This is achieved through the development of a
  choreography language, which is a specification language that allows
  to describe the desired interactions within a concurrent system from
  a global viewpoint. Employing choreographies provides a clear and
  comprehensive view of system interactions, enabling the discernment
  of process flow and detection of potential errors, thus ensuring
  accurate execution and enhancing system reliability. We equip our
  language with a probabilistic semantics and then define a formal
  encoding into the PRISM language and discuss its
  correctness. Properties of programs written in our choreographic
  language can be model-checked by the PRISM model-checker via their
  translation into the PRISM language.  Finally, we implement a
  compiler for our language and demonstrate its practical
  applicability via examples drawn from the use cases featured in the
  PRISM website.
\end{abstract}

\maketitle

\section{Introduction}\label{sec:intro}
\input introduction

%\section{A Motivational Example}
%\input example

\section{Choreography Language}\label{sec:chor}
\input chor

\section{The PRISM Language}\label{sec:prism}
\input prism

\section{Projection}\label{sec:proj}
\input proj

\section{Implementation}\label{sec:impl}
\input implementation
\section{Benchmarking}\label{sec:bench}
\input tests

\section{Related Work and Discussion}
\input discussion

\section*{Acknowledgment}
  \noindent The authors wish to acknowledge fruitful discussions with
  A and B.

  %% the following bibliography is gererated manually for the sake of brevity
  %% only; please use a separate .bib file in your submission

\bibliographystyle{alphaurl}
\bibliography{biblio}

\end{document}
