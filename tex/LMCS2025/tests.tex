{In this section, we present an experimental evaluation of our
  language.  The examples provided highlight two main points: firstly,
  the representation using choreographic language is significantly
  more concise; secondly, we demonstrate that PRISM behaves similarly
  on both the projection and the original model in PRISM also in our
  implementation. }
%In order to show that using choreographies is beneficial in the
%presence of concurrency, we included an example modelling a single
%process.
 % We have decided to provide two examples with concurrency and one
% without .  
In particular, we focus on three benchmarks: a modified version of the
example reported in Section \ref{sec:intro}; the Bitcoin Proof of Work
protocol~\cite{DBLP:journals/concurrency/BistarelliNGLMV23} and the
Hybrid Casper
protocol~\cite{DBLP:journals/distribledger/GallettaLMV23}.  The
generated PRISM files can be found in our online repository
\cite{repository}.

\begin{comment}
\subsection{The Dice Program}
The first test case is the Dice Program \cite{KY76}, a model involving
only a single process.  The following program simulates a die using
only unbiased coins. Starting from the initial vertex (state $s_0$),
the process involves the iterative flipping of a coin. Upon obtaining
heads, the upper branch is chosen, and in the case of tails, the lower
branch is taken. This sequential process persists until the final
determination of the die's value.  In particular, the PRISM model uses
two variables: \codeprism{d} which represents the value of the dice,
ranging from 0 to 6; \codeprism{STATE} which is the state variable,
ranging from 0 to 7.  Initially, both variables are set to 0.
\vspace{-0.05cm}
 \begin{lstlisting}[style=chor-color,caption={Choreography for the Dice Program},captionpos=b,label={ex1-code}]
	{DiceProtocol0 $\coloneqq$  Dice $\rightarrow$ Dice : (+["0.5*1"] ; DiceProtocol1
						 +["0.5*1"] ;  DiceProtocol2)
	
	DiceProtocol1 $\coloneqq$ Dice $\rightarrow$ Dice : (+["0.5*1"] ; Dice $\rightarrow$ Dice : 
							(+["0.5*1"]  ; DiceProtocol1
							 +["0.5*1"]  "(d'=1)" ; DiceProtocol7)
					     +["0.5*1"] ;  Dice $\rightarrow$ Dice : 
							(+["0.5*1"]  "(d'=2)" ; DiceProtocol7
							 +["0.5*1"]  "(d'=3)" ; DiceProtocol7))
	$\ldots$
	DiceProtocol7 $\coloneqq$ Dice $\rightarrow$ Dice : (["1*1"] ; END)}
	\end{lstlisting}
   \vspace{-0.15cm}


As expected, since there is no concurrency, there are no significant improvements to the code in using choreographies.
The PRISM model produced by our method closely aligns with the original model with only minimal details that differ between the two.
Additionally, to ensure the generated program's accuracy, 
we evaluated whether the probability of reaching a state where the dice displays \texttt{d=k} for every possible \texttt{k} from 1 to 6 equals 1/6. Our simulations shown that the probabilities calculated by our generated model fully matched those from the original PRISM model. 
\end{comment}
\mypar{Simple Peer-To-Peer Protocol}
This case study describes a simple peer-to-peer protocol based on BitTorrent\footnote{\url{https://www.prismmodelchecker.org/casestudies/peer2peer.php}}. The model comprises a set of clients trying to download a file that has been partitioned into $K$ blocks. Initially, there is one client that has already obtained all of the blocks and $N$ additional clients with no blocks. Each client can download a block from any of the others but they can only attempt four concurrent downloads for each block.\\
The code we analyze with $k=5$ and $N=4$ is reported in Listing \ref{ex2-code}.
\begin{lstlisting}[style=chor-color,caption={Choreography for the Peer-To-Peer Protocol.},captionpos=b,label={ex2-code}]
{
PeerToPeer := Client[i] $\rightarrow$ Client[i]: 
			(+["rate1*1"]  "(b[i]1'=1)"$\&\&$" " . PeerToPeer
			 +["rate2*1"]  "(b[i]2'=1)"$\&\&$" " . PeerToPeer
			 +["rate3*1"]  "(b[i]3'=1)"$\&\&$" " . PeerToPeer
			 +["rate4*1"]  "(b[i]4'=1)"$\&\&$" " . PeerToPeer
			 +["rate5*1"]  "(b[i]5'=1)"$\&\&$" " . PeerToPeer)
}
\end{lstlisting} 

Part of the generated PRISM code is shown in Listing \ref{ex2-gen} and it is faithful with what is reported in the PRISM documentation. Additionally, we generate the model for four clients (\emph{Client$_1$},$\ldots$, \emph{Client$_4$}).
\begin{lstlisting}[style=prism-color,caption={Generated PRISM program for the Peer-To-Peer Protocol.},captionpos=b,label={ex2-gen}]
module Client1
	Client1 : [0..1] init 0;
	b11 : [0..1]; 
	b12 : [0..1]; 
	b13 : [0..1]; 
	b14 : [0..1]; 
	b15 : [0..1]; 

	[] (Client1=0)  $\rightarrow$ rate1 : (b11'=1)$\&$(Client1'=0); 
	[] (Client1=0)  $\rightarrow$ rate2 : (b12'=1)$\&$(Client1'=0); 
	[] (Client1=0)  $\rightarrow$ rate3 : (b13'=1)$\&$(Client1'=0); 
	[] (Client1=0)  $\rightarrow$ rate4 : (b14'=1)$\&$(Client1'=0); 
	[] (Client1=0)  $\rightarrow$ rate5 : (b15'=1)$\&$(Client1'=0); 
endmodule
\end{lstlisting}


In Figure \ref{ex2-res}, we compare the values obtained for the probability that all clients have received all blocks by time $0\leq T\leq 1.5$ both for our generated model and the model reported in the documentation.
\begin{figure}[h]
\centering
\includegraphics[scale=0.35]{peertopeer.pdf}	
\caption{Probability that clients received all the block before $T$, with $0\leq T \leq 1.5$}
\label{ex2-res}
\end{figure}
\begin{comment}

\subsection{Random Graphs Protocol}
\begin{wrapfigure}[8]{l}{4cm}
	\includegraphics[scale=0.7]{network.pdf}	
\end{wrapfigure} 
The second case study we report is the random graphs protocol presented in the PRISM documentation\footnote{\url{https://www.prismmodelchecker.org/casestudies/graph_connected.php}}.
It investigates the likelihood that a pair of nodes are connected in a
 random graph. More precisely, we take into account the the set of random graphs $G(n,p)$,
  i.e. the set of random graphs with $n$ nodes where the probability of there being an edge 
  between any two nodes equals $p$. 

  The model is divided in two parts: at the beginning the random graph is built.
Then the algorithm finds nodes that have a path to node 2 by searching for nodes for which one can reach (in one step) a node for which the existence of a path to node 2 has already been found.

The choreographic model is shown in Listing \ref{ex4-code}, while
in Listing \ref{ex4-gen}, we report only part of the generated PRISM module (the modules $M_2$, $M_3$ and $P_2$, $P_3$ are equivalent to, respectively, $M_1$ and $P_2$ and can be found in the repository\footnote{\url{https://github.com/adeleveschetti/choreography-to-PRISM}}).

\begin{lstlisting}[style=chor-color,breaklines=true, postbreak=\mbox{\textcolor{red}{$\hookrightarrow$}\space},caption={Choreography for the Random Graphs
	Protocol.},captionpos=b,label={ex4-code}]
preamble
"mdp"
"const double p;"
endpreamble

n = 3;

PC -> PC : " ";
M[i] -> i in [1...n]  M[i] : "varM[i] : bool;";
P[i] -> i in [1...n] P[i] : "varP[i] : bool;";

{
GraphConnected0 := 
	PC -> M[i] : (+["1*p"] " " "(varM[i]'=true)". END
		        +["1*(1-p)"] " " "(varM[i]'=false)". END)
	PC -> P[i] : (+["1*p"] " " "(varP[i]'=true)" . END
		        +["1*(1-p)"] " " "(varP[i]'=false)".
			if "(PC=6)&!varP[i]&((varP[i] & varM[i]) | (varM[i+1] & varP[i+2])) "@P[i] then {
				["1"]"(varP[i]'=true)"@P[i] . GraphConnected0
			}) 								  
}
\end{lstlisting}
\begin{lstlisting}[style=prism-color,caption={Generated PRISM program for the Random Graphs
	Protocol.},captionpos=b,label={ex4-gen}]
mdp
const double p;
	
module PC
   PC : [0..7] init 0;
	
   [DPPGR] (PC=0)  $\rightarrow$ 1 :  (PC'=1); 
   [YCJJG] (PC=1)  $\rightarrow$ 1 :  (PC'=2); 
   [TWGVA] (PC=2)  $\rightarrow$ 1 :  (PC'=3); 
   [NODPZ] (PC=3)  $\rightarrow$ 1 :  (PC'=4); 
   [FDALJ] (PC=4)  $\rightarrow$ 1 :  (PC'=5); 
   [DCKXC] (PC=5)  $\rightarrow$ 1 :  (PC'=6); 
endmodule

module M1
   M1 : [0..1] init 0;
   varM1 : bool; 

   [DPPGR] (M1=0)  $\rightarrow$ p :(varM1'=true)$\&$(M1'=0) + (1-p) :(varM1'=false)$\&$(M1'=0); 
endmodule	

$\ldots$

module P1
   P1 : [0..3] init 0;
   varP1 : bool; 

   [NODPZ] (P1=0)  $\rightarrow$ p:(varP1'=true)$\&$(P1'=0) + (1-p):(varP1'=false)$\&$(P1'=0); 
   [] (P1=0)$\&$(PC=6)$\&$!varP1&((varP1 $\&$ varM1) | (varM2$\&$ varP3))  
   				$\rightarrow$ 1 : (varP1'=true)$\&$(P1'=0); 
endmodule
$\ldots$
\end{lstlisting}

The model is very similar to the one presented in the PRISM repository, the main difference is that we use state variables also for the modules \texttt{P$_i$} and \texttt{M$_i$}, 
where in the original model they were not requires.
However, this does not affect the behaviour of the model, as the reader can notice from the results of the probability that nodes 1 and 2 are connected showed in Figure \ref{ex4-res}.
\begin{figure}[h]
\centering
\includegraphics[scale=0.6]{example5-results.jpeg}	
\caption{Probability that the nodes 1 and 2 are connected.}
\label{ex4-res}
\end{figure}
\end{comment}


\mypar{A Modified thinkteam Protocol.}
In this modified version of the thinkteam protocol introduced in the earlier sections, we extend the protocol to involve generalised interactions with possible many receivers. Specifically, the \texttt{CheckOut} process now communicates with two users simultaneously, \texttt{User1} and \texttt{User2} each tasked with performing distinct actions upon access to the file. 

 In the first branch, \texttt{User1} increments the variable \texttt{x} by 1, while \texttt{User2} decrements the variable \texttt{y} by 1. Conversely, in the second branch, the roles are reversed, with \texttt{User1} decrementing \texttt{x} and \texttt{User2} incrementing \texttt{y}.

 \begin{lstlisting}[style=chor-color,breaklines=true, postbreak=\mbox{\textcolor{red}{$\hookrightarrow$}\space},caption={Choreography for the Modified thinkteam Protocol},captionpos=b,label={ex1-chor}]
   C0 := CheckOut $\rightarrow$ User1, User2 : (+["1*lambda"] " " "(x=x+1)"  "(y=y-1)"; C1
                                       	+["1*lambda"]  " " "(x=x-1)"  "(y=y+1)";  C2)
   C1 :=  CheckOut $\rightarrow$ User1, User2 : (+["1*theta"] ; C0)  
   C2 :=  CheckOut $\rightarrow$ User1, User2 : (+["1*mu "] ; C1   +["1*mu "] ;  C2)
 \end{lstlisting}

 Part of the generated PRISM model is reported in Listing \ref{ex1-gen}. 

 \begin{lstlisting}[style=prism-color,caption={Generated PRISM program},captionpos=b,label={ex1-gen}]
   module CheckOut
      CheckOut_STATE : [0..2] init 0;
      [MMHOL]  (CheckOut_STATE=0) -> 1 :  (CheckOut_STATE'=1);
      [FFSFW]  (CheckOut_STATE=0) -> 1 :  (CheckOut_STATE'=2);
      [ULCFN]  (CheckOut_STATE=1) -> 1 : (CheckOut_STATE'=0);
      [YHHWG]  (CheckOut_STATE=2) -> 1 : (CheckOut_STATE'=1);
      [XWSAO]  (CheckOut_STATE=2) -> 1 : (CheckOut_STATE'=2);
   endmodule
$\ldots$
   module User2
      User2_STATE : [0..2] init 0;
      [MMHOL]  (User2_STATE=0) -> lambda : (y'=y-1)&(User2_STATE'=1);
      [FFSFW]  (User2_STATE=0) -> lambda : (y'=y+1)&(User2_STATE'=2);
      [ULCFN]  (User2_STATE=1) -> mu : (User2_STATE'=0);
      [YHHWG]  (User2_STATE=2) -> theta : (User2_STATE'=1);
      [XWSAO]  (User2_STATE=2) -> theta : (User2_STATE'=2);
endmodule
\end{lstlisting}
The generated PRISM model appears less clear compared to its
choreographic representation, primarily due to its lack of sequential
structure and lower readability.  In the PRISM model, the absence of a
clear sequential structure makes it harder to follow the flow of
interactions between components, % Unlike the choreographic language,
% which explicitly defines sequential steps (e.g., \texttt{C0},
% \texttt{C1}, \texttt{C2}), the PRISM model's modular approach may
% confuse the order of operations and dependencies between different
% parts of the system.  Additionally,
% The PRISM model's syntax, with its
since module definitions and transition labels can be more challenging
to read and comprehend compared to the concise and structured nature
of the choreographic language. % The descriptive actions used in the
% choreographic language (e.g., \texttt{+["1*lambda"]},
% \texttt{+["1*mu"]}) provide clearer insights into the actions
% performed at each step.


\mypar{Proof of Work Bitcoin Protocol.}
\begin{comment}
\begin{wrapfigure}[11]{r}{4cm}
	\includegraphics[scale=0.45]{bitcoin.pdf}	
\end{wrapfigure} 
\end{comment}
In \cite{DBLP:journals/concurrency/BistarelliNGLMV23}, the authors extended the PRISM model checker syntax to incorporate dynamic data types, enhancing its capabilities to model the Proof of Work protocol used in the Bitcoin blockchain \cite{bitcoin}. 

In summary, the code depicts miners engaging in
  solving PoW, updating their ledgers, and communicating with the
  network.  The indices $i$ represent the module renaming feature of
  the choreographic language. Thus, each interaction will be repeated
  for each miner and hasher of the protocol that we are
  considering. The protocol works as follows. 
  \begin{lstlisting}[style=chor-color,breaklines=true, postbreak=\mbox{\textcolor{red}{$\hookrightarrow$}\space},caption={Choreography for the Proof of Work Bitcoin Protocol},captionpos=b,label={ex3-code}]
   {PoW $\coloneqq$ Hasher[i] $\rightarrow$ Miner[i] :
   (+["mR*hR[i]"]  " " "(b[i]'=createB(b[i],B[i],c[i]))&(c[i]'=c[i]+1)" ; 
      Miner[i] $\rightarrow$ Network : (["rB*1"] "(B[i]'=addBlock(B[i],b[i]))" foreach(k!=i) "(set[k]'=addBlockSet(set[k],b[i]))"@Network;PoW)
    +["lR*hR[i]"] ; 
            if "!isEmpty(set[i])"@Miner[i] then { 
               ["r"] "(b[i]'=extractBlock(set[i]))"@Miner[i] ;  
                  Miner[i] $\rightarrow$ Network : 
                     (["1*1"] "(setMiner[i]'=addBlockSet(setMiner[i],b[i]))" "(set[i]' = removeBlock(set[i],b[i]))";PoW) 
            }
            else{
               if "canBeInserted(B[i],b[i])"@Miner[i] then { 
                  ["1"] "(B[i]'=addBlock(B[i],b[i]))&(setMiner[i]'=removeBlock(setMiner[i],b[i]))"@Miner[i];PoW 
               }
               else{PoW}})} 
   \end{lstlisting}
   \vspace{-0.15cm}
  When synchronising with
  the hasher, a miner tries to solve a cryptographic
  puzzle. Successful attempts add a new block to its ledger and update other miners' block sets. Unsuccessful attempts involve extracting a block, updating its ledger and block sets, and continuing the PoW process.
  \begin{wrapfigure}{r}{0.38\textwidth}
   \vspace{-0.75cm}
   \centering
   \includegraphics[scale=0.35]{example.pdf}	
   \vspace{-0.35cm}
   \caption{}
   \label{ex3-res}
   \vspace{-0.75cm}
\end{wrapfigure}
The PRISM model we created is more verbose than the one in \cite{DBLP:journals/concurrency/BistarelliNGLMV23}, mainly because we consistently generate the else branch for if-then-else expressions, resulting in a higher number of instructions. Despite this, the experimental results for block creation probability within a bound time $T$ (Figure \ref{ex3-res}) remain unaffected. Any discrepancies between the original and generated models are due to inherent variations in the simulation-based calculation of probability.


\mypar{Hybrid Casper Protocol.}
\begin{comment}
\begin{wrapfigure}[12]{l}{4.5cm}
	\includegraphics[scale=0.45]{ethereum.pdf}	
\end{wrapfigure} 
\end{comment}
We now present the Hybrid Casper Protocol \cite{DBLP:journals/distribledger/GallettaLMV23}. The Hybrid Casper protocol represents a hybrid consensus protocol for blockchains, merging features from both Proof of Work and Proof of Stake protocols. 
\begin{lstlisting}[style=chor-color,tabsize=2,breaklines=true, postbreak=\mbox{\textcolor{red}{$\hookrightarrow$}\space},	caption={Excerpt of the Hybrid Casper Protocol as a choreography},captionpos=b,label={ex5-code}]
{PoS := Hasher[i] -> Validator[i] :
(+["mR*1"]  "(b[i]'=createB(b[i],L[i],c[i]))&(c[i]'=c[i]+1)"; 
	if "!(mod(getHeight(b[i]),EpochSize)=0)"@Validator[i] then{$\ldots$}
	else{
		Validator[i] -> Vote_Manager :(["1*1"]  "(Votes'=addVote(Votes,b[i],stake[i]))"; PoS)
	}
 +["hR*1"]  ; if "!isEmpty(set[i])"@Validator[i] then { $\dots$ }
 							else{ PoS }
 +["rC*1"] "(lastCheck[i]'=extractCheckpoint(listCheckpoints[i],lastCheck[i]))"$\ldots$}

\end{lstlisting}
The modeling approach is very similar to the one used for the Proof of Work Bitcoin protocol. Specifically, the Hybrid Casper protocol is represented in PRISM as the parallel composition of $n$ \emph{Validator} modules, along with the modules \emph{Vote\_Manager} and \emph{Network}. Each \emph{Validator} module closely resembles the \emph{Miner} module from the previous protocol. The module \emph{Vote\_Manager} is responsible for storing maps containing votes for each block and computing associated rewards/penalties.
The choreographic model for this example is reported in Listing \ref{ex5-code}. 
The code resembles that of the Proof of Work protocol, but each validator can either create a new block, receive blocks from the network module, or determine if it's eligible to vote for specific blocks.
For lack of space, we detailed only part of the code, the complete model can be found in \cite{repository}.

\begin{comment}
\begin{lstlisting}[style=prism-color,caption={Generated PRISM program for the Hybrid Casper	Protocol.},captionpos=b,label={ex5-gen}]
module Validator1
   $\ldots$
	
   [] (Validator1=0)  $\rightarrow$  mR : (b1'=createB(b1,L1,c1))$\&$(c1'=c1+1)&(Validator1'=1); 
   [] (Validator1=0)  $\rightarrow$  lR :  (Validator1'=2); 
   [] (Validator1=0)$\&$(!isEmpty(listCheckpoints1))  $\rightarrow$  
   	rC : (lastCheck1'=extractCheckpoint(listCheckpoints1,lastCheck1))$\&$(heightLast1'=getHeight(extractCheckpoint(listCheckpoints1,lastCheck1)))$\&$(votes1'=calcVotes(Votes,extractCheckpoint(listCheckpoints1,lastCheck1)))$\&$(Validator1'=3); 
   [NGRDF] (Validator1=1)$\&$!(mod(getHeight(b1),EpochSize)=0)  $\rightarrow$  1 : (L1'=addBlock(L1,b1))$\&$(Validator1'=0); 
   [] (Validator1=1)$\&$!(!(mod(getHeight(b1),EpochSize)=0)) $\rightarrow$  1 : (Validator1'=3); 
   [PCRLD] (Validator1=1)$\&$!(mod(getHeight(b1),EpochSize)=0)  $\rightarrow$  
   	1 : (L1'=addBlock(L1,b1))$\&$(Validator1'=4); 
   [VSJBE] (Validator1=5)  $\rightarrow$  1 :  (Validator1'=0); 
   [] (Validator1=2)$\&$!isEmpty(set1) $\rightarrow$  
   	1 : (b1'=extractBlock(set1))$\&$(Validator1'=4); 
   [] (Validator1=4)$\&$!canBeInserted(L1,b1) $\rightarrow$  (Validator1'=0);
   [] (Validator1=4)$\&$!(!canBeInserted(L1,b1)) $\rightarrow$  1 : (Validator1'=6); 
   [MDDCF] (Validator1=6)$\&$!(mod(getHeight(b1),EpochSize)=0)  $\rightarrow$ 
   	1 : (setMiner1' = addBlockSet(setMiner1 , b1))$\&$(Validator1'=0); 
   [] (Validator1=6)$\&$!(!(mod(getHeight(b1),EpochSize)=0)) $\rightarrow$  1 : (Validator1'=8); 
   [IQVPA] (Validator1=6)$\&$!(mod(getHeight(b1),EpochSize)=0)  $\rightarrow$  
   	1 : (setMiner1' = addBlockSet(setMiner1 , b1))$\&$(Validator1'=9); 
   [IFNVZ] (Validator1=10)  $\rightarrow$  1 :  (Validator1'=0); 
   [] (Validator1=2)$\&$!(!isEmpty(set1)) $\rightarrow$  1 : (Validator1'=0);
   [] (Validator1=3)$\&$(heightLast1=heightCheckpoint1+EpochSize)$\&$(votes1>=2/3*tot_stake) $\rightarrow$  (Validator1'=4);
   [] (Validator1=4)$\&$(heightLast1=heightCheckpoint1+EpochSize) $\rightarrow$  
   	1 : (lastJ1'=b1)$\&$(L1'= updateHF(L1,lastJ1))$\&$(Validator1'=6); 
   [EQCYO] (Validator1=6)  $\rightarrow$  1 :  (Validator1'=0); 
   [] (Validator1=4)$\&$!((heightLast1=heightCheckpoint1+EpochSize)) $\rightarrow$  
   	1 : (lastJ1'=b1)$\&$(Validator1'=0); 
   [] (Validator1=3)$\&$!((heightLast1=heightCheckpoint1+EpochSize)$\&$(votes1>=2/3*tot_stake)) $\rightarrow$  1 : (Validator1'=0);
endmodule
$\ldots$
module Network
   Network : [0..1] init 0;
   set1 : list []; 
   set2 : list []; 
   set3 : list []; 
   set4 : list []; 
   set5 : list []; 

   [NGRDF] (Network=0)  $\rightarrow$  
   	1 : (set2'=addBlockSet(set2,b2))$\&$(set3'=addBlockSet(set3,b3))$\&$(set4'=addBlockSet(set4,b4))$\&$(set5'=addBlockSet(set5,b5))$\&$(Network'=0); 
   [PCRLD] (Network=0)  $\rightarrow$  
   	1 : (set2'=addBlockSet(set2,b2))$\&$(set3'=addBlockSet(set3,b3))$\&$(set4'=addBlockSet(set4,b4))$\&$(set5'=addBlockSet(set5,b5))$\&$(Network'=0); 
   [MDDCF] (Network=0)  $\rightarrow$  1 : (set1' = removeBlock(set1,b1))$\&$(Network'=0); 
   [IQVPA] (Network=0)  $\rightarrow$  1 : (set1' = removeBlock(set1,b1))$\&$(Network'=0); 
   $\ldots$
endmodule

module Vote_Manager
   Vote_Manager : [0..1] init 0;
   epoch : [0..10] init 0;
   Votes : hash[];  
   tot_stake : [0..120000] init 50; 
   stake1 : [0..N] init 10; 
   stake2 : [0..N] init 10; 
   stake3 : [0..N] init 10; 
   stake4 : [0..N] init 10; 
   stake5 : [0..N] init 10; 

   [VSJBE] (Vote_Manager=0)  $\rightarrow$  
   	1 : (Votes'=addVote(Votes,b1,stake1))$\&$(Vote_Manager'=0); 
   $\ldots$
endmodule

\end{lstlisting}
\end{comment}

\begin{wrapfigure}{r}{0.38\textwidth}
	\vspace{-0.75cm}
	\centering
	\includegraphics[scale=0.35]{example2.pdf}	
	\vspace{-0.35cm}
	\caption{}
	\label{ex5-res}
	\vspace{-0.75cm}
	\end{wrapfigure}
        The generated code is very similar the one outlined in
        \cite{DBLP:journals/distribledger/GallettaLMV23}, with the
        main distinction being the greater number of lines in our
        generated model.  This difference is due to the fact that
        certain commands could be combined, but our generation lacks
        the automatic capability to perform this check. While the
        results obtained for the probability of creating a block within the time $T$
        reported in Figure \ref{ex5-res} exhibit similarity, running
        simulations for the generated model takes PRISM 39.016
        seconds, compared to the 22.051 seconds required for the
        original model.



\mypar{Synchronous Leader Election.}
This case study examines the synchronous leader election protocol proposed by Itai $\&$ Rodeh~\cite{IR90}, designed to elect a leader in a ring of $N$ processors by exchanging messages. The protocol operates in rounds, where each processor selects a random ID from $\{1, \ldots, K\}$, circulates it around the ring, and determines if a unique maximum ID exists. If so, the processor with this ID becomes the leader; otherwise, the process repeats in the next round.

For illustration, we considered the case where $ N=4 $ and $ K=8 $, following the PRISM model\footnote{\url{https://www.prismmodelchecker.org/casestudies/synchronous_leader.php}} . We modeled this example in our choreographic language, as shown in Listing~\ref{leader-code}, capturing the protocol's behavior and dynamics.

\begin{lstlisting}[style=chor-color,caption={Choreography for the Synchronous Leader Election Protocol.},captionpos=b,label={leader-code}]
   {
   PeerToPeer := allSynch{ j in [1...4]
      Process[j] : (true -> "1/K" : "(p[i]'=0)&(v[i]'=0)&(u[i]'=true)" + $\ldots$ + "1/K" : "(p[i]'=0)&(v[i]'=7)&(u[i]'=true)") }.
      allSynch{
         Counter : ("(c<N-1)" -> "1" : "(c'=c+1)")
         Counter : ("(c=N-1)" -> "1" : "(c'=c)")
         Process1 : ("u1&!(p1=2)&(c<N-1)" -> "1" : "(u1'=true)&(v1'=v2)")
         Process1 : ("u1&(p1=2)&(c<N-1)" -> "1" : "(u1'=false)&(v1'=v2)&(p1'=0)")
         Process1 : ("!u1&(c<N-1)" -> "1" : "(u1'=false)&(v1'=v2)")
         Process1 : ("u1&!(p1=v2)&(c=N-1)" -> "1" : "(u1'=true)&(v1'=0)&(p1'=0)")
         Process1 : ("u1&(p1=v2)&(c=N-1)" -> "1" : "(u1'=false)&(v1'=0)&(p1'=0)")
         Process1 : ("!u1&(c=N-1)" -> "1" : "(u1'=false)&(v1'=0)")
         $\ldots$
      }.
      if "u1 | u2 | u3 | u4"@Counter then {
         Counter -> Process[i] : (["1*1"] "(c'=c)" "(u[i]'=false)&(v[i]'=0)&(p[i]'=0)"
         . allSynch {
            Counter : (true -> "1" : "(c'=c)")
            Process1 : (true -> "1" : " ")
            $\ldots$               
         } . END)
      }
      else{
         Counter -> Process[i] : (["1*1"] "(c'=1)" "(u[i]'=false)&(v[i]'=0)&(p[i]'=0)"  . Election)
      }      
   }
   \end{lstlisting} 
   While the generated model (Listing \ref{leader-prism} successfully replicates the functionality of 
   the PRISM repository model, a key difference lies in the modular 
   structure of the two representations. 
   Specifically, the generated model adopts a simplified modular design by 
   grouping transitions more compactly in certain modules, such as the Counter module. 
   This simplification reduces redundancy and may improve readability without altering the correctness or outcomes of the protocol. Importantly, this structural refinement does not impact the behavior of the system, as the generated model remains functionally equivalent to the original PRISM repository model, as displayed in Figure \ref{leader-res}.

   \begin{lstlisting}[style=prism-color,caption={Generated PRISM program},captionpos=b,label={leader-prism}]
   module Counter
      Counter : [0..4] init 0;
      c : [0..N-1] init 0;
      [YQBDX] (Counter = 0)&(c<N-1) -> 1 : (c'=c+1)&(Counter'=1);
      [YQBDX] (Counter = 0)&(c=N-1) -> 1 : (c'=c)&(Counter'=1);
      [ELTMI] (Counter=1)&(u1 | u2 | u3 | u4) -> 1 : (c'=c)&(Counter'=2);
      [LJTIP] (Counter=1)&!(u1 | u2 | u3 | u4) -> 1 : (c'=1)&(Counter'=0);
      [AWUQP] (Counter = 2)-> 1 : (c'=c)&(Counter'=2);
   endmodule
   module Process1
   Process1 : [0..4] init 0;
      p1 : [0..K-1] init 0;
      v1 : [0..K-1] init 0;
      u1 : bool;
      [BKKXT](Process1 = 0) ->  1/K:(p1'=0)&(v1'=0)&(u1'=true)&(Process1'=1) 
                                 + 1/K:(p1'=1)&(v1'=1)&(u1'=true)&(Process1'=1) 
                                 + 1/K:(p1'=2)&(v1'=2)&(u1'=true)&(Process1'=1) 
                                 + 1/K:(p1'=2)&(v1'=3)&(u1'=true)&(Process1'=1) 
                                 + 1/K:(p1'=2)&(v1'=4)&(u1'=true)&(Process1'=1) 
                                 + 1/K:(p1'=2)&(v1'=5)&(u1'=true)&(Process1'=1) 
                                 + 1/K:(p1'=2)&(v1'=6)&(u1'=true)&(Process1'=1) 
                                 + 1/K:(p1'=2)&(v1'=7)&(u1'=true)&(Process1'=1);
      [YQBDX] (Process1 = 1)&u1&!(p1=2)&(c<N-1) -> 1 : (u1'=true)&(v1'=v2)&(Process1'=2);
      [YQBDX] (Process1 = 1)&u1&(p1=2)&(c<N-1) -> 1 : (u1'=false)&(v1'=v2)&(p1'=0)&(Process1'=2);
      [YQBDX] (Process1 = 1)&!u1&(c<N-1) -> 1 : (u1'=false)&(v1'=v2)&(Process1'=2);
      [YQBDX] (Process1 = 1)&u1&!(p1=v2)&(c=N-1) -> 1 : (u1'=true)&(v1'=0)&(p1'=0)&(Process1'=2);
      [YQBDX] (Process1 = 1)&u1&(p1=v2)&(c=N-1) -> 1 : (u1'=false)&(v1'=0)&(p1'=0)&(Process1'=2);
      [YQBDX] (Process1 = 1)&!u1&(c=N-1) -> 1 : (u1'=false)&(v1'=0)&(Process1'=2);
      [ELTMI] (Process1=2) -> 1 : (u1'=false)&(v1'=0)&(p1'=0)&(Process1'=3);
      [LJTIP] (Process1=2) -> 1 : (u1'=false)&(v1'=0)&(p1'=0)&(Process1'=0);
      [AWUQP] (Process1 = 3) -> 1 :  (Process1'=4);
   endmodule
   $\ldots$
   \end{lstlisting}


   \begin{figure}[h]
      \centering
      \includegraphics[scale=0.35]{leader.pdf}	
      \caption{The probability of electing a leader within $L$ rounds, with $\leq L \leq 10$}
      \label{leader-res}
      \end{figure}
\begin{comment}
\subsection{Problems}
\label{sec:problems}
While testing our choreographic language, we noticed that some of the case studies presented in the 
PRISM documentation \cite{PRISMdoc} cannot be modeled by using our language.
The reasons are various, in this section we try to outline the problems.

\begin{itemize}
\item \textbf{Asynchronous Leader Election}\footnote{\url{https://www.prismmodelchecker.org/casestudies/asynchronous_leader.php}}:
 processes synchronize with the same label but the conditions are different.
 We include in our language the \texttt{it-then-else} statement but we do not allow 
 the \texttt{if-then} (without the \texttt{else}). This is done because in this way, we do not 
 incur in deadlock states.
\item  \textbf{Probabilistic Broadcast Protocols}\footnote{\url{https://www.prismmodelchecker.org/casestudies/prob_broadcast.php}}:
 also in this case, the problem are the labels of the synchronizations.
 In fact, all the processes synchornize with the same label on every actions.
 This is not possible in our language, since a label is unique for every synchronization between two (or more) processes.
\item \textbf{Cyclic Server Polling System}\footnote{\url{https://www.prismmodelchecker.org/casestudies/polling.php}}:
 in this model, the processes \texttt{station$_i$} do two different things in the same state.
 More precicely, at the state 0 (\texttt{s$_i$=0}), the processes may synchornize with the process
 \texttt{server} or may change their state without any synchronization.
 In out language, this cannot be formalized since the synchronization is a branch action,
 so there should be another option with a synchronization.


\end{itemize}
\end{comment}


%%% Local Variables: 
%%% mode: latex
%%% TeX-master: "main"
%%% End: