% ESEMPIO DA INTRODURRE? 
%
% \begin{displaymath}
%   \begin{array}{lllll}
%     \mathsf{p}: \left\{\quad 
%     \begin{array}{ll}
%       {}[\ell_1]\ s_{\mathsf p}=1 \ \rightarrow\  {} 3: (x'=1)\ \&\ (s_{\mathsf p}=s_{\mathsf p}+1)\\
%       {}[\ell_2]\ s_{\mathsf p}=1 \ \rightarrow\  {} 7: (x'=1)\ \&\ (s_{\mathsf p}= \mathsf{end})\\
%       {}[\ell']\ s_{\mathsf p}=2 \ \rightarrow\  {} 100: (x'=0)\ \&\ (s_{\mathsf p}= \mathsf{end})
%     \end{array}
%     \quad\right\}
%   \end{array}
% \end{displaymath}
%
%
% \begin{displaymath}
%   \begin{array}{lllll}
%     \mathsf{q}: \left\{\quad 
%     \begin{array}{ll}
%       {}[\ell_1]\ s_{\mathsf p}=1 \ \rightarrow\  {} 1: (y'=2)\ \&\ (s_{\mathsf q}=s_{\mathsf q}+1)\\
%       {}[\ell_2]\ s_{\mathsf p}=1 \ \rightarrow\  {} 1: (s_{\mathsf q}=s_{\mathsf q}+2)\\
%       {}[\ell']\ s_{\mathsf p}=2 \ \rightarrow\  {} 1: (y'=0)\ \&\ (s_{\mathsf q}= \mathsf{end})\\
%       {}[\ell'']\ s_{\mathsf p}=3 \ \rightarrow\  {} 1: (y'=20)\ \&\ (s_{\mathsf q}= \mathsf{end})
%     \end{array}
%     \quad\right\}
%   \end{array}
% \end{displaymath}

% \begin{displaymath}
%   \begin{array}{lllll}
%     \mathsf{r}: \left\{\quad 
%     \begin{array}{ll}
%       {}[\ell_1]\ s_{\mathsf p}=1 \ \rightarrow\  {} 1: (z'=3)\ \&\ (s_{\mathsf r}= \mathsf{end})\\
%       {}[\ell_2]\ s_{\mathsf p}=1 \ \rightarrow\  {} 1: \ \&\ (s_{\mathsf p}= s_{\mathsf p}+1)\\
%       {}[\ell'']\ s_{\mathsf p}=3 \ \rightarrow\  {} 100: (z'=30)\ \&\ (s_{\mathsf r}= \mathsf{end})
%     \end{array}
%     \quad\right\}
%   \end{array}
% \end{displaymath}
%
Our next task is to provide a mapping that can translate
choreographies into the PRISM language. This section addresses
projection formally.
% The goal of this section is to address the theoretical aspect of
% this.

\mypar{Mapping Choreographies to PRISM.}  The operation of generating
endpoint code from a choreography is known as {\em
  projection}. 
% Normally, projection is defined separately for each module appearing
% in the choreography program.
We observe that to simulate a choreography interaction in
PRISM, we need to use labels on which each involved module can
synchronise. Therefore, without loss of generality, we make a slight
abuse of notation and assume that each interaction in a choreography
is annotated with a unique label (which will be used by the
projection). We call such a choreography an {\em annotated
  choreography}. 
%
% 
For example, the choreography
$\interactBase{p}{q} +\ \lambda_1: \CEnd \ +\ \lambda_2: \ifTE
  {E}{p}{\CEnd}{\CEnd}$ is going to be annotated as
  $\interactBasel{p}{a}{q} +\ \lambda_1: \CEnd \ +\ \lambda_2: \ifTEl
    {E}{}{p}{\CEnd}{\CEnd}$.
% %
% \MC{We need to say that we must run some standard static checks
% because, since there is branching, some terms may not be
% projectable.}
% %
We now separate the definition of projection based on whether we
are dealing with rates or probabilities.  We start with choreographies 
with rates: 
% can then define the projection of a choreography with rates into the
% PRISM language.
% 
\begin{definition}[Projection, CTMC]\label{def:projCTMC} Given an
  annotated choreography with rates $C$, a module $\role p$, and a
  natural number $\iota$, we define the function $\proj$ as:
  \begin{displaymath}\small
        \begin{array}{lr}

          \proj (\role q,\interactl{p}{a}{\role p_1,\ldots,\role p_n}, \iota)= 
          &  \boxed{\text{if }\role q=\role p}\\[2mm]
          \qquad
          % \{ 
          \Big\{\commandBase {a_j} {\ s_{q}\!=\! \iota} {\lambda_j:\ s_{q}\!=\!
          s_{q} +1+\sum_{k=1}^{j-1}\mathsf{nodes}({C_k})}\ \&\ \projE
          {u_j}q\Big\}_{j\in J}
          \\
          \qquad\cup\ \bigcup_{j} \proj (\role q, C_j, \iota+1+\sum_{k=1}^{j-1}\mathsf{nodes}({C_k}))
          \\\\

          \proj (\role q,\interactl{p}{a}{\role p_1,\ldots,\role p_n}, \iota)= 
          &  \!\!\!\!\!\! \!\!\!\!\!\!\boxed{\text{if }\role q\in\{\role p_1,\ldots,\role p_n\}}\\[2mm]
          \qquad
          % \{ 
          \Big\{\commandBase {a_j} {\ s_{q}\!=\! \iota} {1:\ s_{q}\!=\!
          s_{q} +1+\sum_{k=1}^{j-1}\mathsf{nodes}({C_k})}\ \&\ \projE
          {u_j}q\Big\}_{j\in J}
          \\
          \qquad\cup\ \bigcup_{j} \proj (\role q, C_j, \iota+1+\sum_{k=1}^{j-1}\mathsf{nodes}({C_k}))
          \\\\

          \proj (\role q,\interactl{p}a{\role p_1,\ldots,\role p_n}, \iota)=
          % \ \proj (\role{q}, C_1, \iota)
          &  \hspace{-1.8cm}\boxed{\text{if }\role q\not\in\{\role p, 
            \role p_1,\ldots,\role p_n\}}
          \\[2mm]

          \qquad\bigcup_{j} \proj (\role q, C_j, \iota+\sum_{k=1}^{j-1}\mathsf{nodes}({C_k}))

          % \\[2mm]
          % \qquad\text{such that } \forall i,j.\ \proj (\role{q}, C_i, \iota)= \proj (\role{q}, C_j, \iota)
        %  \\\\

        \end{array}
      \end{displaymath}


  \begin{displaymath}\small
        \begin{array}{lr}

          \proj (\role q,\ifTE {E}{p}{C_1}{C_2}, \iota) = 
          &  \boxed{\text{if }\role q=\role p}\\[2mm]
          \qquad\left\{ 
          \begin{array}{lll}
            \commandBase {} {s_{q}\!=\! \iota\ \&\ E}{\ 1: s'_{q}\!=\! \iota+1},\\ 
            \commandBase {} {s_{q}\!=\! \iota\ \&\ \mathsf{not}(E)}
            {\ 1: s'_{q}\!=\! \iota+\mathsf{nodes}(C_1)+1}
          \end{array}
          \right\}
          \\[3mm]
          \qquad\cup\quad \proj (\role{p}, C_1, \iota+1)
          \quad\cup\quad
          \proj (\role{p}, C_2, \iota+\mathsf{nodes}(C_1)+1)
          \\\\

          \proj (\role q,\ifTE {E}{p}{C_1}{C_2}, \iota) = 
          % \ \proj (\role{q}, C_1, \iota)
          &  \boxed{\text{if }\role q\neq\role p}\\[2mm]
          %
          % \quad\cup\quad
          % \proj (\role{q}, C_2, \iota+\mathsf{nodes}(C_1)+1)
          % \\[2mm]


          \qquad \proj (\role{p}, C_1, \iota)
          \quad\cup\quad
          \proj (\role{p}, C_2, \iota+\mathsf{nodes}(C_1))


          % \qquad\text{such that if } 
          % % 
          % \forall i. 
          % % \role q\in C_i
          % \proj (\role{q}, C_i, \iota)\neq\emptyset
          % \text{ then } \proj (\role{q}, C_1, \iota)= \proj (\role{q}, C_2, \iota)

          \\\\

          \proj (\role q,\CEnd, \iota) = \emptyset

          \\\\

          \proj (\role q, X, \iota) = 
          \commandBase {} {s_{q}\!=\! \iota}{\ 1: s'_{q}\!=\! \iota'}
          \qquad\text{ where } \textsf{defs}(X) = \iota'

        \end{array}
      \end{displaymath}
    \end{definition}
    We go through the various cases of the definition above. The first
    three cases handle the projection of an interaction. If we project
    the first module $\role p$, then we create one command per branch,
    assigning the corresponding rate. Note that this is possible since
    we are dealing with rates. The additional variable $s_{\role q}$
    is the counter for this module and we identify uniquely this
    interaction. In order to do it consistently for follow up
    statements of the subterm choreographies, we use the function
    $\mathsf{nodes}(C)$ which returns the number of nodes of $C$,
    i.e., the number of steps of the projection function. Obviously,
    when projecting the next branch we need to consider all other
    possible branches that may have been already
    projected. Intuitively, a label and an integer (ranged by $\iota$)
    identify a node in the abstract syntax tree of a
    choreography. Also, note that we assume that from a label $a$ we
    can generate distinct sublabels $a_j$ just by adding some index
    $j$. For the sake of space, we do not define the function
    precisely, but we observe that this could also be easily defined
    via the label annotations. The second case defines the projection
    of an interaction for one of the modules
    $\{\role p_1,\ldots,\role p_n\}$. Similarly to the previous case,
    we define a command for each branch of the interaction. However,
    the rate of each command is going to be $1$, making sure that each
    branch synchronises with probability $\lambda_j*1$ (see rule
    $\mathsf{(P_2)}$ in Figure~\ref{fig:semantics}).
    %
    The third case is the one when we are projecting a module that is
    not in the set $\{\role p, \role p_1,\ldots,\role p_n\}$. % Our
    % projection takes a standard approach that requires each branch to
    % project the same~\cite{HYC16}. A generalisation of this, which may
    % require a merging operation~\cite{CHY12} and perhaps a more
    % intrinsic treatment of rates and probabilities, is left as future
    % work.
    % 
    % \marco{BEST TO COMMENT ON SELF-INTERACTION HERE and RELATED TO
    %   PROJECTION: The paper does nor require the usual condition of
    %   non-self-interaction (as evident in Listing 1.4). Additionally,
    %   clarification is needed on how projection works in cases where
    %   non-self-interaction is not enforced. Are the projection rules
    %   ordered? }
    The if-then-else construct is only interesting for module
    $\role p$ where the module makes an internal choice based on the
    evaluation of the guard $E$. % As in the previous case, the
    % other modules must project the same in $C_1$ and $C_2$.
    For recursive calls, we generate a command that resets the counter
    to a distinct value given by the auxiliary function
    $\textsf{defs}$. As an example, we can apply the projection
    function to Example~\ref{example2} and obtain the PRISM modules
    from Example~\ref{example3}.

    As hinted above, the projection in Definition~\ref{def:projCTMC}
    would be incorrect if instead of using rates we used
    probabilities. This is simply because we cannot force both
    $\role p$ and $\{\role p_1,\ldots,\role p_n\}$ to take the same
    branch with the probability distribution of the $\lambda_i$'s. To
    fix this problem, we have the following definition instead:
    % 
    \begin{definition}[Projection, DTMC]\label{def:projDTMC} Given an annotated
      choreography with probabilities $C$, a module $\role p$, and a
      natural number $\iota$, we define $\proj$ as:
      \begin{displaymath}\small
        \begin{array}{lr}

          \proj (\role q,\interact{p}{\role p_1,\ldots,\role p_n}, \iota)= 
          &  \boxed{\text{if }\role q=\role p}\\[2mm]
          \qquad
          \left\{
          \begin{array}{lll}
            \commandBase {} {s_{q}\!=\! \iota}{\ \sum_{j\in J} \lambda_j: s'_{q}\!=\! \iota+1+j},\\ 
            \{\commandBase {l_j} {s_{q}\!=\! \iota+1+j}
            \ 1: s'_{q}\!=\! \iota+1+\sum_{k=1}^{j-1}\mathsf{nodes}({C_k})\ \&\ \projE
            {u_j}q\}_{j\in J}
          \end{array}
          \right\}
          \\
          \qquad\cup\ \bigcup_{j} \proj (\role q, C_j, s+1+\sum_{k=1}^{j-1}\mathsf{nodes}({C_k}))
          \\\\

        \end{array}
      \end{displaymath}
      The other cases of the definition are equivalent to those in
      Definition \ref{def:projCTMC}.
    \end{definition}
    The fix is immediate: module $\role p$ takes a (probabilistic)
    internal decision on the $j^{\text{th}}$ branch and then
    synchronises on label $l_j$ with $\{\role p_1,\ldots,\role p_n\}$.       

\smallskip

\mypar{Correctness.} Our projection operations are correct with
respect to the semantics of choreographies and PRISM. In the sequel,
$S_+$ is a state $S$ extended with the extra variables $s_{\role q}$
(for each module in a choreography) used by the projection. In the
projection, we utilize alphabetized parallel composition $\pp$,
wherein modules synchronise solely on labels that appear in both
modules.
% Moreover, we say that a choreography is {\em well-formed} if the
% behaviour in each branching for modules who do not participate to
% the interaction is identical.
Additionally, a choreography is {\em strongly connected},  
%
if (i) subsequent interactions share some modules~\cite{CHY12} and,
%
(ii) in every branch of a probabilistic choice or an if-then-else
involving modules $\role p_1, \ldots, \role p_n$, the fist action (if
any) of every other module $\role q$ must be an interaction with
$\role p_1, \ldots, \role p_n$ (up to unfolding of recursive
calls). For example, the choreography
%
\begin{displaymath}\small
  X\stackrel{\mathsf{def}}{=} 
  \interactBase{p}{q}:\,
  \left(
    \begin{array}{l}
      \lambda_1: u_1;\ \interactBase{q}{r}:\, \lambda_1':u_1';\ X    \\
      \lambda_2: u_2;\ X
    \end{array}
  \right)
\end{displaymath}
%
 is strongly connected while 
%
$%  X\stackrel{\mathsf{def}}{=} 
  \interactBase{p}{q}:\,
  \left(
    \begin{array}{l}
      \lambda_1: u_1;\ \interactBase{r_1}{r_2}:\, \lambda_1':u_1';\ \CEnd.
    \end{array}
  \right) $ is not.
%
\begin{theorem}[EPP]\label{thm:epp}
  Given 
  % a well-formed
  an
  % z
  annotated strongly connected choreography $C$, we have that
  $(S,C) \red{\lambda} (S', C')$ iff
  $\pp_{\role q}\ \proj (\role q, C,\iota)\vdash S_{+} \red{\lambda}
  S_{+}'$.
\end{theorem}
\begin{proof}[Sketch] The proof must be separated into whether we deal
  with rates or probabilities. For Definition~\ref{def:projCTMC}, the
  proof proceeds by induction on the term $C$. For
  $C=\interact{p}{\role p_1,\ldots,\role p_n}$, the key case, we
  clearly have that, for any state $S$ there exists $S'$ such that
  $(S,C)\red{\lambda_j}(S',C_j)$. We need to show two things: first,
  that the projection $\pp_\role q\ \proj{(\role{q},C,\iota)}$ of $C$
  can make the same transition; second, that if the projection makes a
  transition, it must be corresponding to that of the choreography
  above.

  We now observe that the state of a generated CTMC is uniquely
  identified by the counter $\iota$.
  % Technically, this follows by the definition of projection and
  % annotation.
  The uniqueness of the label $a$ makes sure that all and only those
  modules involved in this interaction synchronise with this action
  (this shows from the rules in Figure~\ref{fig:semantics}).  As a
  consequence of this and since choreographies are strongly connected,
  the commands generated by this step of the translation are such that
  any state $S_+$ is exclusively going to enable these commands
  (because of the guard $s_{\role q}=\iota$) which obviously implies
  that it must be done with rate $\lambda_j$ applying the rules in
  Figure~\ref{fig:semantics}. The same argument can be applied for the
  opposite direction.
  % Note that having a unique label for each step of the choreography
  % is only necessary for recursion. Without recursion, having a
  % unique state identifying this particular step of the choreography
  % would have been sufficient.
  The other cases are similar. The case for DTMC is also similar.



  %       \newpage We prove each direction separately.
  %       \begin{itemize}
  %       \item (only if). Assume that
  %         % 
  %   $$(S, \interact{p}{\role p_1,\ldots,\role
  %     p_n})\red{\lambda_j}(S[\sigma(E_j)/x_j], C_j)$$
  %   % 
  %   and let us consider the projection of the term
  %   %
  %   $$\interact{p}{\role p_1,\ldots,\role p_n}$$
  %   % 
  %   Given some fresh $l_1,\ldots, l_n$, we generate the following
  %   commands for each $q$ in
  %   $\{\role p, \role p_1,\ldots,\role p_n\}$:
  %   % 
  %   $$
  %   \Big\{\commandBase {l_j} {\ s_{q}\!=\! s} {\kappa_j:\ s_{q}\!=\!
  %     s_{q} +1+\sum_{k=1}^{j-1}\mathsf{nodes}({C_k})}\ \&\ \projE
  %   {E_j}q\Big\}_{j\in J}
  %   $$
  %   %
  %   where $\kappa_j=\lambda_j$ if $q\neq\role p$ and $\kappa_j=1$
  %   otherwise. 
  %   % 
  %   Because the labels $l_j$ are fresh and the state counter is unique
  %   to this interaction, these are the only commands that can
  %   synchronise together: this can be shown from the rules defining
  %   the semantics of PRISM. Additionally, since all rates are set to 1
  %   besides the commands generated for role $\role p$, the transition
  %   will also be labelled with $\lambda_j$.


  % \item (if). In the opposite direction, we have that 
  %   % 
  %   $$\proj (*, \interact{p}{\role p_1,\ldots,\role p_n})\vdash
  %   S\uplus S_{+}\red{\lambda} S'\uplus S_{+}'$$
  %   %
  %   Again, given some fresh $l_1,\ldots, l_n$, the projection that
  %   reduces must be such that for each $q$ in
  %   $\{\role p, \role p_1,\ldots,\role p_n\}$:
  %   % 
  %   $$
  %   \Big\{\commandBase {l_j} {\ s_{q}\!=\! s} {\kappa_j:\ s_{q}\!=\!
  %     s_{q} +1+\sum_{k=1}^{j-1}\mathsf{nodes}({C_k})}\ \&\ \projE
  %   {E_j}q\Big\}_{j\in J}
  %   $$
  %   %
  %   where $\kappa_j=\lambda_j$ if $q\neq\role p$ and $\kappa_j=1$
  %   otherwise. 
  %   % 
  %   Again, the freshness of the labels $l_j$ together with the working
  %   states $s_qq$

  %   Because the labels $l_j$ are fresh and the state counter is unique
  %   to this interaction, these are the only commands that can
  %   synchronise together: this can be shown from the rules defining
  %   the semantics of PRISM. Additionally, since all rates are set to 1
  %   besides the commands generated for role $\role p$, the transition
  %   will also be labelled with $\lambda_j$.

  % \end{itemize}
\qed
\end{proof}

%%% Local Variables: 
%%% mode: latex
%%% TeX-master: "main"
%%% End:
