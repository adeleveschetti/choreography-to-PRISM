In this section, we provide a rigorous treatment of projection, which
constitutes the mapping from choreographies to the PRISM language.
%

\subsection{Mapping Choreographies to PRISM} 
%
The process of generating endpoint code from a choreography is
commonly referred to as {\em projection} in the literature.
Typically, projection is defined separately for each module appearing
in the choreography program, i.e., given a module (often called a
role) and a choreography, it generates the code for that particular
role.  However, this is not the case in our approach, as PRISM relies
solely on label synchronisation and a notion of state, which can be
modified through standard imperative assignments enabled by conditions
on the state.
%
Thus, our approach simulates a choreography interaction in PRISM by
using labels on which each involved module can synchronise and
leveraging the state to enable the correct commands at the appropriate
times.

Before formalising this idea, we make a slight abuse of notation by
assuming that each interaction in a choreography is annotated with a
unique label, which will be used by the projection. We refer to such a
choreography as an {\em annotated choreography}:
%
\begin{definition}[Annotated Choreography]
  An \emph{annotated choreography} is obtained from the choreography
  syntax by adding a label to each interaction:
  \[C ::= \interactl{p}a{\role p_1,\ldots,\role p_n} \mid\dots\]

  Given a choreography \( C \), its annotated counterpart \( C' \) is
  obtained by systematically assigning a unique label to each
  interaction. 
\end{definition}
This labelling above ensures that every interaction can be uniquely
identified and referenced in the corresponding endpoint projection.
For example, the choreography
% 
$\interactBase{p}{q} +\ \lambda_1: \CEnd \ +\ \lambda_2: 
% 
\ifTE 
%
{E}{p}{\CEnd}{\CEnd}$
% 
is going to be annotated as
$\interactBasel{p}{a}{q}+\ \lambda_1:\CEnd\ +\ \lambda_2:
%
\ifTEl {E}{}{p}{\CEnd}{\CEnd}$.

%
We now define projection. Since there are key differences between using probabilities and using rates, we proceed separately. We begin with choreographies that involve rates:
% 
\begin{definition}[Projection, CTMC]\label{def:projCTMC} Given an
  annotated choreography with rates $C$, a module $\role p$, and a
  natural number $\iota$, we define the function $\proj$ as:
  \begin{displaymath}\small
        \begin{array}{lr}

          \proj (\role q,\interactl{p}{a}{\role p_1,\ldots,\role p_n}, \iota)= 
          &  \boxed{\text{if }\role q=\role p}\\[2mm]
          \qquad
          % \{ 
          \Big\{\commandBase {a_j} {\ s_{q}\!=\! \iota} {\lambda_j:\ s_{q}\!=\!
          s_{q} +1+\sum_{k=1}^{j-1}\mathsf{nodes}({C_k})}\ \&\ \projE
          {u_j}q\Big\}_{j\in J}
          \\
          \qquad\cup\ \bigcup_{j} \proj (\role q, C_j, \iota+1+\sum_{k=1}^{j-1}\mathsf{nodes}({C_k}))
          %\\\\
        \end{array}
      \end{displaymath}


  \begin{displaymath}\small
        \begin{array}{lr}


          \proj (\role q,\interactl{p}{a}{\role p_1,\ldots,\role p_n}, \iota)= 
          &  \!\!\!\!\!\! \!\!\!\!\!\!\boxed{\text{if }\role q\in\{\role p_1,\ldots,\role p_n\}}\\[2mm]
          \qquad
          % \{ 
          \Big\{\commandBase {a_j} {\ s_{q}\!=\! \iota} {1:\ s_{q}\!=\!
          s_{q} +1+\sum_{k=1}^{j-1}\mathsf{nodes}({C_k})}\ \&\ \projE
          {u_j}q\Big\}_{j\in J}
          \\
          \qquad\cup\ \bigcup_{j} \proj (\role q, C_j, \iota+1+\sum_{k=1}^{j-1}\mathsf{nodes}({C_k}))

          \\\\

          \proj (\role q,\interactl{p}a{\role p_1,\ldots,\role p_n}, \iota)=
          % \ \proj (\role{q}, C_1, \iota)
          &  \hspace{-1.8cm}\boxed{\text{if }\role q\not\in\{\role p, 
            \role p_1,\ldots,\role p_n\}}
          \\[2mm]

          \qquad\bigcup_{j} \proj (\role q, C_j, \iota+\sum_{k=1}^{j-1}\mathsf{nodes}({C_k}))

          \\\\

          \proj (\role q,\ifTE {E}{p}{C_1}{C_2}, \iota) = 
          &  \boxed{\text{if }\role q=\role p}\\[2mm]
          \qquad\left\{ 
          \begin{array}{lll}
            \commandBase {} {s_{q}\!=\! \iota\ \&\ E}{\ 1: s'_{q}\!=\! \iota+1},\\ 
            \commandBase {} {s_{q}\!=\! \iota\ \&\ \mathsf{not}(E)}
            {\ 1: s'_{q}\!=\! \iota+\mathsf{nodes}(C_1)+1}
          \end{array}
          \right\}
          \\[3mm]
          \qquad\cup\quad \proj (\role{p}, C_1, \iota+1)
          \quad\cup\quad
          \proj (\role{p}, C_2, \iota+\mathsf{nodes}(C_1)+1)
          \\\\

          \proj (\role q,\ifTE {E}{p}{C_1}{C_2}, \iota) = 
          % \ \proj (\role{q}, C_1, \iota)
          &  \boxed{\text{if }\role q\neq\role p}\\[2mm]
          %
          % \quad\cup\quad
          % \proj (\role{q}, C_2, \iota+\mathsf{nodes}(C_1)+1)
          % \\[2mm]


          \qquad \proj (\role{p}, C_1, \iota)
          \quad\cup\quad
          \proj (\role{p}, C_2, \iota+\mathsf{nodes}(C_1))


          % \qquad\text{such that if } 
          % % 
          % \forall i. 
          % % \role q\in C_i
          % \proj (\role{q}, C_i, \iota)\neq\emptyset
          % \text{ then } \proj (\role{q}, C_1, \iota)= \proj (\role{q}, C_2, \iota)

          \\\\

          \proj (\role q,\CEnd, \iota) = \emptyset

          \\\\

          \proj (\role q, X, \iota) = 
          \commandBase {} {s_{q}\!=\! \iota}{\ 1: s'_{q}\!=\! \iota'}
          \qquad\text{ where } \textsf{defs}(X) = \iota'

        \end{array}
      \end{displaymath}
    \end{definition}
    We now examine the various cases in the definition above. The
    first three cases deal with the projection of an interaction. When
    projecting the first module \( \role p \), we create one command
    for each branch, assigning the corresponding rate. Note that this
    is possible because we are dealing with rates: in the case of
    probabilities this is not possible because we must always ensure
    that probabilities in a branching sum to 1. The reserved variable
    \( s_{\role q} \) is the counter for this module, indicating how
    far we are in the computation. To ensure consistency for
    subsequent statements of the subterm choreographies, we use the
    function \( \mathsf{nodes}(C) \), which returns the number of
    nodes in \( C \), i.e., the number of steps in the projection
    function. Formally, it can be defined as: 
    \begin{displaymath}
      \begin{array}{llll}
        \mathsf{nodes}(\interact{p}{\role p_1,\ldots,\role p_n})\ =\ 1+\sum_{j\in J}\mathsf{nodes}(C_j)\\
        \mathsf{nodes}(\ifTE {E}{p}{C_1}{C_2})\ =\ 1+\mathsf{nodes}(C_1)+\mathsf{nodes}(C_2)\\
        \mathsf{nodes}(X)\ =\ 
        \mathsf{nodes}(\CEnd)\ =\ 1
      \end{array}
    \end{displaymath}
    % Specifically, it counts the steps (or interaction points) within
    % each branch of the choreography. In the context of recursive
    % choreographies, it also counts the recursive steps, helping to
    % compute how far into the choreography a particular interaction
    % or branch has progressed.

    Obviously, when projecting the next branch, we need to consider
    all other possible branches that may have already been
    projected. Intuitively, a label and an integer (denoted by
    \( \iota \)) identify a node in the abstract syntax tree of a
    choreography. Also, from a label \( a \), we generate distinct
    sublabels \( a_j \) by simply adding an index \( j \).
    % For the sake of space, we do not define the function precisely,
    % but we observe that it could also be easily defined via the
    % label annotations.

    The second case defines the projection of an interaction for one
    of the modules \( \{\role p_1, \ldots, \role p_n\} \). Similarly
    to the previous case, we define a command for each branch of the
    interaction. However, the rate of each command is set to 1,
    ensuring that each branch synchronises with probability
    \( \lambda_j \times 1 \) (see rule \( \mathsf{(P_2)} \) in
    Figure~\ref{fig:semantics}).
    %
    The third case is the one when we are projecting a module that is
    not in the set $\{\role p, \role p_1,\ldots,\role p_n\}$. % Our
    % projection takes a standard approach that requires each branch to
    % project the same~\cite{HYC16}. A generalisation of this, which may
    % require a merging operation~\cite{CHY12} and perhaps a more
    % intrinsic treatment of rates and probabilities, is left as future
    % work.
    % 
    % \marco{BEST TO COMMENT ON SELF-INTERACTION HERE and RELATED TO
    %   PROJECTION: The paper does nor require the usual condition of
    %   non-self-interaction (as evident in Listing 1.4). Additionally,
    %   clarification is needed on how projection works in cases where
    %   non-self-interaction is not enforced. Are the projection rules
    %   ordered? }
    The if-then-else construct is only interesting for module
    $\role p$ where the module makes an internal choice based on the
    evaluation of the guard $E$. % As in the previous case, the
    % other modules must project the same in $C_1$ and $C_2$.
    For recursive calls, we generate a command that resets the counter
    to a distinct value given by the auxiliary function
    $\textsf{defs}$. 
    %As an example, we can apply the projection
    %function to Example~\ref{example2} and obtain the PRISM modules
    %from Example~\ref{example3}.

\begin{example}\label{example-proj}
  In order to show how our projection works, consider the following
  example in which we apply the projection to Example~\ref{example2}
  and obtain the PRISM modules from Example~\ref{example3}.
  
  In Example \ref{example2}, we defined a recursive choreography in
  which role \(\role{p}\) interacts with role \(\role{q}\) through two
  branches. Its annotated form can be written as:
  \begin{displaymath}
    \begin{array}{lll}
      C \stackrel{\mathsf{def}}{=} \interactBasel{p}a{q}
      \left\{
      \begin{array}{lll}
        \lambda_1: (x'=1)\&(y'=2);\ C
        \\
        \lambda_2: (x'=3)\&(y'=1);\ C
      \end{array}
      \right.
    \end{array}
  \end{displaymath}
  % 
  From label $a$, each branch of the choreography can be referred with
  a unique label, say \(a_1\) for the first branch and \(a_2\) for the
  second). Then, the state of each module can be tracked by a counter,
  say \(s_{\role p}\) and \(s_{\role q}\).
  %
  Then, the various steps of projection can be summarised as follows:
  \begin{enumerate}
  \item \textbf{Computing the States.} Starting from an initial
    counter value of \( s_{\role{p}} = 0 \) and
    \( s_{\role{q}} = 0 \), we apply our projection function to
    determine the new state values for each interaction. The auxiliary
    function \(\mathsf{nodes}(C)\) counts the steps within each branch
    of the choreography. Since each branch in Example \ref{example2}
    consists of a single interaction followed by a recursive call to
    \(C\), we can compute the state updates as follows.
 
    For the \(j^{\text{th}}\) branch, the new state is given by:
    \[
      \iota + 1 + \sum_{k=1}^{j-1} \mathsf{nodes}(C_k)
    \]
 
    Applying this formula to our example:
    \begin{enumerate}
    \item Branch \(\lambda_1\) (label \(a\)):
      \begin{itemize}
      \item The initial state is \( s_{\role{p}} = 0 \) and
        \( s_{\role{q}} = 0 \).
      \item Since \(\mathsf{nodes}(C_1) = 1\) (one interaction before
        the recursive call), we compute:
        \[
          s'_{\role{p}} = 0 + 1 = 1, \quad s'_{\role{q}} = 0 + 1 = 1
        \]
      \item The update rule for this branch is:
        \[
          s_{\role{p}}' = 1, \quad s_{\role{q}}' = 1
        \]
      \end{itemize}
    \item Branch \(\lambda_2\) (label \(b\)):
      \begin{itemize}
      \item Again, starting from \( s_{\role{p}} = 0 \) and
        \( s_{\role{q}} = 0 \).
      \item Since \(\mathsf{nodes}(C_2) = 1\), we compute:
        \[
          s'_{\role{p}} = 0 + 2 = 2, \quad s'_{\role{q}} = 0 + 2 = 2
        \]
      \item The update rule for this branch is:
        \[
          s_{\role{p}}' = 2, \quad s_{\role{q}}' = 2
        \]
      \end{itemize}
    \end{enumerate}
    Thus, each branch gets a unique state assigned to ensure that only
    one transition can be taken at a time in the PRISM model. The
    recursive nature of the choreography ensures that the state
    counters return to 0 after completing each interaction, allowing
    the process to repeat.

  \item \textbf{Assigning Unique Labels.}  For each branch, a unique
    label is generated from the interaction’s base label. For example,
    the first branch is assigned label \(a\) and the second
    \(b\). These labels serve as synchronization points between the
    interacting modules. In our projection, role \(\role{p}\) (the
    initiator) uses the corresponding rates (e.g. \(\mu_1\) and
    \(\mu_2\) for branches 1 and 2) while role \(\role{q}\) uses a
    fixed rate (1) for synchronization.

  \item \textbf{From Choreography to PRISM.} As detailed in
    Example \ref{example3}, the projected PRISM network is obtained by
    creating commands for each branch in both roles. Each command is
    guarded by a condition on the state counter (for instance,
    \(s_{\role p}=0\) for the first branch) and includes the update
    that sets the counter to the new state computed above. The modules
    for \(\role{p}\) and \(\role{q}\) synchronize on the unique labels
    \(a\) and \(b\), and the overall system’s global transitions
    (e.g. \(F_1\) and \(F_2\)) are derived by the composition of these
    synchronized commands. The rates of these transitions are computed
    as the product of the individual rates (i.e.,
    \(\lambda_i = \mu_i * \gamma_i\)).
  \end{enumerate}

\end{example}

    As hinted above, the projection in Definition~\ref{def:projCTMC}
    would be incorrect if instead of using rates we used
    probabilities. This is simply because we cannot force both
    $\role p$ and $\{\role p_1,\ldots,\role p_n\}$ to take the same
    branch with the probability distribution of the $\lambda_i$'s. To
    fix this problem, we have the following definition instead:
    % 
    \begin{definition}[Projection, DTMC]\label{def:projDTMC} Given an annotated
      choreography with probabilities $C$, a module $\role p$, and a
      natural number $\iota$, we define $\proj$ as:
      \begin{displaymath}\small
        \begin{array}{lr}

          \proj (\role q,\interact{p}{\role p_1,\ldots,\role p_n}, \iota)= 
          &  \boxed{\text{if }\role q=\role p}\\[2mm]
          \qquad
          \left\{
          \begin{array}{lll}
            \commandBase {} {s_{q}\!=\! \iota}{\ \sum_{j\in J} \lambda_j: s'_{q}\!=\! \iota+1+j},\\ 
            \{\commandBase {l_j} {s_{q}\!=\! \iota+1+j}
            \ 1: s'_{q}\!=\! \iota+1+\sum_{k=1}^{j-1}\mathsf{nodes}({C_k})\ \&\ \projE
            {u_j}q\}_{j\in J}
          \end{array}
          \right\}
          \\
          \qquad\cup\ \bigcup_{j} \proj (\role q, C_j, s+1+\sum_{k=1}^{j-1}\mathsf{nodes}({C_k}))
          \\\\

        \end{array}
      \end{displaymath}
      The other cases of the definition are equivalent to those in
      Definition \ref{def:projCTMC}.
    \end{definition}
    The fix is immediate: module $\role p$ takes a (probabilistic)
    internal decision on the $j^{\text{th}}$ branch and then
    synchronises on label $l_j$ with $\{\role p_1,\ldots,\role p_n\}$.       

\smallskip

\subsection{Correctness.} Our projection operations are correct with
respect to the semantics of choreographies and PRISM. In the sequel,
$S_+$ is a state $S$ extended with the extra variables $s_{\role q}$
(for each module in a choreography) used by the projection. In the
projection, we utilize alphabetized parallel composition $\pp$,
wherein modules synchronise solely on labels that appear in both
modules.
% Moreover, we say that a choreography is {\em well-formed} if the
% behaviour in each branching for modules who do not participate to
% the interaction is identical.
Additionally, we use the notion of {\em strongly connected} choreography, defined as follows:

\begin{definition}[Strongly Connected Choreography]
  A choreography is said to be \emph{strongly connected} if it satisfies the following conditions:
  \begin{enumerate}
      \item Each interaction shares at least one module with the subsequent interaction~\cite{CHY12}.
      \item In every branch of a probabilistic choice or an if-then-else involving modules $\role p_1, \ldots, \role p_n$, the first action (if any) of every other module $\role q$ must be an interaction with one of $\role p_1, \ldots, \role p_n$, possibly after unfolding recursive calls.
  \end{enumerate}
\end{definition}

%a choreography is {\em strongly connected},  
%
%if (i) subsequent interactions share some modules~\cite{CHY12} and,
%
%(ii) in every branch of a probabilistic choice or an if-then-else
%involving modules $\role p_1, \ldots, \role p_n$, the fist action (if
%any) of every other module $\role q$ must be an interaction with
%$\role p_1, \ldots, \role p_n$ (up to unfolding of recursive
%calls). 
%

For example, the choreography
\begin{displaymath}\small
  X\stackrel{\mathsf{def}}{=} 
  \interactBase{p}{q}:\,
  \left(
    \begin{array}{l}
      \lambda_1: u_1;\ \interactBase{q}{r}:\, \lambda_1':u_1';\ X    \\
      \lambda_2: u_2;\ X
    \end{array}
  \right)
\end{displaymath}
%
is strongly connected while 
% 
$%  X\stackrel{\mathsf{def}}{=} 
\interactBase{p}{q}:\,
\left(
  \begin{array}{l}
    \lambda_1: u_1;\ \interactBase{r_1}{r_2}:\, \lambda_1':u_1';\ \CEnd.
  \end{array}
\right) $ is not.
% 



\begin{lemma}
  The state of a projected choreography is uniquely identified by the
  counter $\iota$.
\end{lemma}

\begin{theorem}[Projection]\label{thm:epp}
  Given an annotated strongly connected choreography $C$, we have that
  $(S,C) \red{\lambda} (S', C')$ iff
  $\pp_{\role q}\ \proj (\role q, C,\iota)\vdash S_{+} \red{\lambda}
  S_{+}'$.
\end{theorem}
\begin{proof}
  The proof proceeds by induction on the syntax of $C$.
  \begin{itemize}
  \item $C=\interact{p}{\role p_1,\ldots,\role p_n}$.  By (the only
    applicable) rule \textsf{(Interact)}, we have that
    % 
    \[
      (S, \interact{p}{\role p_1,\ldots,\role p_n})
      \red{\lambda_j} (S[u_j], C_j) 
    \]
    % 
    By definition of projection, we obtain the following PRISM
    commands. Role $\role p$ is projected as:
    \[
      \Big\{\commandBase {a_j} {\ s_{q}\!=\! \iota} {\lambda_j:\ s_{q}\!=\!
        s_{q} +1+\sum_{k=1}^{j-1}\mathsf{nodes}({C_k})}\ \&\ \projE
      {u_j}q\Big\}_{j\in J}
      \ \cup\ \bigcup_{j} \proj (\role q, C_j,
      \iota+1+\sum_{k=1}^{j-1}\mathsf{nodes}({C_k}))
    \]
    Roles $\role{p_i}$ are projected as:
    \[
      \Big\{\commandBase {a_j} {\ s_{q}\!=\! \iota} {1:\ s_{q}\!=\!
        s_{q} +1+\sum_{k=1}^{j-1}\mathsf{nodes}({C_k})}\ \&\ \projE
      {u_j}q\Big\}_{j\in J}
      \ \cup\ \bigcup_{j} \proj (\role q, C_j, \iota+1+\sum_{k=1}^{j-1}\mathsf{nodes}({C_k}))
    \]
    %
    while any other role is projected as: 
    \[\bigcup_{j} \proj (\role q, C_j, \iota+\sum_{k=1}^{j-1}\mathsf{nodes}({C_k}))
    \]

  \item $C=\ifTE {E}{p}{C_1}{C_2}$.

  \item $C=X$.

  \item $C=\CEnd$.

  \end{itemize}




  > [Sketch] The proof must be separated into whether we deal with
  rates or probabilities. For Definition~\ref{def:projCTMC}, the proof
  proceeds by induction on the term $C$. For
  $C=\interact{p}{\role p_1,\ldots,\role p_n}$, the key case, we
  clearly have that, for any state $S$ there exists $S'$ such that
  $(S,C)\red{\lambda_j}(S',C_j)$. We need to show two things: first,
  that the projection $\pp_\role q\ \proj{(\role{q},C,\iota)}$ of $C$
  can make the same transition; second, that if the projection makes a
  transition, it must be corresponding to that of the choreography
  above.

  We now observe that the state of a generated CTMC is uniquely
  identified by the counter $\iota$.  The uniqueness of the label $a$
  makes sure that all and only those modules involved in this
  interaction synchronise with this action (this shows from the rules
  in Figure~\ref{fig:semantics}).  As a consequence of this and since
  choreographies are strongly connected, the commands generated by
  this step of the translation are such that any state $S_+$ is
  exclusively going to enable these commands (because of the guard
  $s_{\role q}=\iota$) which obviously implies that it must be done
  with rate $\lambda_j$ applying the rules in
  Figure~\ref{fig:semantics}. The same argument can be applied for the
  opposite direction.
  % Note that having a unique label for each step of the choreography
  % is only necessary for recursion. Without recursion, having a
  % unique state identifying this particular step of the choreography
  % would have been sufficient.
  The other cases are similar. The case for DTMC is also similar.



  %       \newpage We prove each direction separately.
  %       \begin{itemize}
  %       \item (only if). Assume that
  %         % 
  %   $$(S, \interact{p}{\role p_1,\ldots,\role
  %     p_n})\red{\lambda_j}(S[\sigma(E_j)/x_j], C_j)$$
  %   % 
  %   and let us consider the projection of the term
  %   %
  %   $$\interact{p}{\role p_1,\ldots,\role p_n}$$
  %   % 
  %   Given some fresh $l_1,\ldots, l_n$, we generate the following
  %   commands for each $q$ in
  %   $\{\role p, \role p_1,\ldots,\role p_n\}$:
  %   % 
  %   $$
  %   \Big\{\commandBase {l_j} {\ s_{q}\!=\! s} {\kappa_j:\ s_{q}\!=\!
  %     s_{q} +1+\sum_{k=1}^{j-1}\mathsf{nodes}({C_k})}\ \&\ \projE
  %   {E_j}q\Big\}_{j\in J}
  %   $$
  %   %
  %   where $\kappa_j=\lambda_j$ if $q\neq\role p$ and $\kappa_j=1$
  %   otherwise. 
  %   % 
  %   Because the labels $l_j$ are fresh and the state counter is unique
  %   to this interaction, these are the only commands that can
  %   synchronise together: this can be shown from the rules defining
  %   the semantics of PRISM. Additionally, since all rates are set to 1
  %   besides the commands generated for role $\role p$, the transition
  %   will also be labelled with $\lambda_j$.


  % \item (if). In the opposite direction, we have that 
  %   % 
  %   $$\proj (*, \interact{p}{\role p_1,\ldots,\role p_n})\vdash
  %   S\uplus S_{+}\red{\lambda} S'\uplus S_{+}'$$
  %   %
  %   Again, given some fresh $l_1,\ldots, l_n$, the projection that
  %   reduces must be such that for each $q$ in
  %   $\{\role p, \role p_1,\ldots,\role p_n\}$:
  %   % 
  %   $$
  %   \Big\{\commandBase {l_j} {\ s_{q}\!=\! s} {\kappa_j:\ s_{q}\!=\!
  %     s_{q} +1+\sum_{k=1}^{j-1}\mathsf{nodes}({C_k})}\ \&\ \projE
  %   {E_j}q\Big\}_{j\in J}
  %   $$
  %   %
  %   where $\kappa_j=\lambda_j$ if $q\neq\role p$ and $\kappa_j=1$
  %   otherwise. 
  %   % 
  %   Again, the freshness of the labels $l_j$ together with the working
  %   states $s_qq$

  %   Because the labels $l_j$ are fresh and the state counter is unique
  %   to this interaction, these are the only commands that can
  %   synchronise together: this can be shown from the rules defining
  %   the semantics of PRISM. Additionally, since all rates are set to 1
  %   besides the commands generated for role $\role p$, the transition
  %   will also be labelled with $\lambda_j$.

  % \end{itemize}
\qed
\end{proof}

%%% Local Variables: 
%%% mode: latex
%%% TeX-master: "main"
%%% End:
