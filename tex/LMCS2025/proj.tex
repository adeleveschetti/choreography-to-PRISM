In this section, we provide a rigorous treatment of projection, which
constitutes the mapping from choreographies to the PRISM language.
%

\subsection{Mapping Choreographies to PRISM} 
%
The process of generating endpoint code from a choreography is
commonly referred to as {\em projection}.  Typically, projection is
defined separately for each module appearing in the choreography
program, i.e., given a module (often called a role) and a
choreography, it generates the code for that particular role.
However, this is not the case in our approach, as PRISM relies solely
on label synchronisation and a notion of state, which can be modified
through standard imperative assignments enabled by conditions on the
state.
%
Thus, our approach simulates a choreography interaction in PRISM by
using (i) labels on which each involved module can synchronise and
(ii) the state to enable the correct commands at the appropriate
time.

Before formalising this idea, we make a slight abuse of notation by
assuming that each interaction in a choreography is annotated with a
label. We refer to such a choreography as an {\em annotated
  choreography}:
%
\begin{definition}[Annotated Choreography]
  An \emph{annotated choreography} is obtained from the choreography
  syntax by adding a label to each interaction:
  \[C ::= \interactl{p}a{\role p_1,\ldots,\role p_n} \mid\dots\]
\end{definition}
%
The intuition behind annotations is that they allow us to identify a
particular interaction in the choreography, enabling all modules
involved to synchronise. Since these annotations must be unique, we
make the following assumption, which is key to our results:
\begin{assumption}
  Each annotation in an annotated choreography occurs exactly once.
\end{assumption}
% This assumption above ensures that every interaction can be uniquely
% identified and referenced in the corresponding endpoint projection. 
As an example, the following choreography is annotated correctly:
% 
\[
  \interactBasel{p}a{q,r}:
  \left(
    \begin{array}{ll}
      \lambda_1: \ \CEnd \\
      \quad+\\
      \lambda_2:\
      % 
      \ifTE {E}{p}{\ \interactBasel{p}b{q}: \lambda_1: \CEnd\ }{\CEnd}
    \end{array}
  \right)
\]
  % 
while, the one below is not: 
% 
\[
  \interactBasel{p}a{q,r}:
  \left(
    \begin{array}{ll}
      \lambda_1: \ \CEnd \\
      \quad+\\
      \lambda_2:\
      % 
      \ifTE {E}{p}{\ \interactBasel{p}a{q}: \lambda_1: \CEnd\ }{\CEnd}
    \end{array}
  \right)
\]
% 

\smallskip

%
In order to ensure consistency for subsequent statements in a
choreography, our definition of projection uses the function
\( \mathsf{nodes}(C) \), which returns the number of nodes in the abstract
syntax tree (AST) of \( C \). Formally, it is defined as:
\begin{displaymath}
  \begin{array}{llll}
    \mathsf{nodes}(\interact{p}{\role p_1,\ldots,\role p_n})\ =\ 1+\sum_{j\in J}\mathsf{nodes}(C_j)\\
    \mathsf{nodes}(\ifTE {E}{p}{C_1}{C_2})\ =\ 1+\mathsf{nodes}(C_1)+\mathsf{nodes}(C_2)\\
    \mathsf{nodes}(X)\ =\ 
    \mathsf{nodes}(\CEnd)\ =\ 1
  \end{array}
\end{displaymath}
% Specifically, it counts the steps (or interaction points) within
% each branch of the choreography. In the context of recursive
% choreographies, it also counts the recursive steps, helping to
% compute how far into the choreography a particular interaction or
% branch has progressed.

We now define the projection function. Since there are key differences
between using probabilities and using rates, we proceed separately. We
begin with choreographies that involve rates:
% 
\begin{definition}[Projection, CTMC]\label{def:projCTMC} Given an
  annotated choreography with rates $C$, a module $\role p$, a natural
  number $\iota$, and $J=\{1,\dots,m\}$, we define the function
  $\proj$ as:
  \begin{displaymath}\small
        \begin{array}{lr}

          \proj (\role q,\interactl{p}{a}{\role p_1,\ldots,\role p_n}, \iota)= 
          &  \boxed{\text{if }\role q=\role p}\\[2mm]
          \qquad
          % \{ 
          \Big\{\commandBase {a_j} {\ s_{q}\!=\! \iota} {\lambda_j:\ s_{q}'\!=\!
          s_{q} +1+\sum_{k=1}^{j-1}\mathsf{nodes}({C_k})}\ \&\ \projE
          {u_j}q\Big\}_{j\in J}
          \\
          \qquad\cup\ \bigcup_{j} \proj (\role q, C_j, \iota+1+\sum_{k=1}^{j-1}\mathsf{nodes}({C_k}))
          
          \\\\
          % \end{array}
          % \end{displaymath}
          
          % \begin{displaymath}\small
          %   \begin{array}{lr}
          
          \proj (\role q,\interactl{p}{a}{\role p_1,\ldots,\role p_n}, \iota)= 
          &  \!\!\!\!\!\! \!\!\!\!\!\!\boxed{\text{if }\role q\in\{\role p_1,\ldots,\role p_n\}}\\[2mm]
          \qquad
          % \{ 
          \Big\{\commandBase {a_j} {\ s_{q}\!=\! \iota} {1:\ s_{q}'\!=\!
          s_{q} +1+\sum_{k=1}^{j-1}\mathsf{nodes}({C_k})}\ \&\ \projE
          {u_j}q\Big\}_{j\in J}
          \\
          \qquad\cup\ \bigcup_{j} \proj (\role q, C_j, \iota+1+\sum_{k=1}^{j-1}\mathsf{nodes}({C_k}))

          \\\\

          \proj (\role q,\interactl{p}a{\role p_1,\ldots,\role p_n}, \iota)=
          % \ \proj (\role{q}, C_1, \iota)
          &  \hspace{-1.8cm}\boxed{\text{if }\role q\not\in\{\role p, 
            \role p_1,\ldots,\role p_n\}}
          \\[2mm]

          \qquad\bigcup_{j} \proj (\role q, C_j, \iota+\sum_{k=1}^{j-1}\mathsf{nodes}({C_k}))

          \\\\

          \proj (\role q,\ifTE {E}{p}{C_1}{C_2}, \iota) = 
          &  \boxed{\text{if }\role q=\role p}\\[2mm]
          \qquad\left\{ 
          \begin{array}{lll}
            \commandBase {} {s_{q}\!=\! \iota\ \&\ E}{\ 1: s'_{q}\!=\! \iota+1},\\ 
            \commandBase {} {s_{q}\!=\! \iota\ \&\ \mathsf{not}(E)}
            {\ 1: s'_{q}\!=\! \iota+\mathsf{nodes}(C_1)+1}
          \end{array}
          \right\}
          \\[3mm]
          \qquad\cup\quad \proj (\role{p}, C_1, \iota+1)
          \quad\cup\quad
          \proj (\role{p}, C_2, \iota+\mathsf{nodes}(C_1)+1)
          \\\\

          \proj (\role q,\ifTE {E}{p}{C_1}{C_2}, \iota) = 
          % \ \proj (\role{q}, C_1, \iota)
          &  \boxed{\text{if }\role q\neq\role p}\\[2mm]
          %
          % \quad\cup\quad
          % \proj (\role{q}, C_2, \iota+\mathsf{nodes}(C_1)+1)
          % \\[2mm]


          \qquad \proj (\role{q}, C_1, \iota)
          \quad\cup\quad
          \proj (\role{q}, C_2, \iota+\mathsf{nodes}(C_1))


          % \qquad\text{such that if } 
          % % 
          % \forall i. 
          % % \role q\in C_i
          % \proj (\role{q}, C_i, \iota)\neq\emptyset
          % \text{ then } \proj (\role{q}, C_1, \iota)= \proj (\role{q}, C_2, \iota)

          \\\\

          \proj (\role q,\CEnd, \iota) = \emptyset

          \\\\

          \proj (\role q, X, \iota) = 
          \{ \commandBase {} {s_{q}\!=\! \iota}{\ 1: s'_{q}\!=\! \iota'} \}
          \qquad\text{ where } \textsf{defs}(X) = \iota'

        \end{array}
      \end{displaymath}
    \end{definition}
    We examine the various cases in the definition above. The first
    three cases deal with the projection of an interaction. When
    projecting the first module \( \role p \), we create one command
    $\commandBase {a_j} {s_{\role q}\!=\! \iota} {\lambda_j:\ s_{\role
        q}'\!=\!  s_{\role q}
      +1+\sum_{k=1}^{j-1}\mathsf{nodes}({C_k})}\ \&\ \projE
    {u_j}{\role q}$ for each branch such that
    % 
    \begin{itemize}
    \item the label $a_j$ and its uniqueness ensure that all modules
      take the same branch;
    \item the guard $s_{\role q}\!=\! \iota$ ensures that these
      commands are only executable when the system has reached this
      particular state, identified by the reserved variable
      $s_{\role q}$;
    \item the rate $\lambda_j$ is the rate that appears in the same
      branch of the choreography
    \item the successor state is determined by incrementing
      $s_{\role q}$, depending on which branch was selected—the
      function $\mathsf{nodes}$ ensures that every interaction in all
      branches is assigned to a different counter value, thereby also
      discarding all branches that are not selected;
    \item the projected update $\projE {u_j}{\role q}$ acts as a
      filter on the list of updates in $u_j$, ensuring that only those
      variables local to $\role q$ are updated.
    \end{itemize}
    %
    %
    Note that the translation works only with rates. In the case of
    probabilities, the definition above is incorrect, as we must
    ensure that the probabilities in a branching sum to 1.
    % Obviously, when projecting the next branch, we need to consider
    % all other possible branches that may have already been
    % projected. Intuitively, a label and an integer (denoted by
    % \( \iota \)) identify a node in the abstract syntax tree of a
    % choreography. Also, from a label \( a \), we generate distinct
    % sublabels \( a_j \) by simply adding an index \( j \).
    % % For the sake of space, we do not define the function precisely,
    % % but we observe that it could also be easily defined via the
    % % label annotations.

    The second case defines the projection of an interaction for one
    of the modules \( \{\role p_1, \ldots, \role p_n\} \). Similarly
    to the previous case, we define a command for each branch of the
    interaction. However, the rate of each command is set to 1,
    ensuring that each branch synchronises with probability
    \( \lambda_j \cdot 1 \) (see rule \( \mathsf{(P_2)} \) in
    Figure~\ref{fig:semantics}).
    %
    The third case is the one when we are projecting a module that is
    not in the set $\{\role p, \role p_1,\ldots,\role p_n\}$. % Our
    % projection takes a standard approach that requires each branch to
    % project the same~\cite{HYC16}. A generalisation of this, which may
    % require a merging operation~\cite{CHY12} and perhaps a more
    % intrinsic treatment of rates and probabilities, is left as future
    % work.
    % 
    % \marco{BEST TO COMMENT ON SELF-INTERACTION HERE and RELATED TO
    %   PROJECTION: The paper does nor require the usual condition of
    %   non-self-interaction (as evident in Listing 1.4). Additionally,
    %   clarification is needed on how projection works in cases where
    %   non-self-interaction is not enforced. Are the projection rules
    %   ordered? }
    The if-then-else construct focuses on the module $\role p$ where
    the guard $E$ must be evaluated. As a consequence, we do not need
    to have any label for synchronisation. 
    % As in the previous case, the other modules must project the same
    % in $C_1$ and $C_2$.
    For recursive calls, we generate a command that resets the counter
    to a distinct value given by the auxiliary function
    $\textsf{defs}$. 
    %As an example, we can apply the projection
    %function to Example~\ref{example2} and obtain the PRISM modules
    %from Example~\ref{example3}.

\begin{example}\label{example-proj}
  %
  In order to show how our projection works, consider the following
  example in which we apply the projection to the choreography in
  Example~\ref{example2} and obtain the PRISM modules from
  Example~\ref{example3}.
  %
  In Example \ref{example2}, we defined a recursive choreography in
  which role \(\role{p}\) interacts with role \(\role{q}\) through two
  branches. Its annotated form can be written as:
  \begin{displaymath}
    \begin{array}{lll}
      C \stackrel{\mathsf{def}}{=} \interactBasel{p}a{q}
      \left\{
      \begin{array}{lll}
        \lambda_1: (x'=1)\&(y'=2);\ C
        \\
        \lambda_2: (x'=3)\&(y'=1);\ C
      \end{array}
      \right.
    \end{array}
  \end{displaymath}
  % 
  From label $a$, each branch of the choreography is identified by a
  unique label, say \(a_1\) for the first branch and \(a_2\) for the
  second. Then, the state of each module can be tracked by counters
  \(s_{\role p}\) and \(s_{\role q}\).
  % 
  The various steps of the projection can then be summarised as
  follows:
  \begin{enumerate}
  \item \textbf{Computing the States.} Starting from the initial
    counter values \( s_{\role{p}} = 0 \) and \( s_{\role{q}} = 0 \),
    we apply our projection function to determine the new state values
    for each interaction. The auxiliary function \(\mathsf{nodes}(C)\)
    counts the steps within each branch of the choreography. Since
    each branch in Example \ref{example2} consists of a single
    interaction followed by a recursive call to \(C\), we can compute
    the state updates as follows.
    %
    In particular, for the \(j^{\text{th}}\) branch, the new state is
    given by $ \iota + 1 + \sum_{k=1}^{j-1} \mathsf{nodes}(C_k) $.
    %
    Applying this formula to our example:
    \begin{enumerate}
    \item Branch \(\lambda_1\) (label \(a_1\)):
      \begin{itemize}
      \item The initial state is \( s_{\role{p}} = 0 \) and
        \( s_{\role{q}} = 0 \).
      \item in the first branch ($j=1$), the sum
        $\sum_{k=1}^{j-1} \mathsf{nodes}(C_k)$ is equal to $0$,
        therefore
        \[
          s'_{\role{p}} = 0 + 1 = 1, \quad s'_{\role{q}} = 0 + 1 = 1
        \]
      \item The update rule for this branch is
        $s_{\role{p}}' = 1 \quad \& \quad s_{\role{q}}' = 1$
      \end{itemize}
    \item Branch \(\lambda_2\) (label \(a_2\)):
      \begin{itemize}
      \item Again, starting from \( s_{\role{p}} = 0 \) and
        \( s_{\role{q}} = 0 \).
      \item in the second branch ($j=2$), the sum
        $\sum_{k=1}^{j-1} \mathsf{nodes}(C_k)$ is equal to $1$ because
        the first branch contains only the recursive call to
        $C$. Hence,
        \[
          s'_{\role{p}} = 0 + 2 = 2, \quad s'_{\role{q}} = 0 + 2 = 2
        \]
      \item The update rule for this branch is
        $s_{\role{p}}' = 2\quad\& \quad s_{\role{q}}' = 2$
      \end{itemize}
    \end{enumerate}
    Thus, each branch is assigned a unique state to ensure that only
    one transition can be taken at a time in the PRISM model. The
    recursive nature of the choreography ensures that the state
    counters return to 0 after each interaction, allowing the process
    to repeat.

  \item \textbf{Assigning Unique Labels.}  For each branch, a unique
    label is generated from the interaction’s base label. In this
    example, the first branch is assigned label \(a_1\) and the second
    \(a_2\). These labels serve as synchronisation points between the
    interacting modules. In our projection, role \(\role{p}\) (the
    initiator) uses the corresponding rates (e.g. \(\mu_1\) and
    \(\mu_2\) for branches 1 and 2) while role \(\role{q}\) uses a
    fixed rate (equal to 1) for synchronisation.

  \item \textbf{From Choreography to PRISM.} As detailed in
    Example \ref{example3}, the projected PRISM network is obtained by
    creating commands for each branch in both roles. Each command is
    guarded by a condition on the state counter (for instance,
    \(s_{\role p}=0\) for the first branch) and includes the update
    that sets the counter to the new state computed above. The modules
    for \(\role{p}\) and \(\role{q}\) synchronise on the unique labels
    \(a\) and \(b\), and the overall system's global transitions
    (e.g. \(F_1\) and \(F_2\)) are derived by the composition of these
    synchronised commands. The rates of these transitions are computed
    as the product of the individual rates (i.e.,
    \(\lambda_i = \mu_i * \gamma_i\)).
  \end{enumerate}

\end{example}

    As hinted above, the projection in Definition~\ref{def:projCTMC}
    would be incorrect if instead of using rates we used
    probabilities. This is simply because we cannot force both
    $\role p$ and $\{\role p_1,\ldots,\role p_n\}$ to take the same
    branch with the probability distribution of the $\lambda_i$'s. To
    fix this problem, we have the following definition instead:
    % 
    \begin{definition}[Projection, DTMC]\label{def:projDTMC} Given an annotated
      choreography with probabilities $C$, a module $\role p$, and a
      natural number $\iota$, we define $\proj$ as:
      \begin{displaymath}\small
        \begin{array}{lr}

          \proj (\role q,\interact{p}{\role p_1,\ldots,\role p_n}, \iota)= 
          &  \boxed{\text{if }\role q=\role p}\\[2mm]
          \qquad
          \left\{
          \begin{array}{lll}
            \commandBase {} {s_{q}\!=\! \iota}{\ \sum_{j\in J} \lambda_j: s'_{q}\!=\! \iota+1+j},\\ 
            \{\commandBase {l_j} {s_{q}\!=\! \iota+1+j}
            \ 1: s'_{q}\!=\! \iota+1+\sum_{k=1}^{j-1}\mathsf{nodes}({C_k})\ \&\ \projE
            {u_j}q\}_{j\in J}
          \end{array}
          \right\}
          \\
          \qquad\cup\ \bigcup_{j} \proj (\role q, C_j, s+1+\sum_{k=1}^{j-1}\mathsf{nodes}({C_k}))
          \\\\

        \end{array}
      \end{displaymath}
      The other cases of the definition are equivalent to those in
      Definition \ref{def:projCTMC}.
    \end{definition}
    The fix is immediate: module $\role p$ takes a (probabilistic)
    internal decision on the $j^{\text{th}}$ branch and then
    synchronises on label $l_j$ with $\{\role p_1,\ldots,\role p_n\}$.       

\smallskip

\subsection{Correctness.} Our projection operations are correct with
respect to the semantics of choreographies and PRISM.
%
In order to state our main result, we need to use the notions of {\em
  head modules} and {\em strongly connected} choreography. The former
identifies the modules involved in the next action of a choreography:
\begin{definition}[Head Modules]
  The function $\hmodules$ is defined as follows:
  \begin{displaymath}
    \begin{array}{rll}
      \hmodules(\interact{p}{\role p_1,\ldots,\role p_n})\ & = \ \{\role p, \role p_1,\dots,\role p_n\}\\
      \hmodules(\ifTE {E}{p}{C_1}{C_2}) \ & =\ \{\role p\}\\
      \hmodules(X) \ & =\ \hmodules (C)\qquad\qquad(\text{if } X\stackrel{\mathsf{def}}{=} C\in\defin)\\
      \hmodules(\CEnd) \ & =\ \emptyset
    \end{array}
  \end{displaymath}
\end{definition}
Then, the property of strongly connected is defined below.
%
\begin{definition}[Strongly Connected Choreography]
  A choreography $C$ is \emph{strongly connected}, written
  $\sconnected(C)$, if it satisfies the following conditions:
  \begin{displaymath}
    \begin{array}{c}
      \infer {
      \sconnected(\interact{p}{\role p_1,\ldots,\role p_n})
      } {
      \forall j\in J.\ \sconnected(C_j)\ \land\ \hmodules(C_j)\neq\emptyset\Rightarrow \hmodules(C_j)\cap\{\role p, \role p_1,\dots,\role p_n\}\neq\emptyset
      }      
      
      \\\\

      \infer {
      \sconnected(\ifTE {E}{p}{C_1}{C_2})
      } {
      \forall j\in \{1,2\}.\ \sconnected(C_j)\ \land\ \hmodules(C_j)\neq\emptyset\Rightarrow \role p\in\hmodules(C_j)
      }      

      \\\\

      \infer {
      \sconnected(X)
      } {
      \sconnected(C)\qquad\qquad X\stackrel{\mathsf{def}}{=} C\in\defin
      }

      \qquad\qquad

      \infer {
      \sconnected(\CEnd)
      } {
      }

    \end{array}
  \end{displaymath}
\end{definition}
% 
The notion of connectedness is quite well-known in the
literature. Since our framework is based on synchronous communication,
we follow the same approach as that of Carbone et
al.~\cite{CHY12}. The basic idea is that each interaction shares at
least one module with the subsequent choreography. In particular, in
every branch of a probabilistic choice or an if-then-else involving
modules $\role p_1, \ldots, \role p_n$, the first action (if any) of
every other module $\role q$ must be an interaction with one of
$\role p_1, \ldots, \role p_n$, possibly after unfolding recursive
calls.
%
For example, the choreography
\begin{displaymath}\small
  X\stackrel{\mathsf{def}}{=} 
  \interactBase{p}{q}:\,
  \left(
    \begin{array}{l}
      \lambda_1: u_1;\ \interactBase{q}{r}:\, \lambda_1':u_1';\ X    \\
      \lambda_2: u_2;\ X
    \end{array}
  \right)
\end{displaymath}
%
is strongly connected while 
% 
$%  X\stackrel{\mathsf{def}}{=} 
\interactBase{p}{q}:\,
\left(
  \begin{array}{l}
    \lambda_1: u_1;\ \interactBase{r_1}{r_2}:\, \lambda_1':u_1';\ \CEnd.
  \end{array}
\right) $ is not.
% 

%%%%  THIS LEMMA HERE MEANS NOTHING AS IT IS STATED
% Then, we can state the following Lemma which identifies some
% uniqueness properties of projections. 
% \begin{lemma}
%   The state of a projected choreography is uniquely identified by the
%   counter $\iota$.
% \end{lemma}

\smallskip

We are now ready for our main theorem. In the sequel, $S_+$ is
obtained from state $S$ by extending its domain with the extra
variables $s_{\role q}$ (one for each module in a choreography) used
by the projection. 
%%% THIS ASSUMPTION BELOW IS NOT GOOD, REMOVED
% Also, in the theorem statement, we assume that such variables are
% initialised to $0$.
In the projection, we utilize
alphabetised parallel composition $\pp$, wherein modules synchronise
solely on labels that appear in both modules.
%
\begin{theorem}[Projection]\label{thm:epp}
  Given a choreography $C$ such that $\sconnected(C)$, we have that
  $(S,C) \red{\lambda} (S', C')$ if and only if\quad
  $\pp_{\role q\in C}\ \proj (\role q, C,\iota)\vdash S_{+}
  \red{\lambda} S_{+}'$.
\end{theorem}
\begin{proof}
  The proof proceeds by cases on the syntax of $C$.
  \begin{itemize}
  \item $C=\interact{p}{\role p_1,\ldots,\role p_n}$.  By (the only
    applicable) rule \textsf{(Interact)}, we have that
    % 
    \[
      (S, \interact{p}{\role p_1,\ldots,\role p_n})
      \red{\lambda_j} (S[u_j], C_j) 
    \]
    % 
    By definition of projection, we obtain the following PRISM
    commands. Role $\role p$ is projected as:
    \[
      \Big\{\commandBase {a_j} {\ s_{\role p}\!=\! \iota} {\lambda_j:\ s_{\role p}\!=\!
        s_{\role p} +1+\sum_{k=1}^{j-1}\mathsf{nodes}({C_k})}\ \&\ \projE
      {u_j}{\role p}\Big\}_{j\in J}
      \ \cup\ \bigcup_{j} \proj (\role q, C_j,
      \iota+1+\sum_{k=1}^{j-1}\mathsf{nodes}({C_k}))
    \]
    Roles $\role{p_i}$ are projected as:
    \[
      \Big\{\commandBase {a_j} {\ s_{\role p_i}\!=\! \iota} {1:\ s_{\role p_i}\!=\!
        s_{\role p_i} +1+\sum_{k=1}^{j-1}\mathsf{nodes}({C_k})}\ \&\ 
       \projE{u_j}{\role p_i}\Big\}_{j\in J}
      \ \cup\ \bigcup_{j} \proj (\role p_i, C_j, \iota+1+\sum_{k=1}^{j-1}\mathsf{nodes}({C_k}))
    \]
    %
    while any other role is projected as: 
    \[\bigcup_{j} \proj (\role q, C_j, \iota+\sum_{k=1}^{j-1}\mathsf{nodes}({C_k}))
    \]
    We need to show two things: first, that the projection above can
    make the same transition; second, that if the projection makes a
    transition, it must be corresponding to that of the choreography
    above.
    %
    Observe that the state of a generated CTMC is uniquely identified
    by the counter $\iota$.  The uniqueness of the label $a$
    (Assumption~1) makes sure that all and only those modules involved
    in this interaction synchronise with this action (this shows from
    the rules in Figure~\ref{fig:semantics}).  As a consequence of
    this and since choreographies are strongly connected, the commands
    generated by this step of the translation are such that any state
    $S_+$ is exclusively going to enable these commands (because of
    the guard $s_{\role q}=\iota$) which obviously implies that it
    must be done with rate $\lambda_j$ applying the rules in
    Figure~\ref{fig:semantics}. This argument is key for proving both
    directions of the if and only if.

  \item $C=\ifTE {E}{p}{C_1}{C_2}$. In this case, the projection of
      $\role p$ is
      \begin{equation*}
        \begin{array}{lll}
          \left\{ 
          \begin{array}{lll}
            \commandBase {} {s_{\role p}\!=\! \iota\ \&\ E}{\ 1: s'_{\role p}\!=\! \iota+1},\\ 
            \commandBase {} {s_{\role p}\!=\! \iota\ \&\ \mathsf{not}(E)}
            {\ 1: s'_{\role p}\!=\! \iota+\mathsf{nodes}(C_1)+1}
          \end{array}
          \right\}
          \qquad\cup\quad 
          \\\\
          \qquad\proj (\role{p}, C_1, \iota+1)
          \quad\cup\quad
          \proj (\role{p}, C_2, \iota+\mathsf{nodes}(C_1)+1)
        \end{array}
      \end{equation*}
      while, for all other roles, we have
      \begin{equation*}
          \qquad \proj (\role{q}, C_1, \iota)
          \quad\cup\quad
          \proj (\role{q}, C_2, \iota+\mathsf{nodes}(C_1))
      \end{equation*}
      In this case, role $\role p$ is enabled by the counter. All
      other roles are not enabled simply because we assume that the
      choreography is strongly connected; hence, any other
      synchronisation or if-then-else statement is blocked, as it must
      involve $\role p$ (on a different counter value).

    \item $C=X$, and $C=\CEnd$. Similar to the other case.
  \end{itemize}


  The case for DTMC is also similar.



  %       \newpage We prove each direction separately.
  %       \begin{itemize}
  %       \item (only if). Assume that
  %         % 
  %   $$(S, \interact{p}{\role p_1,\ldots,\role
  %     p_n})\red{\lambda_j}(S[\sigma(E_j)/x_j], C_j)$$
  %   % 
  %   and let us consider the projection of the term
  %   %
  %   $$\interact{p}{\role p_1,\ldots,\role p_n}$$
  %   % 
  %   Given some fresh $l_1,\ldots, l_n$, we generate the following
  %   commands for each $q$ in
  %   $\{\role p, \role p_1,\ldots,\role p_n\}$:
  %   % 
  %   $$
  %   \Big\{\commandBase {l_j} {\ s_{q}\!=\! s} {\kappa_j:\ s_{q}\!=\!
  %     s_{q} +1+\sum_{k=1}^{j-1}\mathsf{nodes}({C_k})}\ \&\ \projE
  %   {E_j}q\Big\}_{j\in J}
  %   $$
  %   %
  %   where $\kappa_j=\lambda_j$ if $q\neq\role p$ and $\kappa_j=1$
  %   otherwise. 
  %   % 
  %   Because the labels $l_j$ are fresh and the state counter is unique
  %   to this interaction, these are the only commands that can
  %   synchronise together: this can be shown from the rules defining
  %   the semantics of PRISM. Additionally, since all rates are set to 1
  %   besides the commands generated for role $\role p$, the transition
  %   will also be labelled with $\lambda_j$.


  % \item (if). In the opposite direction, we have that 
  %   % 
  %   $$\proj (*, \interact{p}{\role p_1,\ldots,\role p_n})\vdash
  %   S\uplus S_{+}\red{\lambda} S'\uplus S_{+}'$$
  %   %
  %   Again, given some fresh $l_1,\ldots, l_n$, the projection that
  %   reduces must be such that for each $q$ in
  %   $\{\role p, \role p_1,\ldots,\role p_n\}$:
  %   % 
  %   $$
  %   \Big\{\commandBase {l_j} {\ s_{q}\!=\! s} {\kappa_j:\ s_{q}\!=\!
  %     s_{q} +1+\sum_{k=1}^{j-1}\mathsf{nodes}({C_k})}\ \&\ \projE
  %   {E_j}q\Big\}_{j\in J}
  %   $$
  %   %
  %   where $\kappa_j=\lambda_j$ if $q\neq\role p$ and $\kappa_j=1$
  %   otherwise. 
  %   % 
  %   Again, the freshness of the labels $l_j$ together with the working
  %   states $s_qq$

  %   Because the labels $l_j$ are fresh and the state counter is unique
  %   to this interaction, these are the only commands that can
  %   synchronise together: this can be shown from the rules defining
  %   the semantics of PRISM. Additionally, since all rates are set to 1
  %   besides the commands generated for role $\role p$, the transition
  %   will also be labelled with $\lambda_j$.

  % \end{itemize}
\end{proof}

%%% Local Variables: 
%%% mode: latex
%%% TeX-master: "main"
%%% End:
