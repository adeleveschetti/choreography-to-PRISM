\documentclass[11pt]{article}

% Page layout
\usepackage{geometry}
\geometry{margin=1in}

% Formatting
\usepackage{xcolor}
\usepackage{setspace}
\usepackage{enumitem}

% Boxes for reviewer comments
\usepackage{tcolorbox}
\tcbuselibrary{skins,breakable}

% Very light grey background
\definecolor{reviewgray}{gray}{0.95}

\newtcolorbox{reviewercomment}{
  breakable,
  colback=reviewgray,
  colframe=black!20,
  boxrule=0.3pt,
  arc=2pt,
  left=6pt,
  right=6pt,
  top=6pt,
  bottom=6pt
}

% Commands
\newcommand{\revcomment}[1]{%
  \begin{reviewercomment}
  \textbf{Reviewer comment.} #1
  \end{reviewercomment}
}

\newcommand{\revresponse}[1]{%
  \par\noindent\textbf{Response.} #1
}

\newcommand{\revchanges}[1]{%
  \par\noindent\textbf{Changes in the manuscript.} #1
}

\begin{document}

\title{Response to Reviewers}
\author{<Paper title>}
\date{}
\maketitle

\section*{General Response}

We thank the Editor and the Reviewers for their careful reading of the manuscript and
for their detailed and constructive feedback. We have revised the paper accordingly
and believe that the changes substantially improve both clarity and technical depth.
Below we address each comment point by point.

\newpage

% ============================================================
\section*{Reviewer 1}

\subsection*{Summary and General Evaluation}

\revcomment{
This paper introduces a probabilistic version of the Choreography programming approach.
[...] An additional example highlights the limitations of the approach for modeling dining cryptographers.
}

\revresponse{
Thank you for the positive assessment of the contribution and for highlighting both
its novelty and its relevance. We address the technical concerns in detail below.
}

\subsection*{Major Technical Concerns}

\revcomment{
I found the technical result establishing operational equivalence between choreographies
and their projections rather weak. [...] I would also expect a result establishing
operational correspondence for a sequence of reductions.
}

\revresponse{
<Your response here. Typically: clarify scope of correspondence, explain limitations,
justify design choice, and outline strengthening of theorem or discussion added.>
}

\revchanges{
<Describe changes to Theorem 1, proof structure, added discussion, or limitations section.>
}

\revcomment{
I have also some concerns about the way in which the choreography state is projected on
different participants.
}

\revresponse{
<Your response here.>
}

\revchanges{
<Where the projection of state is clarified or formalised.>
}

% ------------------------------------------------------------
\subsection*{Detailed Comments (Reviewer 1)}

\subsubsection*{Page 2}

\revcomment{
When giving examples, the use of prime variables is not clear at this point.
}

\revresponse{
<Your response here.>
}

\revchanges{
<Clarification added to example or notation section.>
}

\subsubsection*{Page 3}

\revcomment{
“this is the first probabilistic choreography language that is not a type abstraction”.
What does this mean?
}

\revresponse{
<Your response here.>
}

\subsubsection*{Page 4}

\revcomment{
I did not understand the explanation of foreach. What is the role of k? What is op?
A formal encoding would be welcome.
}

\revresponse{
<Your response here.>
}

\subsubsection*{Page 5}

\revcomment{
I am puzzled about the notion of Non-deterministic Synchronization. [...]
}

\revresponse{
<Your response here, possibly structured around the three bullet questions.>
}

\revchanges{
<Formal definition added / well-formedness conditions clarified.>
}

\subsubsection*{Page 10}

\revcomment{
Example 3: The modules in this example share variables (y and x). [...]
}

\revresponse{
<Your response here.>
}

\subsubsection*{Page 11}

\revcomment{
The choice of annotating choreographies with labels seems somewhat ad hoc.
}

\revresponse{
<Your response here.>
}

\subsubsection*{Page 12}

\revcomment{
Projection of conditionals: I do not understand why the conditional is not projected.
}

\revresponse{
<Your response here.>
}

\revcomment{
Projection of choreography variable X: This relies on defs(X). [...]
}

\revresponse{
<Your response here.>
}

\revchanges{
<Formal definition of defs added.>
}

\subsubsection*{Page 13}

\revcomment{
Projection of updates: Are you assuming that updates at the choreography level are
well-formed in some way?
}

\revresponse{
<Your response here.>
}

\subsubsection*{Page 14}

\revcomment{
Definition 4: Is it not the case that the choreography is annotated?
}

\revresponse{
<Your response here.>
}

\subsubsection*{Page 15}

\revcomment{
The correspondence theorem (Th. 1) appears somewhat weak [...]
}

\revresponse{
<Your response here.>
}

\subsubsection*{Page 25}

\revcomment{
Although I can follow the reasoning, I wonder what the fundamental limitation is here.
}

\revresponse{
<Your response here.>
}

\revchanges{
Added explanation on page 24.
}

\newpage

% ============================================================
\section*{Reviewer 2}

\subsection*{General Comments}

\revcomment{
The authors present an extended version of the COORDINATION 2024 paper.
}

\revresponse{
We thank the reviewer for their careful comparison with the conference version.
}

\revcomment{
The main changes as stated by the authors at the end of the Introduction are:
[...] The paper does not help the reader in identifying them.
}

\revresponse{
<Your response here.>
}

\revchanges{
<Explicit pointers to sections added in the Introduction.>
}

\revcomment{
The experimental analysis in Section 7 needs to be improved.
}

\revresponse{
<Your response here.>
}

\revcomment{
Section 6 is irrelevant and can be removed.
}

\revresponse{
<Your response here (removed, shortened, or justified).>
}

% ------------------------------------------------------------
\subsection*{Detailed Comments (Reviewer 2)}

\revcomment{
It is confusing that you use lambda1,2 as label in page 2 [...]
}

\revresponse{
<Your response here.>
}

\revcomment{
Are you sure Def. 1 works as intended? What happens if x appears in u?
}

\revresponse{
<Your response here.>
}

\revchanges{
<Definition corrected and semantics clarified.>
}

\revcomment{
Is it relevant that your implementation consists of 1246 lines of Java?
}

\revresponse{
<Your response here.>
}

% (Continue similarly for remaining detailed comments as needed.)

\newpage

% ============================================================
\section*{Reviewer 3}

\subsection*{General Assessment}

\revcomment{
The paper is well written, well organised, and the results are sound [...]
}

\revresponse{
We thank the reviewer for the positive assessment and for the detailed suggestions.
}

\subsection*{General Comments}

\revcomment{
The pros and cons of choreographic programming as opposed to process-oriented
programming are not elaborated upon very much.
}

\revresponse{
<Your response here.>
}

\revchanges{
Added discussion on pages 2 and 25--26.
}

\revcomment{
How does this relate to the notions of implementability and realisability [...]?
}

\revresponse{
<Your response here.>
}

\revchanges{
Related work extended on page 25.
}

\revcomment{
Could the authors imagine a more thorough comparison, e.g., bisimilarity?
}

\revresponse{
<Your response here.>
}

\revchanges{
Added discussion on page 17.
}

\subsection*{Detailed Comments}

\revcomment{
Various spelling inconsistencies (UK vs US spelling).
}

\revresponse{
All spelling inconsistencies have been corrected to UK English throughout the paper.
}

\revchanges{
Systematic spelling revision applied across the manuscript.
}

\revcomment{
Pages 27--28: inconsistent formatting of authors' first names in references.
}

\revresponse{
<Your response here.>
}

\end{document}
