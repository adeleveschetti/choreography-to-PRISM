\section{A denotational semantics for PRISM} 

We proceed by steps. First, we define $\sem{-}$, as the closure of the
following rules:
% 
\begin{displaymath}\small
  \begin{array}{ccc}
    \infer[\mathsf{(Module)}]
    {F_i\in\sem{\role{p}:\{F_i\}_i}}
    {}
    \qquad
    \infer[\mathsf{(Par_1)}]
    {\cmd{}E{\lambda_i}{x_i}{E_i}{i\in I}\in \sem{M_1\ppp A M_2}}
    {\cmd{}E{\lambda_i}{x_i}{E_i}{i\in I}\in \sem{M_j}
    & j=1 \lor j=2}
    \\\\
    \infer[\mathsf{(Par_2)}]
    {\cmd aE{\lambda_i}{x_i}{E_i}{i\in I}\in \sem{M_1\ppp A M_2}}
    {\cmd aE{\lambda_i}{x_i}{E_i}{i\in I}\in \sem{M_j}
    & a\not\in A    & j=1 \lor j=2}
    \\\\
    \infer[\mathsf{(Par_3)}]
    {
    [] E\land E' \rightarrow \{\lambda_i * \lambda_j': x_i = E_i \land y_j = E_j' \}_{i\in I, j\in J}
    \in \sem{M_1\ppp A M_2}
    }
    {\cmd aE{\lambda_i}{x_i}{E_i}{i\in I}\in \sem{M_1}{}
    & \cmd a{E'}{\lambda_j'}{y_j}{E_j'}{j\in J}\in \sem{M_2}{}
    & a\in A}
    \\\\
    \infer[\mathsf{(Hide_1)}]
    {
    \cmd {}E{\lambda_j}{x_i}{E_i}{i\in I}\in \sem{M/A}{}
    }
    {
    \cmd {}E{\lambda_j}{x_i}{E_i}{i\in I}\in \sem{M}{}
    }
    \qquad
    \infer[\mathsf{(Hide_2)}]
    {
    \cmd {a}E{\lambda_j}{x_i}{E_i}{i\in I}\in \sem{M/A}{}
    }
    {
    \cmd {a}E{\lambda_j}{x_i}{E_i}{i\in I}\in \sem{M}{}
    & a\not\in A
      }
    \\\\
    \infer[\mathsf{(Hide_3)}]
    {
    \cmd {}E{\lambda_j}{x_i}{E_i}{i\in I}\in \sem{M/A}{}
    }
    {
    \cmd {a}E{\lambda_j}{x_i}{E_i}{i\in I}\in \sem{M}{}
    & a\in A
      }
      \qquad
      \infer[\mathsf{(Subst_1)}]
      {
      \cmd {}E{\lambda_j}{x_i}{E_i}{i\in I}\in \sem{\sigma M}{}
      }
      {
      \cmd {}E{\lambda_j}{x_i}{E_i}{i\in I}\in \sem{M}{}
      }
    \\\\
    \infer[\mathsf{(Subst_2)}]
    {
    \cmd {a}E{\lambda_j}{x_i}{E_i}{i\in I}\in \sem{\sigma M}{}
    }
    {
    \cmd {a}E{\lambda_j}{x_i}{E_i}{i\in I}\in \sem{M}{}
    & a\not\in\mathsf{dom}(\sigma)
      }
    \\\\
    \infer[\mathsf{(Subst_3)}]
    {
    \cmd {\sigma a}E{\lambda_j}{x_i}{E_i}{i\in I}\in \sem{\sigma M}{}
    }
    {
    \cmd {a}E{\lambda_j}{x_i}{E_i}{i\in I}\in \sem{M}{}
    & a\in\mathsf{dom}(\sigma)
      }
  \end{array}
\end{displaymath}
The rules above work with modules, parallel composition, name hiding,
and substitution. The idea is that given a network, we wish to collect
all those commands $F$ that are contained in the network,
independently from which module they are being executed
in. Intuitively, we can regard $\sem{N}$ as a set, where starting from
all commands present in the syntax, we do some filtering and renaming,
based on the structure of the network.

Now, given $\sem N$, we define a transition system that shows how the
system evolves.  Let $\state$ be a function that given a variable in
$\mathsf{Var}$ returns a value in $\mathsf{Val}$. Then, given an
initial state $\state_0$, we can define a transition system where each
of node is a (different) $\state$ function. Then, we can move from
$\state_1$ to $\state_2$. 



%%% Local Variables: 
%%% mode: latex
%%% TeX-master: "main"
%%% End: