Understanding and programming distributed systems 
%
present significant challenges due to their inherent complexity, and
the possibility of obscure edge cases arising from their complex
interactions.
%
Unlike monolithic systems, distributed programs involve multiple nodes
operating concurrently and communicating over networks, introducing a
multitude of potential failure scenarios and nondeterministic
behaviours.
%
One of the primary challenges in understanding distributed systems
lies in the fact that the interactions between multiple components can
diverge from the sum of their individual behaviours. This emergent
behaviour often results from subtle interactions between nodes, making
it difficult to predict and reason about the system's overall
behaviour.

PRISM \cite{PRISMdoc} is a probabilistic model checker that offers a
specialised language for the specification and verification of
probabilistic concurrent systems. PRISM has been used in various
fields, including multimedia protocols \cite{multimedia}, randomised
distributed algorithms \cite{distr1,distr2}, security protocols
\cite{security1,security2}, and biological systems \cite{bio1,bio2}.
At its core, PRISM provides a declarative language with a set of
constructs for describing probabilistic behaviours and properties
within a system.
%
Given a distributed system, we can use PRISM to model the behaviour of
each of its nodes,
% For example, in a distributed voting system where
% multiple nodes collaborate to reach a consensus, we could use PRISM to
% model the behaviour of each voting node individually, % specifying
% states, transitions, and communications between nodes,
and then verify desired properties for the entire system. However,
this approach can become difficult to manage as the number of nodes
increases.

%
%Furthermore, the inherent asynchrony of distributed systems
%complicates the task of programming them as intended. Conventional
%programming paradigms, which rely on sequential execution and shared
%state, struggle to capture the asynchronous nature of distributed
%systems accurately. As a result, developers may inadvertently overlook
%edge cases or race conditions, leading to unintended behaviors and
%system failures. [NON É MEGLIO EVITARE QUESTO? QUA ABBIAMO TUTTO
%SINCRONO]

Choreographic programming~\cite{M23} is an emerging programming
paradigm in which programs, referred to as choreographies, serve as
specifications providing a global perspective on the communication
patterns inherent in a distributed system. 
%
In particular, instead of relying on a central orchestrator or
controller to dictate the behavior of individual components,
choreographic languages focus on defining communication patterns and
protocols that govern the interactions between entities.
%
In essence, choreographies abstract away the internal details of
individual components and emphasise the global behaviour as a
composition of decentralised interactions. % Within the context of
% concurrent and distributed systems, choreographic languages serve as
% tools for defining interaction protocols that govern the communication
% between processes. 
%
%
Although, this approach can be used for a proper subset of distributed
systems, choreographies have proven to be very useful in many use
cases since they facilitate the automatic generation of decentralised
implementations that are inherently correct-by-construction.
%
% \marco{Choreographies particularly excel in environments characterised
%   by deterministic communication patterns and clear sequencing of
%   actions. However, they may not be well-suited for managing highly
%   unpredictable or dynamically changing interactions. In scenarios
%   where processes exhibit significant variability or intricate
%   coordination logic, choreographies may lack the flexibility or
%   adaptability required for effective orchestration.}

This paper presents a choreographic language designed for modelling
concurrent probabilistic systems.
%\textcolor{red}{Our language enhances the modeling of concurrent probabilistic systems, providing clarity often absent in 
%traditional languages. However, it may lack expressivity in certain scenarios. For instance, in cases  where intricate behaviors rely 
%on features such as conditional statements or probabilistic branching, our language's limitations become 
%apparent, limiting accurate representation of these complex dynamics.}
% with clarity and precision. 
Additionally, we introduce a compiler capable of translating protocols
described in this language into PRISM code. This choreographic
approach not only simplifies the modelling process but also ensures
integration with PRISM's powerful analysis capabilities. 

To illustrate the idea with concrete commands, consider the following example. We have a choreographic language specification: 

\[
\begin{array}{lll}
C \stackrel{\mathsf{def}}{=} \interactBase{p}{q}
\left\{
\begin{array}{lll}
\lambda_1: \ (x'=1)\ \&\ (y'=2);\ C\\[1mm]
\lambda_2: \ (x'=3)\ \&\ (y'=1);\ C
\end{array}
\right.
\end{array}
\]

This specification recursively defines an interaction where process \(\role{p}\) communicates with process \(\role{q}\), choosing either branch \(\lambda_1\) (setting \(x\) to 1 and \(y\) to 2) or branch \(\lambda_2\) (setting \(x\) to 3 and \(y\) to 1). 
The projection mechanism translates this global choreography into local PRISM modules by annotating each interaction with a unique label (e.g., \(a\) for \(\lambda_1\) and \(b\) for \(\lambda_2\)) and using a state counter (e.g., \(s_{\role p}\) and \(s_{\role q}\)) to uniquely identify each step. 
The corresponding PRISM translation is:

\[
\begin{array}{llll}
\role{p}: \{ & \quad \commandBase {a}{s_{\role p}=0} {\mu_1:(x'=1)\ \&\ (s_{\role p}'=1)} \quad\qquad \commandBase {}{s_{\role p}=1} {1:(s_{\role p}'=0)}\\[2mm]
              & \quad \commandBase {b}{s_{\role p}=0} {\mu_2:(x'=3)\ \&\ (s_{\role p}'=2)} \quad\qquad \commandBase {}{s_{\role p}=2} {1:(s_{\role p}'=0)} \quad \}\\[2mm]
\role{q}: \{ & \quad \commandBase {a}{s_{\role q}=0} {\gamma_1:(y'=2)\ \&\ (s_{\role q}'=1)} \qquad\quad \commandBase {}{s_{\role q}=1} {1:(s_{\role q}'=0)}\\[2mm]
              & \quad \commandBase {b}{s_{\role q}=0} {\gamma_2:(y'=1)\ \&\ (s_{\role q}'=2)} \qquad\quad \commandBase {}{s_{\role q}=2} {1:(s_{\role q}'=0)} \quad \}\\
\end{array}
\]

In this translation, the command labeled \(a\) in \(\role{p}\)’s module corresponds to branch \(\lambda_1\) (with rate \(\mu_1\)), and the matching command in \(\role{q}\)’s module (with rate \(\gamma_1\)) sets \(y\) to 2; similarly, label \(b\) corresponds to branch \(\lambda_2\) (with rates \(\mu_2\) for \(\role{p}\) and \(\gamma_2\) for \(\role{q}\)), setting \(x\) and \(y\) appropriately. 
After executing an interaction command in each module, a subsequent reset command (with rate 1) sets the state counters \( s_{\role p} \) and \( s_{\role q} \) to 0 due to the recursive call following the branching. This mechanism ensures that these commands are only executable when the system has reached this particular state.


Through our contributions, we aim to provide a smooth workflow for
modeling, analyzing, and verifying concurrent probabilistic systems,
ultimately increasing their usability in various application domains.
%
In particular, by employing choreographies, we gain a clear and
comprehensive view of the interactions occurring within the system,
allowing us to discern the flow of processes and detect any potential
sources of error in the modelling phase.
% This transparency ensures that each step is understood and executed
% accurately, minimizing the likelihood of errors and enhancing the
% reliability of the system.


\mypar{Contributions and Overview.} We summarise our contributions as
follows:
\begin{itemize} 
\item we propose a choreographic language equipped with well-defined
  syntax and semantics, tailored specifically for describing
  concurrent systems with probabilistic behaviours
  (\S~\ref{sec:chor}). To the best of our knowledge, this is the first
  probabilistic choreography language that is not a type abstraction;

\item we introduce a semantics for the minimal fragment of PRISM
  needed for code generation from choreographies (\S~\ref{sec:prism}),
  which follows PRISM's original semantics~\cite{PRISMdoc};
  % enhancing its usability and facilitating clearer understanding for
  % users.

\item we establish a rigorous definition for a translation function
  from choreographies to PRISM (\S~\ref{sec:proj}), and address its
  correctness. This translation serves as a crucial intermediary step
  in transforming models described in our choreographic language into
  PRISM-compatible representations;

\item we give a compiler implementation that executes the defined
  translation function (\S~\ref{sec:impl}),
  % This compiler transforms choreographies into PRISM models,
  enabling users to utilise PRISM's robust analysis features while
  benefiting from the expressiveness of choreographies;

\item we show some use cases in order to demonstrate the applicability
  of our language (\S~\ref{sec:bench}).
\end{itemize}

\mypar{Changes with respect to the conference version.}
%
This article is an expanded version of our paper presented at
COORDINATION 2024~\cite{CV24}. It includes detailed definitions and
full proofs, which were previously omitted. Additionally, we provide a
thorough analysis of richer language constructs incorporated in the
implementation. Furthermore, we introduce a new set of use cases,
analysed through choreographies, including an example that illustrates
when the PRISM endpoint setting can be just as effective as a
choreography-based approach.


%%% Local Variables: 
%%% mode: latex
%%% TeX-master: "main"
%%% End:
