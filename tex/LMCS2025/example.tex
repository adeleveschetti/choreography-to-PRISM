As an example, we consider a simplified version of the
\text{thinkteam} example, a three-tier data management system
\cite{DBLP:journals/entcs/BeekMLGFS05}, presented in the PRISM
documentation\footnote{\url{https://www.prismmodelchecker.org/casestudies/thinkteam.php}}.
{The protocol aims to manage exclusive access to a single file,
  controlled by the \codeprism{CheckOut} process. Users can request
  access, which can be granted based on the file's current status. The
  goal is to ensure that only one user possesses the file at any given
  time while allowing for efficient access requests and retries in
  case of denial.}  Users move between different states (0, 1, or 2)
based on the granted or denied access to the file with corresponding
rates $(\lambda, \mu, \theta)$; the \codeprism{CheckOut} process
transitions between two states (0 or 1):
%
\begin{lstlisting}[style=prism-color,% caption={A PRISM example},captionpos=b,
  frame=none,label={example1},escapechar=|]
	ctmc 
	module User|\label{user-init}|
		User_STATE : [0..2] init 0;
	
		[alpha_1] (User_STATE=0) $\rightarrow$ lambda : (User_STATE'=1);|\label{first-line}|
		[alpha_2] (User_STATE=0) $\rightarrow$ lambda : (User_STATE'=2);
		[beta] (User_STATE=1) $\rightarrow$ mu : (User_STATE'=0);
		[gamma_1] (User_STATE=2) $\rightarrow$ theta : (User_STATE'=1);
		[gamma_2] (User_STATE=2) $\rightarrow$ theta : (User_STATE'=2);
	endmodule|\label{user-end}|
	
	module CheckOut|\label{check-init}|
		CheckOut_STATE : [0..1] init 0;
	
		[alpha_1,alpha_2] (CheckOut_STATE=0) $\rightarrow$ 1 : (CheckOut_STATE'=1);
		[beta] (CheckOut_STATE=1) $\rightarrow$ 1 : (CheckOut_STATE'=0);
		[gamma_1,gamma_2] (CheckOut_STATE=1) $\rightarrow$ 1 : (CheckOut_STATE'=1);
	endmodule|\label{check-end}|
\end{lstlisting}
%
In PRISM, modules are individual processes whose behaviour is
specified by a collection of commands, in a declarative fashion.
Processes have a local state, can interact with other modules and
query each other's state. Above, the modules \codeprism{User} (lines
\ref{user-init}-\ref{user-end}) and \codeprism{CheckOut} (lines
\ref{check-init}-\ref{check-end}) can synchronise on different labels, e.g., 
\codeprism{alpha_1}. %, \codeprism{alpha_2}, \codeprism{beta},
% \codeprism{gamma_1} and \codeprism{gamma_2}.
On line \ref{first-line}, \codeprism{(User_STATE=0)} is a condition
indicating that this transition is enabled when \codeprism{User_STATE}
has value 0. The variable \codeprism{lambda} is a rate, since the
program models a Continuous Time Markov Chain (CTMC). The command
\codeprism{(User_STATE'=1)} is an update, indicating that
\codeprism{User_STATE} changes to 1 when this transition fires.

Understanding the interactions between processes in this example might
indeed be challenging, especially without additional context or
explanation.  Alternatively, when formalised using our choreographic
language, the same model becomes significantly clearer.
% , as shown in Listing \ref{example2}.
\begin{lstlisting}[style=chor-color,% caption={Example of Listing \ref{example1} in our choreographic language},captionpos=b,
  frame=none, label={example2}]
  C0 := CheckOut $\rightarrow$ User : (+["1*lambda"] ; C1	 +["1*lambda"]  ;  C2)
  C1 := CheckOut $\rightarrow$ User : (+["1*theta"] ; C0)  
  C2 := CheckOut $\rightarrow$ User : (+["1*mu"] ; C1   +["1*mu"] ;  C2)
\end{lstlisting}
In this model, we define three distinct choreographies, namely
\texttt{C0}, \texttt{C1}, and \texttt{C2}. These choreographies
describe the interaction patterns between the modules \texttt{CheckOut}
and \texttt{User}. The state updates resulting from these
interactions are not explicitly depicted as they are not relevant for
this particular protocol, but necessary in the PRISM implementation.
% Similarly, this consideration extends to the labels used within the
% model.  Just as the state updates are automatically generated,
% unique labels used to denote various transitions in the model are
% also automatically assigned.
%
% \marco{ADD THIS COMMENT HERE: Second, is the choreography not just the
%   product of both modules (seen as finite state machines)? Would that
%   be the general idea, i.e. to model the state space as one
%   finite-state machine directly?}
%
As evident from this example, the choreographic language facilitates a
straightforward understanding of the interactions between processes,
minimizing the likelihood of errors. In fact, we can think of the
choreography representation of this example as the product of the two
PRISM modules seen above.


To formally analyze the system, we need to project the choreography onto individual components, resulting in a PRISM model that captures the behavior of each process explicitly. The projection process ensures that each role in the choreography, such as \texttt{User} and \texttt{CheckOut}, is assigned a state variable that tracks its execution progress. Interactions in the choreography correspond to PRISM synchronization labels, ensuring consistency across transitions. Furthermore, the rates in the choreography, such as $\lambda, \mu, \theta$, translate into transition rates in the PRISM model.

Each choreographic rule maps to a corresponding PRISM command. For instance, the choreographic transition:

\begin{lstlisting}[style=chor-color]
C0 := CheckOut $\rightarrow$ User : (+["1*lambda"] ; C1)
\end{lstlisting}

projects to PRISM as \codeprism{[alpha_1] (User_STATE=0) -> lambda : (User_STATE'=1);}
 for the module \texttt{User} and as \codeprism{[alpha_1] (CheckOut_STATE=0) -> 1 : (CheckOut_STATE'=1);} for the \texttt{CheckOut}.
This ensures both modules execute in sync under the label \codeprism{alpha_1}. Similar transformations apply to other interactions.

By systematically projecting the high-level choreography into PRISM, we obtain an equivalent, structured, and analyzable model while retaining the clarity of the original specification. 

%%% Local Variables: 
%%% mode: latex
%%% TeX-master: "main"
%%% End: