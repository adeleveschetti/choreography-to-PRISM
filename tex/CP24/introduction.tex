Understanding and programming distributed systems pose formidable
challenges due to their inherent complexity and the potential for
elusive edge cases
% to lurk
within their intricate interactions. Unlike monolithic systems,
distributed programs involve multiple nodes operating concurrently and
communicating over networks, introducing a multitude of potential
failure scenarios and nondeterministic behaviours.
%
One of the primary challenges in understanding distributed systems
lies in the fact that the interactions between multiple components can
diverge from the sum of their individual behaviours. This emergent
behaviour often results from subtle interactions between nodes, making
it difficult to predict and reason about the system's overall
behaviour.

PRISM \cite{PRISMdoc} is a probabilistic model checker that offers a
specialised language for the specification and verification of
probabilistic concurrent systems. PRISM has been used in various
fields, including multimedia protocols \cite{multimedia}, randomised
distributed algorithms \cite{distr1,distr2}, security protocols
\cite{security1,security2}, and biological systems \cite{bio1,bio2}.
At its core, PRISM provides a declarative language with a set of
constructs for describing probabilistic behaviours and properties
within a system.
%
Given a distributed system, we can use PRISM to model the behaviour of
each of its nodes,
% For example, in a distributed voting system where
% multiple nodes collaborate to reach a consensus, we could use PRISM to
% model the behaviour of each voting node individually, % specifying
% states, transitions, and communications between nodes,
and then verify desired properties for the entire system. However,
this approach can become difficult to manage as the number of nodes
increases.

%
%Furthermore, the inherent asynchrony of distributed systems
%complicates the task of programming them as intended. Conventional
%programming paradigms, which rely on sequential execution and shared
%state, struggle to capture the asynchronous nature of distributed
%systems accurately. As a result, developers may inadvertently overlook
%edge cases or race conditions, leading to unintended behaviors and
%system failures. [NON É MEGLIO EVITARE QUESTO? QUA ABBIAMO TUTTO
%SINCRONO]

Choreographic programming~\cite{M23} is an emerging programming
paradigm in which programs, referred to as choreographies, serve as
specifications providing a global perspective on the communication
patterns inherent in a distributed system. 
%
In particular, instead of relying on a central orchestrator or
controller to dictate the behavior of individual components,
choreographic languages focus on defining communication patterns and
protocols that govern the interactions between entities.
%
In essence, choreographies abstract away the internal details of
individual components and emphasise the global behaviour as a
composition of decentralised interactions. % Within the context of
% concurrent and distributed systems, choreographic languages serve as
% tools for defining interaction protocols that govern the communication
% between processes. 
%
This approach facilitates the automatic generation of decentralised
implementations that are inherently correct-by-construction.

This paper presents a choreographic language designed for modelling
concurrent probabilistic systems.
% with clarity and precision. 
Additionally, we introduce a compiler capable of translating protocols
described in this language into PRISM code. This choreographic
approach not only simplifies the modelling process but also ensures
integration with PRISM's powerful analysis capabilities. 

As an example, consider the following simplified version of a system
presented in the PRISM
documentation\footnote{\url{https://www.prismmodelchecker.org/casestudies/thinkteam.php}},
where a user moves between different states (0, 1, or 2) based on
certain events $(\alpha, \beta, \gamma)$ with corresponding rates
$(\lambda, \mu, \theta)$, and where a checkout process transitions
between two states (0 or 1).
%
\begin{lstlisting}[style=prism-color,% caption={A PRISM example},captionpos=b,
  frame=none,label={example1},escapechar=|]
	ctmc 
	module User|\label{user-init}|
		User_STATE : [0..2] init 0;
	
		[alpha_1] (User_STATE=0) $\rightarrow$ lambda : (User_STATE'=1);|\label{first-line}|
		[alpha_2] (User_STATE=0) $\rightarrow$ lambda : (User_STATE'=2);
		[beta] (User_STATE=1) $\rightarrow$ mu : (User_STATE'=0);
		[gamma_1] (User_STATE=2) $\rightarrow$ theta : (User_STATE'=1);
		[gamma_2] (User_STATE=2) $\rightarrow$ theta : (User_STATE'=2);
	endmodule|\label{user-end}|
	
	module CheckOut|\label{check-init}|
		CheckOut_STATE : [0..1] init 0;
	
		[alpha_1,alpha_2] (CheckOut_STATE=0) $\rightarrow$ 1 : (CheckOut_STATE'=1);
		[beta] (CheckOut_STATE=1) $\rightarrow$ 1 : (CheckOut_STATE'=0);
		[gamma_1,gamma_2] (CheckOut_STATE=1) $\rightarrow$ 1 : (CheckOut_STATE'=1);
	endmodule|\label{check-end}|
\end{lstlisting}
%
In PRISM, modules are individual processes whose behaviour is
specified by a collection of commands, in a declarative fashion.
Processes have a local state, can interact with other modules and
query each other's state. Above, the modules \codeprism{User} (lines
\ref{user-init}-\ref{user-end}) and \codeprism{CheckOut} (lines
\ref{check-init}-\ref{check-end}) synchronise on labels
\codeprism{alpha_1}, \codeprism{alpha_2}, \codeprism{beta},
\codeprism{gamma_1} and \codeprism{gamma_2}. On line \ref{first-line},
\codeprism{(User_STATE=0)} is a condition indicating that this
transition is enabled when \codeprism{User_STATE} has value 0. The
variable \codeprism{lambda} represents a rate, since the program
models a Continuous Time Markov Chain (CTMC). The command
\codeprism{(User_STATE'=1)} is an update, indicating that
\codeprism{User_STATE} changes to 1 when this transition fires.

Understanding the interactions between processes in this example might
indeed be challenging, especially without additional context or
explanation.  Alternatively, when formalised using our choreographic
language, the same model becomes significantly clearer.
% , as shown in Listing \ref{example2}.
\begin{lstlisting}[style=chor-color,% caption={Example of Listing \ref{example1} in our choreographic language},captionpos=b,
  frame=none, label={example2}]
  C0 := User $\rightarrow$ Check : (+["lambda*1"] ; C1	 +["lambda*1"]  ;  C2)
  C1 := User $\rightarrow$ Check : (+["beta*1"] ; C0)  
  C2 := User $\rightarrow$ Check : (+["mu*1"] ; C1   +["mu*1"] ;  C2)
\end{lstlisting}
In this model, we define three distinct choreographies, namely
\texttt{C0}, \texttt{C1}, and \texttt{C2}. These choreographies
describe the interaction patterns between the modules \texttt{User}
and \texttt{Check}. The state updates resulting from these
interactions are not explicitly depicted as they are not relevant for
this particular protocol, but necessary in the PRISM implementation.
% Similarly, this consideration extends to the labels used within the
% model.  Just as the state updates are automatically generated,
% unique labels used to denote various transitions in the model are
% also automatically assigned.
%
As evident from this example, the choreographic language facilitates a
straightforward understanding of the interactions between processes,
minimizing the likelihood of errors.



%%% Local Variables: 
%%% mode: latex
%%% TeX-master: "main"
%%% End:


Through our contributions, we aim to provide a smooth workflow for
modeling, analyzing, and verifying concurrent probabilistic systems,
ultimately increasing their usability in various application domains.
%
In particular, by employing choreographies, we gain a clear and
comprehensive view of the interactions occurring within the system,
allowing us to discern the flow of processes and detect any potential
sources of error in the modelling phase.
% This transparency ensures that each step is understood and executed
% accurately, minimizing the likelihood of errors and enhancing the
% reliability of the system.


\mypar{Contributions and Overview.} Our contributions can be
categorised as:
\begin{itemize} 
\item Firstly, we propose a choreographic language equipped with
  well-defined syntax and semantics, tailored specifically for
  describing concurrent systems with probabilistic behaviours
  (\S~\ref{sec:chor}). To the best of our knowledge, this is the first
  probabilistic choreography language that is not a type abstraction.
\item Then, we introduce a semantics for PRISM (\S~\ref{sec:prism}),
  based on the original semantics for PRISM~\cite{PRISMdoc}.
  % enhancing its usability and facilitating clearer understanding for
  % users.
\item Furthermore, we establish a rigorous definition for a
  translation function from choreographies to PRISM
  (\S~\ref{sec:proj}), and address its correctness. This translation
  (projection) serves as a crucial intermediary step in transforming
  (compiling) models described in our choreographic language into
  PRISM-compatible representations.
\item Lastly, we implement a compiler that translates choreographies
  into PRISM, allowing users to use PRISM's robust analysis features
  while benefiting from the expressiveness of our choreographic
  language.
\end{itemize}



%%% Local Variables: 
%%% mode: latex
%%% TeX-master: "main"
%%% End:
