



% That means that once we have a set of executable rules, we can start
% building a transition system. In order to do so, we

% No idea what this stuff below is. 
% \begin{displaymath}
%   \begin{array}{lll}
%     W(M) = \{F \mid F\in\sem{M}\}
%     \\\\
%     X = \{x_1,\ldots,x_n\}
%     \\\\
%     \sigma : X \rightarrow V
%   \end{array}
% \end{displaymath}




\newpage


\subsection{Choreographies}

\mypar{Syntax.} Our choreographic language is defined by the following
syntax:
%
\begin{displaymath}
  \begin{array}{lll}
    \text{(Chor)}
    \quad & C \ ::=\
    &
      \interact{p}{\role p_1,\ldots,\role p_n}
      \ \mid\
      \ifTE {E}{p}{C_1}{C_2}
      \ \mid\
      X
      \ \mid\
      \CEnd
  \end{array}
\end{displaymath}
We comment the various constructs. The syntactic category $C$ denotes
choreographic programmes. The term
$\role p\rightarrow \{\role p_1,\ldots,\role p_n\}\,\Sigma\{\lambda_j:
x_j=E_j;\ C_j\}_{j\in J}$ denotes an interaction initiated by role
$\role p$ with roles $\role p_i$. Unlike in PRISM, a choreography
specifies what interaction must be executed next, shifting the focus
from what can happen to what must happen. When the synchronisation
happens then, in a probabilistic way, one of the branches is selected
as a continuation. The term
$\textsf{if } E@\role p \textsf{ then } C_1 \textsf{
  else } C_2$ factors in some local choices for some particular
roles. [write a bit more about procedure calls, recursion and the zero
process]

\mypar{Semantics.} Similarly to how we did for the PRISM language, we
consider the state space $\mathsf{Val}^n$ where $n$ is the number of
variables present in the choreography. We then inductively define the
transition function for the state space as follows: 
\begin{displaymath}
  \begin{array}{lllll}
    (\sigma, \interact{p}{\role p_1,\ldots,\role p_n}) 
    \red{\lambda_j}
    (\sigma[\sigma(E_j)/x_j], C_j) 
    \\\\
    (\sigma,\ifTE {E}{p}{C_1}{C_2}) 
    \red{}
    (\sigma, C_1)
    \\\\
    X\stackrel{\mathsf{def}}{=} C \quad\Rightarrow\quad (\sigma, X) \red{}(\sigma,C)
  \end{array}
\end{displaymath}
From the transition relation above, we can immediately define an LTS
on the state space. Given an initial state $\sigma_0$ and a
choreography $C$, the LTS is given by all the states reachable from
the pair $(\sigma_0,C)$. I.e., for all derivations
$(\sigma_0,C)\red{\lambda_0}\ldots\red{\lambda_n}(\sigma_n, C_n)$ and
$i<n$, we have that $(\sigma_i,\sigma_{i+1})\in\delta$ [adjust once
the definition of probabilistic LTS is in].




\subsection{Projection from Choreographies to PRISM}
\mypar{Mapping Choreographies to PRISM.} We need to run some standard
static checks because, since there is branching, some terms may not be
projectable.
%
\begin{displaymath}
  \begin{array}{lllll}
    \big(q\in\{\role p, \role p_1,\ldots,\role p_n\}, J=\{1,2\},\ l_1, l_2 \text{ fresh}  
\big)\\
    \proj (q,\interact{p}{\role p_1,\ldots,\role p_n}, s)= \\
    \qquad\{ 
    \commandBase {l_1} {s_{\role p_1} = s} {\lambda_1: s_{\role p_1} = s_{\role p_1} +1},\  
    \commandBase {l_2} {s_{\role p_1} = s} {\lambda_2: s_{\role p_1} = s_{\role p_1} +2}  
    \}
    \quad\cup\\
    \qquad \proj (\role{p}_1, C_1, s+1)
    \quad\cup\quad
    \proj (\role{p}_1, C_2, s+\mathsf{nodes}(C_1))
    \\\\
    \big(q\notin\{\role p, \role p_1,\ldots,\role p_n\}\big)\\
    \proj (q,\interact{p}{\role p_1,\ldots,\role p_n}, s)\ = \ \proj (\role{p}_1, C_1, s)
    \ \cup\
    \proj (\role{p}_1, C_2, s+\mathsf{nodes}(C_1))
    \\\\
    \big(q = \role p\big)\\
    \proj (q,\ifTE {E}{p}{C_1}{C_2}, s)= \\
    \qquad\{ 
    \command {} {s_{\role p_1} = s \& E} \lambda: s_{\role p_1} = s_{\role p_1} +1,  
    \command {} {s_{\role p_1} = s \& \mathsf{not}(E)} \lambda: s_{\role p_1} = s_{\role p_1} +1  
    \}
    \quad\cup\\
    \qquad \proj (\role{p}_1, C_1, s+1)
    \quad\cup\quad
    \proj (\role{p}_1, C_2, s+\mathsf{nodes}(C_1))
  \end{array}
\end{displaymath}


       


% proj (p2, p -> {p1} [ lambda1: (x  = 4). C1   +_l   lambda2: (x  = 4123). C2 ], s) = 
%      proj (p2, C1, s+1)
%      \union
%      proj (p2, C2, s+1)




% proj (p, if E@p then C1 else C2, s) = 
%      {
%       [] s_p = s & E s_p = s_p+1, 
%       [] s_p = s & E s_p = s_p+length(C1)
%       }   

% \begin{displaymath}
%   \begin{array}{ll}
%     f\Big(\; \role p_1\longrightarrow\{\role
%     p_i\}_{i\in I} +\{\lambda_j : x_j=E_j: C_j\}_{j\in J}, \texttt{network}, \overline{s}
%     \Big)
%     \\\\
%     =
%     \\\\
%     \ell \text{ fresh};\\
%     %\textsf{label} = \texttt{newlabel}();\\
%     \textsf{for } \role p_k \in \mathbf{roles} \{\\
%     \quad\textsf{for } j \in J \{\\
%     \quad\quad \texttt{network} = \texttt{add}(\role p_k,[\ell]\; {s_{\role p_k} = s_k}
%     \rightarrow\ \Sigma_j\lambda_j : x_j=E_j\;\& \;s_{\role p_k}'=s_k+1);\\
%   	\quad\}\\
%   	\}\\
% 	\textsf{for } j \in J \{\\
% 	\quad \texttt{network} = f(C_j, \texttt{network}, \overline{s}');  \\ 
% 	\}\\
%   	\textsf{return } \texttt{network}
%   \end{array}
% \end{displaymath}




\begin{comment}
\begin{displaymath}
  \begin{array}{ll}
    f\Big(\quad \role p_1\longrightarrow\{\role p_2\} \oplus
    \left\{
    \begin{array}{l}{}
      [\lambda_1] x=5:  \role p_1\longrightarrow\{\role p_2\} \oplus\{[\lambda_3] y=5\}
      \\{}
      [\lambda_2] y=10: \role p_1\longrightarrow\{\role p_2\} \oplus\{[\lambda_4] x=10\}
    \end{array}
    \right\}
    , \ \role p_1:\emptyset\ \pp\ \role p_2:\emptyset
    \Big)
    \\\\
    =
    \\\\
    \text{label} = \text{newlabel}();\\
    \text{for } \role p_i \{\\
    \ add(p_i, [\text{label}] {s_{p_i} = state(p_i)} \rightarrow\
    \left\{
    \begin{array}{ll}
      \lambda_1: x'=5; \textsf{state}(\role p_i)' = \textsf{generatenewstate} (\role p_i)
      \\
      \lambda_2: y'=10; \textsf{state}(\role p_i)' = \textsf{generatenewstate} (\role p_i)
    \end{array}
    \right\}
    \\

    \ f(\role p_1\longrightarrow\{\role p_2\} \oplus\{[\lambda_3] y=5\}, network') = network''\\
    \ return f(\role p_1\longrightarrow\{\role p_2\} \oplus\{[\lambda_4] x=10\}, network'')
    
  \end{array}
\end{displaymath}

\hrule
\end{comment}

% \begin{displaymath}
% 	f : C\times\texttt{network}\times States \longrightarrow \texttt{network} \quad\quad \texttt{network} : \mathcal R \longrightarrow \text{Set}(F)
% \end{displaymath}

% \begin{displaymath}
%   \begin{array}{ll}
%     f\Big(\; \role p_1\longrightarrow\{\role
%  p_i\}_{i\in I} +\{\lambda_j : x_j=E_j: C_j\}_{j\in J}, \texttt{network}, \overline{s}
%     \Big)
%     \\\\
%     =
%     \\\\
%     \ell \text{ fresh};\\
%     %\textsf{label} = \texttt{newlabel}();\\
%     \textsf{for } \role p_k \in \mathbf{roles} \{\\
%     \quad\textsf{for } j \in J \{\\
%     \quad\quad \texttt{network} = \texttt{add}(\role p_k,[\ell]\; {s_{\role p_k} = s_k}
%     \rightarrow\ \Sigma_j\lambda_j : x_j=E_j\;\& \;s_{\role p_k}'=s_k+1);\\
%   	\quad\}\\
%   	\}\\
% 	\textsf{for } j \in J \{\\
% 	\quad \texttt{network} = f(C_j, \texttt{network}, \overline{s}');  \\ 
% 	\}\\
%   	\textsf{return } \texttt{network}
%   \end{array}
% \end{displaymath}

% where $\overline{s} = (s_1, \ldots, s_n)$ and $\overline{s}' = (s_1+1, \ldots, s_n+1)$ for $n = |\mathbf{roles}|$.
% \begin{displaymath}
%   \begin{array}{ll}
%     f\Big(\; \textsf{if } E @ \role p \textsf{ then } C_1 \textsf{ else } C_2, \texttt{network}, \overline{s} 
%     \Big)
%     \\\\
%     =
%     \\\\
%     \textsf{for } \role p_k \in \mathbf{roles} \{\\
%     \quad\texttt{network} = \texttt{add}(\role p_k,[\;]\; s_{\role p_k} = s_k \;\& \;f(E))+f(C_1, \texttt{network},\overline{s});\\
%     \quad\texttt{network} = \texttt{add}(\role p_k,[\;]\; s_{\role p_k} = s_k \;\& \;!f(E))+f(C_1, \texttt{network},\overline{s});\\
%   \}\\
%   \text{return } \texttt {network}\\

%   \end{array}
% \end{displaymath}



%%% Local Variables: 
%%% mode: latex
%%% TeX-master: "main"
%%% End:
