In this section, we discuss the choices made in the development of the choreographic language for modeling and analyzing concurrent probabilistic systems as presented in this paper. The framework covers  several key components aimed at improving usability, correctness, and efficiency in modeling and analyzing such systems.

The decision to introduce a choreographic language for modeling processes for PRISM stems from the recognition of the need to improve the usability of this probabilistic model checker. PRISM, while powerful, can be complex and daunting for users, particularly those without a strong background in formal methods. 
The development of a choreographic language with tailored syntax and semantics represents a significant methodological choice aimed at improving the intuitiveness of modeling concurrent probabilistic systems. Traditional modeling languages may lack the expressive clarity necessary to capture the intricacies of such systems effectively. By designing a language specifically geared towards choreographing system behaviors, we provide practitioners with a more intuitive means of specifying system dynamics. This choice facilitates a more natural and straightforward approach to modeling, which is essential for accurately representing real-world systems and ensuring the efficacy of subsequent analysis.

The choreographic language and projection function aim to abstract away low-level details and provide a higher-level representation of system behaviors. While this abstraction facilitates intuitive modeling, it also introduces the risk of oversimplification. Certain aspects in the behavior of concurrent probabilistic systems may be lost during the abstraction process, potentially leading to inaccuracies in the analysis results. Moreover, as discussed in Section \ref{sec:problems}, some case studies presented in the PRISM repository cannot be modeled using our choreographic language. Thus, future research would focus on enriching the expressiveness of the choreographic language to capture a broader range of system behaviors and modeling requirements.