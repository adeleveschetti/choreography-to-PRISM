PRISM \cite{PRISMdoc} is a probabilistic model checker that offers a
specialized language for the specification and verification of
probabilistic systems. Originally developed for modeling and analyzing
systems in the domain of probabilistic model checking [I DONT GET
THIS], PRISM has evolved into a versatile platform with applications
spanning various fields, including multimedia protocols, randomised
distributed algorithms, security protocols, biological systems.  At
its core, PRISM provides a rich set of constructs for describing
probabilistic behaviors, transitions, and properties within a system.
For example, let us consider a distributed voting system where
multiple nodes collaborate to reach a consensus. In PRISM, we would
describe the behavior of each voting node individually, specifying
states, transitions, and communication protocols between
nodes. However, this approach can become difficult to manage as the
number of nodes increases.

Understanding and programming distributed systems pose formidable
challenges due to their inherent complexity and the potential for
elusive edge cases to lurk within their intricate interactions. Unlike
their centralized counterparts, distributed programs involve multiple
nodes operating concurrently and communicating over networks,
introducing a multitude of potential failure scenarios and
nondeterministic behaviours.
%
One of the primary challenges in understanding distributed systems
lies in the fact that the interactions between multiple components can
diverge from the sum of their individual behaviors. This emergent
behavior often results from subtle interactions between nodes, making
it difficult to predict and reason about the system's overall
behaviour.
%
Furthermore, the inherent asynchrony of distributed systems
complicates the task of programming them as intended. Conventional
programming paradigms, which rely on sequential execution and shared
state, struggle to capture the asynchronous nature of distributed
systems accurately. As a result, developers may inadvertently overlook
edge cases or race conditions, leading to unintended behaviors and
system failures. [NON É MEGLIO EVITARE QUESTO? QUA ABBIAMO TUTTO
SINCRONO]

Choreographic programming offers a paradigm for designing coordination
plans for distributed systems from a global perspective. This approach
enables the automatic generation of decentralized implementations that
are inherently correct-by-construction.
%
In particular, instead of relying on a central orchestrator or
controller to dictate the behavior of individual components,
choreographic languages focus on defining communication patterns and
protocols that govern the interactions between entities.
%
In essence, choreography abstracts away the internal details of
individual components and emphasizes the global behavior of the system
as a composition of decentralized interactions. Within the context of
concurrent and distributed systems, choreographic languages serve as
tools for defining interaction protocols that govern the communication
between processes.


In this paper, we present a choreographic language designed for
modeling concurrent probabilistic systems with clarity and
precision. Additionally, we introduce a compiler capable of
translating protocols described in this language into PRISM code. This
choreographic approach not only simplifies the modeling process but
also ensures integration with PRISM's powerful analysis
capabilities. Through our contributions, we aim to provide a smooth
workflow for modeling, analyzing, and verifying concurrent
probabilistic systems, ultimately increasing their usability in
various application domains.

\mypar{Contributions and Overview.} Our contributions can be
categorised as follows:
\begin{itemize} 
\item Firstly, we propose a novel choreographic language equipped with
  well-defined syntax and semantics, tailored specifically for
  describing concurrent systems with probabilistic behaviors.
\item Then, we introduce a simplified semantics for PRISM, enhancing
  its usability and facilitating clearer understanding for users.
\item Furthermore, we establish a rigorous definition for a projection
  function, and discuss its correctness. This projection function
  serves as a crucial intermediary step in transforming (compiling)
  models described in our choreographic language into PRISM-compatible
  representations.
\item Lastly, we create a compiler implementation that translates the
  choreographic language models into PRISM, allowing users to use
  PRISM's robust analysis features while benefiting from the
  expressiveness of our choreographic language.
\end{itemize}



%%% Local Variables: 
%%% mode: latex
%%% TeX-master: "main"
%%% End:
