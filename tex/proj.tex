\subsection{Projection from Choreographies to PRISM}
\mypar{Mapping Choreographies to PRISM.} We need to run some standard
static checks because, since there is branching, some terms may not be
projectable.
%
\begin{displaymath}
  \begin{array}{lllll}
    \big(q\in\{\role p, \role p_1,\ldots,\role p_n\}, J=\{1,\ldots,n\},\ l_1,\ldots, l_n \text{ fresh}  
    \big)\\
    \proj (q,\interact{p}{\role p_1,\ldots,\role p_n}, s)= \\
    \qquad
    % \{ 
    \Big\{\commandBase {l_j} {\ s_{q}\!=\! s} {\kappa_j:\ s_{q}\!=\!
    s_{q} +1+\sum_{k=1}^{j-1}\mathsf{nodes}({C_k})}\ \&\ \projE
    {E_j}q\Big\}_{j\in J}
    % \commandBase {l_1} {s_{q} = s} {\lambda_1: s_{q} = s_{q} +1},\  
    % \commandBase {l_2} {s_{q} = s} {\lambda_2: s_{q} = s_{q} +\mathsf{nodes}(C_1)}  
    % \}
    \quad\cup\\
    \qquad\bigcup_{j} \proj (q, C_j, s+1+\sum_{k=1}^{j-1}\mathsf{nodes}({C_k}))
    \\\\
    \big(q\notin\{\role p, \role p_1,\ldots,\role p_n\}\big)\\
    \proj (q,\interact{p}{\role p_1,\ldots,\role p_n}, s)\ = \ \proj (\role{p}_1, C_1, s)
    \ \cup\
    \proj (\role{p}_1, C_2, s+\mathsf{nodes}(C_1))
    \\\\
    \big(q = \role p\big)\\
    \proj (q,\ifTE {E}{p}{C_1}{C_2}, s)= \\
    \qquad\{ 
    \command {} {s_{\role p_1} = s \& E} \lambda: s_{\role p_1} = s_{\role p_1} +1,  
    \command {} {s_{\role p_1} = s \& \mathsf{not}(E)} \lambda: s_{\role p_1} = s_{\role p_1} +1  
    \}
    \quad\cup\\
    \qquad \proj (\role{p}_1, C_1, s+1)
    \quad\cup\quad
    \proj (\role{p}_1, C_2, s+\mathsf{nodes}(C_1))
  \end{array}
\end{displaymath}


       


% proj (p2, p -> {p1} [ lambda1: (x  = 4). C1   +_l   lambda2: (x  = 4123). C2 ], s) = 
%      proj (p2, C1, s+1)
%      \union
%      proj (p2, C2, s+1)




% proj (p, if E@p then C1 else C2, s) = 
%      {
%       [] s_p = s & E s_p = s_p+1, 
%       [] s_p = s & E s_p = s_p+length(C1)
%       }   

% \begin{displaymath}
%   \begin{array}{ll}
%     f\Big(\; \role p_1\longrightarrow\{\role
%     p_i\}_{i\in I} +\{\lambda_j : x_j=E_j: C_j\}_{j\in J}, \texttt{network}, \overline{s}
%     \Big)
%     \\\\
%     =
%     \\\\
%     \ell \text{ fresh};\\
%     %\textsf{label} = \texttt{newlabel}();\\
%     \textsf{for } \role p_k \in \mathbf{roles} \{\\
%     \quad\textsf{for } j \in J \{\\
%     \quad\quad \texttt{network} = \texttt{add}(\role p_k,[\ell]\; {s_{\role p_k} = s_k}
%     \rightarrow\ \Sigma_j\lambda_j : x_j=E_j\;\& \;s_{\role p_k}'=s_k+1);\\
%   	\quad\}\\
%   	\}\\
% 	\textsf{for } j \in J \{\\
% 	\quad \texttt{network} = f(C_j, \texttt{network}, \overline{s}');  \\ 
% 	\}\\
%   	\textsf{return } \texttt{network}
%   \end{array}
% \end{displaymath}




\begin{comment}
\begin{displaymath}
  \begin{array}{ll}
    f\Big(\quad \role p_1\longrightarrow\{\role p_2\} \oplus
    \left\{
    \begin{array}{l}{}
      [\lambda_1] x=5:  \role p_1\longrightarrow\{\role p_2\} \oplus\{[\lambda_3] y=5\}
      \\{}
      [\lambda_2] y=10: \role p_1\longrightarrow\{\role p_2\} \oplus\{[\lambda_4] x=10\}
    \end{array}
    \right\}
    , \ \role p_1:\emptyset\ \pp\ \role p_2:\emptyset
    \Big)
    \\\\
    =
    \\\\
    \text{label} = \text{newlabel}();\\
    \text{for } \role p_i \{\\
    \ add(p_i, [\text{label}] {s_{p_i} = state(p_i)} \rightarrow\
    \left\{
    \begin{array}{ll}
      \lambda_1: x'=5; \textsf{state}(\role p_i)' = \textsf{generatenewstate} (\role p_i)
      \\
      \lambda_2: y'=10; \textsf{state}(\role p_i)' = \textsf{generatenewstate} (\role p_i)
    \end{array}
    \right\}
    \\

    \ f(\role p_1\longrightarrow\{\role p_2\} \oplus\{[\lambda_3] y=5\}, network') = network''\\
    \ return f(\role p_1\longrightarrow\{\role p_2\} \oplus\{[\lambda_4] x=10\}, network'')
    
  \end{array}
\end{displaymath}

\hrule
\end{comment}

% \begin{displaymath}
% 	f : C\times\texttt{network}\times States \longrightarrow \texttt{network} \quad\quad \texttt{network} : \mathcal R \longrightarrow \text{Set}(F)
% \end{displaymath}

% \begin{displaymath}
%   \begin{array}{ll}
%     f\Big(\; \role p_1\longrightarrow\{\role
%  p_i\}_{i\in I} +\{\lambda_j : x_j=E_j: C_j\}_{j\in J}, \texttt{network}, \overline{s}
%     \Big)
%     \\\\
%     =
%     \\\\
%     \ell \text{ fresh};\\
%     %\textsf{label} = \texttt{newlabel}();\\
%     \textsf{for } \role p_k \in \mathbf{roles} \{\\
%     \quad\textsf{for } j \in J \{\\
%     \quad\quad \texttt{network} = \texttt{add}(\role p_k,[\ell]\; {s_{\role p_k} = s_k}
%     \rightarrow\ \Sigma_j\lambda_j : x_j=E_j\;\& \;s_{\role p_k}'=s_k+1);\\
%   	\quad\}\\
%   	\}\\
% 	\textsf{for } j \in J \{\\
% 	\quad \texttt{network} = f(C_j, \texttt{network}, \overline{s}');  \\ 
% 	\}\\
%   	\textsf{return } \texttt{network}
%   \end{array}
% \end{displaymath}

% where $\overline{s} = (s_1, \ldots, s_n)$ and $\overline{s}' = (s_1+1, \ldots, s_n+1)$ for $n = |\mathbf{roles}|$.
% \begin{displaymath}
%   \begin{array}{ll}
%     f\Big(\; \textsf{if } E @ \role p \textsf{ then } C_1 \textsf{ else } C_2, \texttt{network}, \overline{s} 
%     \Big)
%     \\\\
%     =
%     \\\\
%     \textsf{for } \role p_k \in \mathbf{roles} \{\\
%     \quad\texttt{network} = \texttt{add}(\role p_k,[\;]\; s_{\role p_k} = s_k \;\& \;f(E))+f(C_1, \texttt{network},\overline{s});\\
%     \quad\texttt{network} = \texttt{add}(\role p_k,[\;]\; s_{\role p_k} = s_k \;\& \;!f(E))+f(C_1, \texttt{network},\overline{s});\\
%   \}\\
%   \text{return } \texttt {network}\\

%   \end{array}
% \end{displaymath}




\subsection{Correctness}

In the sequel, $S_+$ is a state $S$ extended with extra variables used
by the projection.
\begin{theorem}[EPP]\label{thm:epp}
  Given a well-formed choreography $C$, we have that
  $(S,C) \red{\lambda} (S', C')$ iff
  $\proj (*, C)\vdash S\uplus S_{+}\red{\lambda} S'\uplus S_{+}'$.
\end{theorem} 
\begin{proof} We prove each direction separately.
  \begin{itemize}
  \item (only if). Assume that
    % 
    $$(S, \interact{p}{\role p_1,\ldots,\role
      p_n})\red{\lambda_j}(S[\sigma(E_j)/x_j], C_j)$$
    % 
    and let us consider the projection of the term
    %
    $$\interact{p}{\role p_1,\ldots,\role p_n}$$
    % 
    Given some fresh $l_1,\ldots, l_n$, we generate the following
    commands for each $q$ in
    $\{\role p, \role p_1,\ldots,\role p_n\}$:
    % 
    $$
    \Big\{\commandBase {l_j} {\ s_{q}\!=\! s} {\kappa_j:\ s_{q}\!=\!
      s_{q} +1+\sum_{k=1}^{j-1}\mathsf{nodes}({C_k})}\ \&\ \projE
    {E_j}q\Big\}_{j\in J}
    $$
    %
    where $\kappa_j=\lambda_j$ if $q\neq\role p$ and $\kappa_j=1$
    otherwise. 
    % 
    Because the labels $l_j$ are fresh and the state counter is unique
    to this interaction, these are the only commands that can
    synchronise together: this can be shown from the rules defining
    the semantics of PRISM. Additionally, since all rates are set to 1
    besides the commands generated for role $\role p$, the transition
    will also be labelled with $\lambda_j$.


  \item (if). In the opposite direction, we have that 
    % 
    $$\proj (*, \interact{p}{\role p_1,\ldots,\role p_n})\vdash
    S\uplus S_{+}\red{\lambda} S'\uplus S_{+}'$$
    %
    Again, given some fresh $l_1,\ldots, l_n$, the projection that
    reduces must be such that for each $q$ in
    $\{\role p, \role p_1,\ldots,\role p_n\}$:
    % 
    $$
    \Big\{\commandBase {l_j} {\ s_{q}\!=\! s} {\kappa_j:\ s_{q}\!=\!
      s_{q} +1+\sum_{k=1}^{j-1}\mathsf{nodes}({C_k})}\ \&\ \projE
    {E_j}q\Big\}_{j\in J}
    $$
    %
    where $\kappa_j=\lambda_j$ if $q\neq\role p$ and $\kappa_j=1$
    otherwise. 
    % 
    Again, the freshness of the labels $l_j$ together with the working
    states $s_qq$

    Because the labels $l_j$ are fresh and the state counter is unique
    to this interaction, these are the only commands that can
    synchronise together: this can be shown from the rules defining
    the semantics of PRISM. Additionally, since all rates are set to 1
    besides the commands generated for role $\role p$, the transition
    will also be labelled with $\lambda_j$.

  \end{itemize}
\end{proof}

%%% Local Variables: 
%%% mode: latex
%%% TeX-master: "main"
%%% End:
