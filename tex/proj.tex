Our next task is to provide a mapping that can translate
choreographies into the PRISM language. The goal of this section is to
address the theoretical aspect of this.

\mypar{Mapping Choreographies to PRISM.}  The operation of generating
endpoint code from a choreography is known as {\em
  projection}. Normally, projection is defined separately for each
module appearing in the choreography program. We first observe that in
order to simulate interactions in PRISM, we need to use labels on
which each module in the interaction can synchronise. Therefore,
without loss of generality, we make a slight abuse of notation and
assume that each step in a choreography is annotated with unique
labels that then it is going to be used by the projection. We call
such a choreography a {\em well-annotated choreography}. For example,
the choreography
$\interactBase{p}{q}\ \lambda_1: \CEnd \ +\ \lambda_2: \ifTE
  {E}{p}{\CEnd}{\CEnd}$ is going to be annotated as
  $\interactBasel{p}{a_1}{q}\ \lambda_1: \CEnd \ +\ \lambda_2: \ifTEl
    {E}{a_2}{p}{\CEnd}{\CEnd}$.


%
    \MC{We need to say that we must run some standard static checks
      because, since there is branching, some terms may not be
      projectable.}
%

    We can now formally define the projection of a choreography with
    rates into the PRISM language.
    % 
    \begin{definition}[Projection, CTMC]\label{def:projCTMC} Given a
      well-annotated choreography $C$ with
      rates, a module $\role p$, and a natural number $\iota$, we
      define the projection function $\proj$ as:
      \begin{displaymath}\small
        \begin{array}{lr}

          \proj (\role q,\interact{p}{\role p_1,\ldots,\role p_n}, \iota)= 
          &  \boxed{\text{if }\role q=\role p}\\[2mm]
          \qquad
          % \{ 
          \Big\{\commandBase {l_j} {\ s_{q}\!=\! \iota} {\lambda_j:\ s_{q}\!=\!
          s_{q} +1+\sum_{k=1}^{j-1}\mathsf{nodes}({C_k})}\ \&\ \projE
          {E_j}q\Big\}_{j\in J}
          \\
          \qquad\cup\ \bigcup_{j} \proj (\role q, C_j, s+1+\sum_{k=1}^{j-1}\mathsf{nodes}({C_k}))
          \\\\

          \proj (\role q,\interact{p}{\role p_1,\ldots,\role p_n}, \iota)= 
          &  \!\!\!\!\!\! \!\!\!\!\!\!\boxed{\text{if }\role q\in\{\role p_1,\ldots,\role p_n\}}\\[2mm]
          \qquad
          % \{ 
          \Big\{\commandBase {l_j} {\ s_{q}\!=\! \iota} {1:\ s_{q}\!=\!
          s_{q} +1+\sum_{k=1}^{j-1}\mathsf{nodes}({C_k})}\ \&\ \projE
          {E_j}q\Big\}_{j\in J}
          \\
          \qquad\cup\ \bigcup_{j} \proj (\role q, C_j, s+1+\sum_{k=1}^{j-1}\mathsf{nodes}({C_k}))
          \\\\

          \proj (\role q,\interact{p}{\role p_1,\ldots,\role p_n}, \iota)= 
          &  \!\!\!\!\!\! \!\!\!\!\!\!\boxed{\text{if }\role q\not\in\{\role p, 
            \role p_1,\ldots,\role p_n\}}\\[2mm]
          \qquad\ \proj (\role{q}, C_1, s)=
          \ \cup\
          \proj (\role{q}, C_2, s+\mathsf{nodes}(C_1))
          \\\\

          \proj (\role q,\ifTE {E}{p}{C_1}{C_2}, \iota) = 
          &  \boxed{\text{if }\role q=\role p}\\[2mm]
          \qquad\left\{ 
          \begin{array}{lll}
            \commandBase {} {s_{q}\!=\! \iota\ \&\ E}{\ 1: s'_{q}\!=\! \iota+1},\\ 
            \commandBase {} {s_{q}\!=\! \iota\ \&\ \mathsf{not}(E)}
            {\ 1: s'_{q}\!=\! \iota+\mathsf{nodes}(C_1)+1}
          \end{array}
          \right\}
          \\
          \qquad\cup\quad \proj (\role{p}, C_1, \iota+1)
          \quad\cup\quad
          \proj (\role{p}, C_2, \iota+\mathsf{nodes}(C_1)+1)
          \\\\

          \proj (\role q,\ifTE {E}{p}{C_1}{C_2}, \iota) = 
          &  \boxed{\text{if }\role q\neq\role p}\\[2mm]
          \qquad\proj (\role{q}, C_1, \iota+1)
          \quad\cup\quad
          \proj (\role{q}, C_2, \iota+\mathsf{nodes}(C_1)+1)

          \\\\

          \proj (\role q,\CEnd, \iota) = \emptyset

        \end{array}
      \end{displaymath}
    \end{definition}



    \begin{definition}[Projection, DTMC] Given a well-annotated
      choreography $C$ with probabilities, a module $\role p$, and a
      natural number $\iota$, we define the projection function
      $\proj$ as:
      \begin{displaymath}\small
        \begin{array}{lr}

          \proj (\role q,\interact{p}{\role p_1,\ldots,\role p_n}, \iota)= 
          &  \boxed{\text{if }\role q=\role p}\\[2mm]
          \qquad
          \left\{
          \begin{array}{lll}
            \commandBase {} {s_{q}\!=\! \iota}{\ \sum_{j\in J} \lambda_j: s'_{q}\!=\! \iota+1+j},\\ 
            \{\commandBase {l_j} {s_{q}\!=\! \iota+1+j}
            \ 1: s'_{q}\!=\! \iota+1+\sum_{k=1}^{j-1}\mathsf{nodes}({C_k})\ \&\ \projE
            {E_j}q\}_{j\in J}
          \end{array}
          \right\}
          \\
          \qquad\cup\ \bigcup_{j} \proj (\role q, C_j, s+1+\sum_{k=1}^{j-1}\mathsf{nodes}({C_k}))
          \\\\

        \end{array}
      \end{displaymath}
      For the other cases, the definitions are equivalent to the one presented in Definition \ref{def:projCTMC}.
\end{definition}


       


% proj (p2, p -> {p1} [ lambda1: (x  = 4). C1   +_l   lambda2: (x  = 4123). C2 ], s) = 
%      proj (p2, C1, s+1)
%      \union
%      proj (p2, C2, s+1)




% proj (p, if E@p then C1 else C2, s) = 
%      {
%       [] s_p = s & E s_p = s_p+1, 
%       [] s_p = s & E s_p = s_p+length(C1)
%       }   

% \begin{displaymath}
%   \begin{array}{ll}
%     f\Big(\; \role p_1\longrightarrow\{\role
%     p_i\}_{i\in I} +\{\lambda_j : x_j=E_j: C_j\}_{j\in J}, \texttt{network}, \overline{s}
%     \Big)
%     \\\\
%     =
%     \\\\
%     \ell \text{ fresh};\\
%     %\textsf{label} = \texttt{newlabel}();\\
%     \textsf{for } \role p_k \in \mathbf{roles} \{\\
%     \quad\textsf{for } j \in J \{\\
%     \quad\quad \texttt{network} = \texttt{add}(\role p_k,[\ell]\; {s_{\role p_k} = s_k}
%     \rightarrow\ \Sigma_j\lambda_j : x_j=E_j\;\& \;s_{\role p_k}'=s_k+1);\\
%   	\quad\}\\
%   	\}\\
% 	\textsf{for } j \in J \{\\
% 	\quad \texttt{network} = f(C_j, \texttt{network}, \overline{s}');  \\ 
% 	\}\\
%   	\textsf{return } \texttt{network}
%   \end{array}
% \end{displaymath}




\begin{comment}
\begin{displaymath}
  \begin{array}{ll}
    f\Big(\quad \role p_1\longrightarrow\{\role p_2\} \oplus
    \left\{
    \begin{array}{l}{}
      [\lambda_1] x=5:  \role p_1\longrightarrow\{\role p_2\} \oplus\{[\lambda_3] y=5\}
      \\{}
      [\lambda_2] y=10: \role p_1\longrightarrow\{\role p_2\} \oplus\{[\lambda_4] x=10\}
    \end{array}
    \right\}
    , \ \role p_1:\emptyset\ \pp\ \role p_2:\emptyset
    \Big)
    \\\\
    =
    \\\\
    \text{label} = \text{newlabel}();\\
    \text{for } \role p_i \{\\
    \ add(p_i, [\text{label}] {s_{p_i} = state(p_i)} \rightarrow\
    \left\{
    \begin{array}{ll}
      \lambda_1: x'=5; \textsf{state}(\role p_i)' = \textsf{generatenewstate} (\role p_i)
      \\
      \lambda_2: y'=10; \textsf{state}(\role p_i)' = \textsf{generatenewstate} (\role p_i)
    \end{array}
    \right\}
    \\

    \ f(\role p_1\longrightarrow\{\role p_2\} \oplus\{[\lambda_3] y=5\}, network') = network''\\
    \ return f(\role p_1\longrightarrow\{\role p_2\} \oplus\{[\lambda_4] x=10\}, network'')
    
  \end{array}
\end{displaymath}

\hrule
\end{comment}

% \begin{displaymath}
% 	f : C\times\texttt{network}\times States \longrightarrow \texttt{network} \quad\quad \texttt{network} : \mathcal R \longrightarrow \text{Set}(F)
% \end{displaymath}

% \begin{displaymath}
%   \begin{array}{ll}
%     f\Big(\; \role p_1\longrightarrow\{\role
%  p_i\}_{i\in I} +\{\lambda_j : x_j=E_j: C_j\}_{j\in J}, \texttt{network}, \overline{s}
%     \Big)
%     \\\\
%     =
%     \\\\
%     \ell \text{ fresh};\\
%     %\textsf{label} = \texttt{newlabel}();\\
%     \textsf{for } \role p_k \in \mathbf{roles} \{\\
%     \quad\textsf{for } j \in J \{\\
%     \quad\quad \texttt{network} = \texttt{add}(\role p_k,[\ell]\; {s_{\role p_k} = s_k}
%     \rightarrow\ \Sigma_j\lambda_j : x_j=E_j\;\& \;s_{\role p_k}'=s_k+1);\\
%   	\quad\}\\
%   	\}\\
% 	\textsf{for } j \in J \{\\
% 	\quad \texttt{network} = f(C_j, \texttt{network}, \overline{s}');  \\ 
% 	\}\\
%   	\textsf{return } \texttt{network}
%   \end{array}
% \end{displaymath}

% where $\overline{s} = (s_1, \ldots, s_n)$ and $\overline{s}' = (s_1+1, \ldots, s_n+1)$ for $n = |\mathbf{roles}|$.
% \begin{displaymath}
%   \begin{array}{ll}
%     f\Big(\; \textsf{if } E @ \role p \textsf{ then } C_1 \textsf{ else } C_2, \texttt{network}, \overline{s} 
%     \Big)
%     \\\\
%     =
%     \\\\
%     \textsf{for } \role p_k \in \mathbf{roles} \{\\
%     \quad\texttt{network} = \texttt{add}(\role p_k,[\;]\; s_{\role p_k} = s_k \;\& \;f(E))+f(C_1, \texttt{network},\overline{s});\\
%     \quad\texttt{network} = \texttt{add}(\role p_k,[\;]\; s_{\role p_k} = s_k \;\& \;!f(E))+f(C_1, \texttt{network},\overline{s});\\
%   \}\\
%   \text{return } \texttt {network}\\

%   \end{array}
% \end{displaymath}




\subsection{Correctness}

In the sequel, $S_+$ is a state $S$ extended with extra variables used
by the projection.
\begin{theorem}[EPP]\label{thm:epp}
  Given a well-formed choreography $C$, we have that
  $(S,C) \red{\lambda} (S', C')$ iff
  $\proj (*, C)\vdash S\uplus S_{+}\red{\lambda} S'\uplus S_{+}'$.
\end{theorem} 
\begin{proof} We prove each direction separately.
  \begin{itemize}
  \item (only if). Assume that
    % 
    $$(S, \interact{p}{\role p_1,\ldots,\role
      p_n})\red{\lambda_j}(S[\sigma(E_j)/x_j], C_j)$$
    % 
    and let us consider the projection of the term
    %
    $$\interact{p}{\role p_1,\ldots,\role p_n}$$
    % 
    Given some fresh $l_1,\ldots, l_n$, we generate the following
    commands for each $q$ in
    $\{\role p, \role p_1,\ldots,\role p_n\}$:
    % 
    $$
    \Big\{\commandBase {l_j} {\ s_{q}\!=\! s} {\kappa_j:\ s_{q}\!=\!
      s_{q} +1+\sum_{k=1}^{j-1}\mathsf{nodes}({C_k})}\ \&\ \projE
    {E_j}q\Big\}_{j\in J}
    $$
    %
    where $\kappa_j=\lambda_j$ if $q\neq\role p$ and $\kappa_j=1$
    otherwise. 
    % 
    Because the labels $l_j$ are fresh and the state counter is unique
    to this interaction, these are the only commands that can
    synchronise together: this can be shown from the rules defining
    the semantics of PRISM. Additionally, since all rates are set to 1
    besides the commands generated for role $\role p$, the transition
    will also be labelled with $\lambda_j$.


  \item (if). In the opposite direction, we have that 
    % 
    $$\proj (*, \interact{p}{\role p_1,\ldots,\role p_n})\vdash
    S\uplus S_{+}\red{\lambda} S'\uplus S_{+}'$$
    %
    Again, given some fresh $l_1,\ldots, l_n$, the projection that
    reduces must be such that for each $q$ in
    $\{\role p, \role p_1,\ldots,\role p_n\}$:
    % 
    $$
    \Big\{\commandBase {l_j} {\ s_{q}\!=\! s} {\kappa_j:\ s_{q}\!=\!
      s_{q} +1+\sum_{k=1}^{j-1}\mathsf{nodes}({C_k})}\ \&\ \projE
    {E_j}q\Big\}_{j\in J}
    $$
    %
    where $\kappa_j=\lambda_j$ if $q\neq\role p$ and $\kappa_j=1$
    otherwise. 
    % 
    Again, the freshness of the labels $l_j$ together with the working
    states $s_qq$

    Because the labels $l_j$ are fresh and the state counter is unique
    to this interaction, these are the only commands that can
    synchronise together: this can be shown from the rules defining
    the semantics of PRISM. Additionally, since all rates are set to 1
    besides the commands generated for role $\role p$, the transition
    will also be labelled with $\lambda_j$.

  \end{itemize}
\end{proof}

%%% Local Variables: 
%%% mode: latex
%%% TeX-master: "main"
%%% End:
