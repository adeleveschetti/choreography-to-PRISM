As an example, consider the following simplified version of a system
presented in the PRISM
documentation\footnote{\url{https://www.prismmodelchecker.org/casestudies/thinkteam.php}},
where a user moves between different states (0, 1, or 2) based on
certain events $(\alpha, \beta, \gamma)$ with corresponding rates
$(\lambda, \mu, \theta)$, and where a checkout process transitions
between two states (0 or 1).
%
\begin{lstlisting}[style=prism-color,% caption={A PRISM example},captionpos=b,
  frame=none,label={example1},escapechar=|]
	ctmc 
	module User|\label{user-init}|
		User_STATE : [0..2] init 0;
	
		[alpha_1] (User_STATE=0) $\rightarrow$ lambda : (User_STATE'=1);|\label{first-line}|
		[alpha_2] (User_STATE=0) $\rightarrow$ lambda : (User_STATE'=2);
		[beta] (User_STATE=1) $\rightarrow$ mu : (User_STATE'=0);
		[gamma_1] (User_STATE=2) $\rightarrow$ theta : (User_STATE'=1);
		[gamma_2] (User_STATE=2) $\rightarrow$ theta : (User_STATE'=2);
	endmodule|\label{user-end}|
	
	module CheckOut|\label{check-init}|
		CheckOut_STATE : [0..1] init 0;
	
		[alpha_1,alpha_2] (CheckOut_STATE=0) $\rightarrow$ 1 : (CheckOut_STATE'=1);
		[beta] (CheckOut_STATE=1) $\rightarrow$ 1 : (CheckOut_STATE'=0);
		[gamma_1,gamma_2] (CheckOut_STATE=1) $\rightarrow$ 1 : (CheckOut_STATE'=1);
	endmodule|\label{check-end}|
\end{lstlisting}
%
In PRISM, modules are individual processes whose behaviour is
specified by a collection of commands, in a declarative fashion.
Processes have a a local state and can interact with other modules as
well as querying each other's state. Above, the modules
\codeprism{User} (lines \ref{user-init}-\ref{user-end}) and
\codeprism{CheckOut} (lines \ref{check-init}-\ref{check-end})
synchronise on different actions with labels \codeprism{alpha_1},
\codeprism{alpha_2}, \codeprism{beta}, \codeprism{gamma_1} and
\codeprism{gamma_2}. On line \ref{first-line},
\codeprism{(User_STATE=0)} is the condition, indicating that this
transition is enabled when \codeprism{User_STATE} has value 0. The
variable \codeprism{lambda} represents a rate, since the program
models a Continuous Time Markov Chain (CTMC). Moreover, the command
\codeprism{(User_STATE'=1)} is an update, indicating that
\codeprism{User_STATE} changes to 1 when this transition fires.

Understanding the interactions between processes in this example might
indeed be challenging, especially without additional context or
explanation.  Alternatively, when formalised using our choreographic
language, the same model becomes significantly clearer.
% , as shown in Listing \ref{example2}.
\begin{lstlisting}[style=chor-color,% caption={Example of Listing \ref{example1} in our choreographic language},captionpos=b,
  frame=none, label={example2}]
  C0 := User $\rightarrow$ Check : (+["lambda*1"] . C1	 +["lambda*1"]  .  C2)
  C1 := User $\rightarrow$ Check : (+["beta*1"] . C0)  
  C2 := User $\rightarrow$ Check : (+["mu*1"] . C1   +["mu*1"] .  C2)
\end{lstlisting}
In this model, we define three distinct choreographies, namely
\texttt{C0}, \texttt{C1}, and \texttt{C2}. These choreographies
describe the interaction patterns between the modules \texttt{User}
and \texttt{Check}. The state updates resulting from these
interactions are not explicitly depicted as they are not relevant for
the protocol itself, but necessary in the PRISM implementation.
% Similarly, this consideration extends to the labels used within the
% model.  Just as the state updates are automatically generated,
% unique labels used to denote various transitions in the model are
% also automatically assigned.
%
As evident from this example, the choreographic language facilitates a
straightforward understanding of the interactions between processes,
minimizing the likelihood of errors.



%%% Local Variables: 
%%% mode: latex
%%% TeX-master: "main"
%%% End: