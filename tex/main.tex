%\documentclass[a4paper,UKenglish,cleveref, autoref, thm-restate]{lipics-v2021}
\documentclass[runningheads]{llncs}
\usepackage[T1]{fontenc}

\usepackage{wrapfig}
 \usepackage{mathtools}
\usepackage{sml-highlight}
\usepackage{listings}
\usepackage{xcolor}
\usepackage{epstopdf}
\usepackage{inputenc}
\usepackage{comment}
\usepackage{url}
\usepackage{graphicx} 
\graphicspath{{Figures/}}
\usepackage{proof}
\usepackage{todonotes}
\usepackage{amssymb}
\usepackage{tikz}
\usetikzlibrary{automata, positioning, arrows}
\tikzset{every picture/.style={line width=0.5pt}} %set default line width to 0.75pt        

%%% Random Macros

\newcommand{\sem}[1]{\{\![#1]\!\}}
\newcommand{\eval}[2]{#1\!\!\downarrow_{#2}}

\newcommand{\mypar}[1]{\noindent{\bf #1}}

\newcommand{\state}{\mathsf{state}}

\newcommand{\proj}{\mathsf{proj}}
\newcommand{\projE}[2]{#1\downarrow_#2}

\newcommand{\marco}[1]{\textcolor{cyan}{{\footnotesize Marco: #1}}}


% Syntax
\newcommand{\interactBase}[2]{\role{#1}\rightarrow \{\role{#2}\}}
\newcommand{\interact}[2]{\interactBase{#1}{#2}\,\Sigma_{j\in J}\lambda_j: u_j.\ C_j}
\newcommand{\doubleand}{\text{\&\&}}

\newcommand{\command}[4] {\commandBase {#1} {#2} \Sigma_{i\in I}#3_i: #4_i}
\newcommand{\commandBase}[3] {[#1]\ #2\ \rightarrow #3}
\newcommand{\ifTE}[4]{\textsf{if } #1@\role {#2} \textsf{ then } {#3} \textsf{ else } {#4}}


\newcommand{\role}[1]{\mathsf{#1}}
\newcommand{\Var}{\mathsf{Var}}
\newcommand{\Val}{\mathsf{Val}}


\newcommand{\CEnd}{\mathbf{0}}
\newcommand{\pp}{|\!|}
\newcommand{\ppp}[1]{|[#1]|}
\newcommand{\cmd}[6]{[#1] #2 \rightarrow \{#3: #4 = #5\}_{#6}}



  
% Semantics
\newcommand{\red}[1]{\ \longrightarrow_{#1}\ } 
\newcommand{\prismred}[1]{\ \stackrel{#1}{\rightsquigarrow}\ } 
\newcommand{\lab}[3]{\ [#1][#2:\;#3]\ } 

\definecolor{pblue}{rgb}{0.13,0.13,1}
\definecolor{pgreen}{rgb}{0,0.5,0}
\definecolor{pred}{rgb}{0.9,0,0}
\definecolor{pgrey}{rgb}{0.46,0.45,0.48}
\definecolor{keywordColor}{rgb}{0.50,0.00,0.33}
\definecolor{stringColor}{rgb}{0.16,0.00,1.00}
\definecolor{lineNumberColor}{rgb}{0.47,0.47,0.47}

\lstdefinelanguage{Eclipse}{
language=Java,
tabsize=3,
showspaces=false,
captionpos=b,
breaklines=true,
extendedchars=true,
numbers=none,
postbreak=\raisebox{0ex}[0ex][0ex]{\ensuremath{\color{red}\hookrightarrow\space}},
  basicstyle=\ttfamily\tiny,
  emphstyle=\bfseries,
  keywordstyle=\color{keywordColor}\bfseries,
  commentstyle=\markupComments,
  stringstyle=\color{stringColor},
  numberstyle=\color{lineNumberColor}\tiny,
  morecomment=[s][\markupJavadocs]{/**}{*/}, % For Javadoc comments
  showstringspaces=false,
  morekeywords={and},
  numbers=left,
  mathescape=true
%  ,frame=lines%shadowbox%trBL
}

\definecolor{carminepink}{rgb}{0.92, 0.3, 0.26}
\definecolor{jesuscolour}{rgb}{0.3, 0.92, 0.26}
\newcommand{\AV}[1]{\todo[inline, color=carminepink]{\textbf{AV}:~#1}}
\newcommand{\MC}[1]{\todo[inline, color=jesuscolour]{\textbf{MC}:~#1}}

%%% Local Variables: 
%%% mode: latex
%%% TeX-master: "main"
%%% End:


\title{A Probabilistic Choreography Language for
  PRISM} 

\author{Marco Carbone\inst{1}\orcidID{0000-0001-9479-2632} \and
Adele Veschetti\inst{2}\orcidID{0000-0002-0403-1889}}

\authorrunning{M. Carbone and A. Veschetti}

\institute{IT University of Copenhagen \\\email{maca@itu.dk} \and
Technische Universit{\"a}t Darmstadt\\
\email{adele.veschetti@tu-darmstadt.de}}


\begin{document}
\maketitle
\begin{abstract}
  % This paper presents a unified framework for modeling and analyzing
  % concurrent probabilistic systems. We introduce a simpler semantics
  % for PRISM, enhancing its usability. Additionally, we define a
  % choreographic language with syntax and semantics tailored for
  % intuitive modeling. We establish the correctness of a projection
  % function translating choreographic models to PRISM-compatible
  % formats. Finally, we develop a compiler enabling seamless
  % translation of choreographic models to PRISM, facilitating
  % powerful analysis while maintaining expressive clarity. These
  % contributions bridge the gap between high-level modeling and
  % robust analysis in probabilistic systems.
  % 
  %
  We present a choreographic framework for modelling and analysing
  concurrent probabilistic systems based on the PRISM
  model-checker. This is achieved through the development of a
  choreography language, that is a specification language that allows
  to describe the desired interactions within a concurrent system from
  a global viewpoint. Employing choreographies provides a clear and
  comprehensive view of system interactions, enabling the discernment
  of process flow and detection of potential errors, thus ensuring
  accurate execution and enhancing system reliability. We equip our
  language with a probabilistic semantics and then define a formal
  encoding into the PRISM language and discuss its
  correctness. Properties of programs written in our choreographic
  language can be model-checked by the PRISM model-checker via their
  translation into the PRISM language.  Finally, we implement a
  compiler for our language and demonstrate its practical
  applicability via examples drawn from the use cases featured on the
  PRISM website.
\end{abstract}

\section{Introduction}
\input introduction

%\section{A Motivational Example}
%\input example

\section{Choreography Language}\label{sec:chor}
\input chor

\section{The PRISM Language}\label{sec:prism}
\input prism

\section{Projection}\label{sec:proj}
\input proj

\section{Implementation}
\input implementation
\section{Benchmarking}
\input tests

\section{Related Work and Discussion}
\input discussion

\begin{credits}
  \subsubsection{\ackname} The work is partially funded by
  H2020-MSCA-RISE Project 778233 (BEHAPI), by the ATHENE project
  "Model-centric Deductive Verification of Smart Contracts”, and by
  the Carlsberg Foundation.
\end{credits}


\newpage

\bibliographystyle{splncs04}
\bibliography{biblio}

% \newpage
% \appendix
% \input appendix

\end{document}

