%\documentclass[a4paper,UKenglish,cleveref, autoref, thm-restate]{lipics-v2021}
\documentclass[runningheads]{llncs}
\usepackage[T1]{fontenc}
\usepackage{wrapfig}
 \usepackage{mathtools}
\usepackage{sml-highlight}
\usepackage{listings}
\usepackage{xcolor}
\usepackage{epstopdf}
\usepackage{inputenc}
\usepackage{comment}
\usepackage{url}
\usepackage{graphicx} 
\graphicspath{{Figures/}}
\usepackage{proof}
\usepackage{todonotes}
\usepackage{amssymb}

%%% Random Macros

\newcommand{\role}[1]{\textsf{#1}}

\newcommand{\CEnd}{\mathbf{0}}
\newcommand{\pp}{|\!|}
\newcommand{\ppp}[1]{|[#1]|}
\newcommand{\cmd}[6]{[#1] #2 \rightarrow \{#3: #4 = #5\}_{#6}}

\newcommand{\sem}[1]{\{\![#1]\!\}}


\newcommand{\mypar}[1]{\noindent{\bf #1}}

\newcommand{\state}{\mathsf{state}}

\newcommand{\proj}{\mathsf{proj}}

\newcommand{\marco}[1]{\textcolor{cyan}{{\footnotesize Marco: #1}}}


% Syntax
\newcommand{\interact}[2]{\role{#1}\rightarrow \{#2\}\,\Sigma\{\lambda_j: x_j=E_j;\ C_j\}_{j\in J}}

\newcommand{\command}[4] {[#1] #2 \rightarrow \Sigma_{i\in I}\{#3_i: #4_i\}}
\newcommand{\commandBase}[3] {[#1] #2 \rightarrow #3}
\newcommand{\ifTE}[4]{\textsf{if } #1@\role {#2} \textsf{ then } {#3} \textsf{ else } {#4}}

%%% Local Variables: 
%%% mode: latex
%%% TeX-master: "main"
%%% End:


\title{A Probabilistic Choreography Language for
  PRISM} 

\author{Marco Carbone\inst{1}\orcidID{0000-1111-2222-3333} \and
Adele Veschetti\inst{2}\orcidID{1111-2222-3333-4444}}

\authorrunning{M. Carbone and A. Veschetti}

\institute{IT University of Copenhagen \\\email{maca@itu.dk} \and
Technische Universit{\"a}t Darmstadt\\
\email{adele.veschetti@tu-darmstadt.de}}


\begin{document}
\maketitle
\begin{abstract}
  This paper presents a unified framework for modeling and analyzing concurrent probabilistic systems. We introduce a simpler semantics for PRISM, enhancing its usability. Additionally, we define a choreographic language with syntax and semantics tailored for intuitive modeling. We establish the correctness of a projection function translating choreographic models to PRISM-compatible formats. Finally, we develop a compiler enabling seamless translation of choreographic models to PRISM, facilitating powerful analysis while maintaining expressive clarity. These contributions bridge the gap between high-level modeling and robust analysis in probabilistic systems.
    
    \keywords{First keyword  \and Second keyword \and Another keyword.}
\end{abstract}

\section{Introduction}
\input introduction

\section{The Prism Language}
\input prism

\section{Choreographic Language}
\input chor


\newpage
\section{Benchmarking}
\input tests

\section{Discussion}
\input discussion

\begin{credits}
    \subsubsection{\ackname} The work is partially funded by H2020-MSCA-RISE Project 778233 (BEHAPI) and by the ATHENE project "Model-centric Deductive Verification of Smart Contracts”.
\end{credits}
\bibliographystyle{splncs04}
\bibliography{biblio}

\newpage
\appendix
\input appendix

\end{document}

